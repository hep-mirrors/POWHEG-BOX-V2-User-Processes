\documentclass[paper]{JHEP3}
\usepackage{amsmath,amssymb,enumerate,url}
\bibliographystyle{JHEP}
\usepackage{lineno}
%\linenumbers

%%%%%%%%%% Start TeXmacs macros
\newcommand{\tmtextit}[1]{{\itshape{#1}}}
\newcommand{\tmtexttt}[1]{{\ttfamily{#1}}}
\newenvironment{enumeratenumeric}{\begin{enumerate}[1.] }{\end{enumerate}}
\newcommand\sss{\mathchoice%
{\displaystyle}%
{\scriptstyle}%
{\scriptscriptstyle}%
{\scriptscriptstyle}%
}

\newcommand\as{\alpha_{\sss\rm S}}
\newcommand\POWHEG{{\tt POWHEG}}
\newcommand\POWHEGBOX{{\tt POWHEG BOX}}
\newcommand\HERWIG{{\tt HERWIG}}
\newcommand\PYTHIASIX{{\tt PYTHIA6}}
\newcommand\PYTHIAEIGHT{{\tt PYTHIA8}}
\newcommand\PYTHIA{{\tt PYTHIA}}
\newcommand\PHOTOS{{\tt PHOTOS}}
\newcommand\PHOTOSPP{{\tt PHOTOS++}}
\newcommand\MCatNLO{{\tt MC@NLO}}
\newcommand\DIR{{\tt W\_ew-BMNNP}}

\newcommand\kt{k_{\sss\rm T}}

\newcommand\pt{p_{\sss\rm T}}
\newcommand\LambdaQCD{\Lambda_{\scriptscriptstyle QCD}}
%%%%%%%%%% End TeXmacs macros


\title{The  $\boldsymbol{W^\pm}$ EW NLO \& QCD production \\
in the POWHEG BOX-V2 user manual~\footnote{Upgraded version prepared 
with the collaboration of Mauro Chiesa and Homero Martinez.} } \vfill
\author{Luca Barz\`e\\
  PH-TH Department, CERN, CH-1211 Geneva 23, Switzerland\\
  E-mail: \email{Luca.Barze@cern.ch}}
\author{Guido Montagna\\
  Dipartimento di Fisica Nucleare e Teorica, Universit\`a di Pavia and INFN, Sezione di Pavia, Via A. Bassi 6, 27100 Pavia, Italy\\
  E-mail: \email{Guido.Montagna@pv.infn.it}}
\author{Paolo Nason\\
  INFN, Sezione di Milano-Bicocca,
  Piazza della Scienza 3, 20126 Milan, Italy\\
  E-mail: \email{Paolo.Nason@mib.infn.it}}
\author{Oreste Nicrosini\\
  INFN, Sezione di Pavia, Via A. Bassi 6, 27100 Pavia, Italy\\
  E-mail: \email{Oreste.Nicrosini@pv.infn.it}}
\author{Fulvio Piccinini\\
  INFN, Sezione di Pavia, Via A. Bassi 6, 27100 Pavia, Italy\\
  E-mail: \email{Fulvio.Piccinini@pv.infn.it}}

\vskip -0.5truecm

\keywords{POWHEG, Shower Monte Carlo, NLO, Electroweak}

\abstract{This note documents the use of the package
  \tmtexttt{POWHEG-BOX-V2} for $W^\pm$ production processes including
  QCD and ElectroWeak NLO corrections.
  Results can be easily interfaced to shower Monte Carlo programs, in
  such a way that both NLO and shower accuracy are maintained.}
\preprint{\today\\ \tmtexttt{POWHEG-BOX-V2}}

\begin{document}


\section{Introduction}

The \tmtexttt{POWHEG BOX} program is a framework for implementing NLO
calculations in Shower Monte Carlo programs according to the
\POWHEG{} method. An explanation of the method and a discussion of
how the code is organized can be found in
Refs.~\cite{Nason:2004rx,Frixione:2007vw,Alioli:2010xd}. The code is
distributed according to the ``MCNET GUIDELINES for Event Generator
Authors and Users'' and can be found at the web page \\
%
\begin{center}
 \url{http://powhegbox.mib.infn.it}.
\end{center}
%
~\\
%
This program is an implementation of the Drell-Yan NLO cross sections
$pp\to W \to \ell \nu$ including QCD and ElectroWeak (EW) 
radiative corrections. 
A detailed description of the implementation can be found in 
Ref.~\cite{Barze:2012tt}. 
%Please cite the paper when you use the program. \\
In order to run the \tmtexttt{POWHEG BOX} program, we recommend the
reader to start from the \tmtexttt{POWHEG BOX} user manual, which
contains all the information and settings that are common between
all subprocesses. In this note we focus on  the settings and
parameters specific to the $W$ implementation.

\section{Generation of events}
\label{sec:gen_events}

Build the executable\\

\noindent\tmtexttt{\$ cd POWHEG-BOX-V2/\DIR{} \\
\$ make pwhg\_main }\\

Then do (for example) \\

\noindent\tmtexttt{\$
cd runtest-lhc-8TeV\\
\$ ../pwhg\_main}\\

At the end of the run, the file \tmtexttt{pwgevents.lhe} will contain
100000 events for $W^+~\to~e^+~\nu_e$  in the Les Houches format. 
In order to shower them with \PYTHIASIX{} or \PYTHIAEIGHT{} you 
must have \PHOTOS{} compiled (even if you do not use it, it must be linked in). 
The minimal procedure to do this is:
\\~\\
\noindent\tmtexttt{\$ 
cd PHOTOS \\ 
\$./configure --without-hepmc  \\ 
\$ make}
\\
\\
Then, in order to shower them with \PYTHIASIX{} do\\~\\
\tmtexttt{\$
cd POWHEG-BOX-V2/\DIR{}\\ \$ make
main-PYTHIA-lhef \\ \$ cd
runtest-lhc-8TeV \\
\$ ../main-PYTHIA-lhef}
\\

If you prefer to shower the events with \PYTHIAEIGHT{} do\\~\\
\tmtexttt{\$
cd POWHEG-BOX-V2/\DIR{} \\ \$ make
main-PYTHIA8-lhef \\ \$ cd
runtest-lhc-8TeV \\
\$ ../main-PYTHIA8-lhef}
\\

The executables \tmtexttt{main-PYTHIA-lhef} and \tmtexttt{main-PYTHIA8-lhe} are ``interfaces" that process the events in the file \tmtexttt{pwgevents.lhe} and give them as inputs to the parton shower, performing the required vetoes and setting the required flags of the shower programs, in a manner consistent with the physical accuracy of the input events. The output of the shower can be analyzed looking at the histograms in the \tmtexttt{.top}. file. The histograms and cuts can be customized editing the file \verb+pwhg_analysis.f+.

\PYTHIA{} is used to perform the QCD shower. The QED shower can be done using \PYTHIA{} or the independent program \PHOTOS{}~\cite{Golonka:2005pn}. The external source codes required by \tmtexttt{main-PYTHIA-lhef} are the Fortran codes of \PYTHIASIX{} and \PHOTOS{}. They are included in the \POWHEGBOX{} and in the \DIR{} folder, respectively. 

On the other hand, the interface \tmtexttt{main-PYTHIA8-lhef} requires the linking to the external libraries \PYTHIAEIGHT{} and \PHOTOSPP{} (both programs written in C++). \PHOTOSPP{} version 3.56 is included in the \DIR{} package, and can be compiled doing:\\

\tmtexttt{
\noindent \$ cd POWHEG-BOX-V2/\DIR{}/PHOTOS \\ 
\$ configure --without-hepmc \\
\$ make} \\

The flag \verb+PHOTOSCC_LOCATION+ must be set in the \tmtexttt{Makefile} to the path of installation of \PHOTOSPP{}. This path can be the folder \tmtexttt{\DIR{}/PHOTOS} if the user wants to use the included version of \PHOTOSPP{}, otherwise it should be set to the folder of an external installation. The interface has been tested with \PHOTOSPP{} version 3.56.

\PYTHIAEIGHT{} has to be downloaded and compiled by the user. The interface is designed to work with versions of \PYTHIAEIGHT{} up to 8.186 (an interface to {\tt PYTHIA8.2} is in progress). 
The script \tmtexttt{pythia8-config} should configure automatically the path in the \tmtexttt{Makefile}, if this is not the case, the user must set the \verb+PYTHIA8LOCATION+ flag to the correct path of installation of \PYTHIAEIGHT{}. 

Once \PHOTOSPP{} and \PYTHIAEIGHT{} are compiled and the flags are set properly in the \tmtexttt{Makefile}, the interface \tmtexttt{main-PYTHIA8-lhef} should compile. Then, before running it, the path to \PHOTOSPP{} libraries needs to be added to the list of dynamically linked libraries, and a variable pointing to the path of \PYTHIAEIGHT{} particle data (\tmtexttt{.xml}) files) needs to be set. In order to do this, the script \tmtexttt{setlibrarypaths.sh} must be edited to point to the correct paths, and then executed in the current shell, doing:
\\~\\
\tmtexttt{\$ source setlibrarypaths.sh }\\

Another interface is provided which processes the \POWHEG{} generated events, calls the QED final state shower implemented by \PHOTOSPP{} and generates a new event file in LHE format, that can be then interfaced to a QCD shower program, where QED radiation must
be switched off to avoid double counting. To compile and execute this interface, do:
\\~\\
\noindent\tmtexttt{\$
cd POWHEG-BOX-V2/\DIR{} \\ 
\$ make main-PHOTOS-lhef \\ \$ cd
runtest-lhc-8TeV \\
\$ ../main-PHOTOS-lhef}
\\

The compilation requires the setting of the flag \verb+PHOTOSCC_LOCATION+ in the \tmtexttt{Makefile}, and the \PHOTOSPP{} path as environmental variable, as explained above.

%There is the possibility to set the flag \verb+HEPMCLOCATION+ in the \tmtexttt{Makefile}, in case the user wants to link to \tmtexttt{HEPMC} libraries. However, they are in principle not needed by the executables mentioned before, so this flag can be left empty.

\section{Process specific input parameters}

All the parameters and flags are set in the input card file \tmtexttt{powheg.input}. The mandatory parameters are those needed to select the final state
leptonic species coming from the vector-boson:
~\\
\tmtexttt{
  idvecbos 24 \phantom{a} ! PDG code for vector boson to be produced\\
  \phantom{aaaaaaaaaaaaa} ! ( W+: 24 W-: -24 )\\
  vdecaymode 11 !  code for selected W decay\\
  \phantom{aaaaaaaaaaaaa} ! (11(-11): electronic; 13(-13): muonic; 15(-15): tauonic)}
\\
In addition to the mandatory parameters, the \POWHEGBOX{} input allows for an easy 
setting of EW and run parameters, by explicitly adding the relevant 
lines to the input card. 
If one of the following entries is not present in the input card  the reported default 
value is assumed. In any case, these parameters are printed in the output of the program, so their values can be easily tracked down. 
~\\~\\
 \tmtexttt{
   Wmass\phantom{aa}  80.398 \phantom{aaaaaaa} ! W mass in GeV\\
   Wwidth\phantom{a} 2.141 \phantom{aaaaaaaa} ! W width in GeV\\
   Zmass\phantom{aa}  91.1876 \phantom{aaaaaa}   ! Z mass in GeV\\
   Zwidth\phantom{a} 2.4952 \phantom{aaaaaaa} ! Z width in GeV\\
   alphaem 0.00729735254 \phantom{a}! em coupling alpha(0)\\
   gmu\phantom{aaaa} 1.16637d-5 \phantom{aaa} ! Fermi constant in GeV\^{}-2\\
   Hmass\phantom{aa}  120. \phantom{aaaaaaaa} ! Higgs mass in GeV\\
   Tmass\phantom{aa} 172.9 \phantom{aaaaaaaa} ! Top mass in GeV\\
   Bmass\phantom{aa} 4.6 \phantom{aaaaaaaaaa} ! B quark mass in GeV\\
   Cmass\phantom{aa} 1.2 \phantom{aaaaaaaaaa} ! C quark mass in GeV\\
   Smass\phantom{aa} 0.15 \phantom{aaaaaaaaa} ! S quark mass in GeV\\
   Umass\phantom{aa} 0.06983 \phantom{aaaaaa} ! U quark mass in GeV\\
   Dmass\phantom{aa} 0.06984 \phantom{aaaaaa} ! D quark mass in GeV\\
   Elmass\phantom{a} 0.005109989 \phantom{aa} ! Electron mass in GeV\\
   Mumass\phantom{a} 0.105658369 \phantom{aa} ! Mu mass in GeV\\
   Taumass\phantom{} 1.77699 \phantom{aaaaaa} ! Tau mass in GeV\\
}
\\
The following parameter limits from below the virtuality of the $W$ boson:
\\
\tmtexttt{mass\_low 1 \phantom{aaaaaaaa}! W virtuality > mass\_low in GeV}\\
\\
If absent, it is set to 1~GeV. 
In order to avoid edge effects, the lower limit \tmtexttt{mass\_low} 
should be more inclusive w.r.t. cuts applied at the analysis level.
Notice that, if photons are generated, the $W$ virtuality is not
necessarily the mass of the charged lepton neutrino system.
\\
\\
\tmtexttt{runningscale 0 \phantom{aaaa}! choice for ren and fac scales in Bbar integration}\\
\phantom{pppppppppppppppppppppppppppppp} \tmtexttt{ 0: fixed scale M\_W}\\
\phantom{pppppppppppppppppppppppppppppp} \tmtexttt{ 1: running scale 
$l \nu (\gamma)$ inv mass}\\ 
\phantom{pppppppppppppppppppppppppppppppppp} 
\tmtexttt{ $\gamma$ included with QED FSR }\\ 
\tmtexttt{scheme 1! choice for EW NLO scheme calculation}\\
\phantom{pppppppppppppppppppppppppppppp} \tmtexttt{ 0: Alpha(0)}\\
\phantom{pppppppppppppppppppppppppppppp} \tmtexttt{ 1: G\_$\mu$}\\ 
\tmtexttt{
   CKM\_Vud   0.975  ! Entries of CKM mixing matrix\\
   CKM\_Vus   0.222  \\
   CKM\_Vub   1d-10 \\
   CKM\_Vcd   0.222 \\
   CKM\_Vcs   0.975 \\
   CKM\_Vcb   1d-10 \\
   CKM\_Vtd   1d-10 \\
   CKM\_Vts   1d-10 \\
   CKM\_Vtb   1.0\\}
\\
The EW radiative corrections can be calculated according to two 
different schemes: the $\alpha(0)$ scheme, where the input parameters 
are $\alpha(0)$, $M_W$ and $M_Z$; the $G_\mu$ scheme, where the 
input parameters are $G_\mu$, $M_W$ and $M_Z$. The latter one 
(default in the code) is preferred because it minimizes 
the EW corrections and the uncertainties due to the light quark masses. 
\\
The EW corrections can be switched off by setting
\\
\tmtexttt{no\_ew 1       ! default 0}
\\
and the strong corrections can be switched off by setting
\\
\tmtexttt{no\_strong 1       ! default 0}
\\
This last option is just to check EW corrections at the NLO level (i.e.,
the Les Houches events do not have much meaning).
\\
\\
The program can be interfaced to both \PYTHIASIX{} and  \PYTHIAEIGHT{}, as explained in section~\ref{sec:gen_events}. In order to switch off photon radiation from \PYTHIA{} and use \PHOTOS{} instead, use the setting:
\\
\tmtexttt{use\_photos 1   ! default 0}
\\
in the \tmtexttt{powheg.input} file.
\\
For photon final state radiation a comment is in order. According to 
the \POWHEG{} method, the radiation by the shower has to be generated 
from a starting scale given by the hardest $\pt$ tried at the matrix element 
level (the variable {\tt scalup} written in the event file 
\tmtexttt{pwgevents.lhe}). This is true also in the case that both 
QCD and QED radiation are present, as detailed in 
Ref.~\cite{Barze:2012tt} . Both \PYTHIASIX{} and \PYTHIAEIGHT{} 
do not use this starting scale for the generation of QED final state radiation 
from the $W$. Hence, in order to avoid double counting of QED radiation, 
a veto algorithm is necessary. The same problem is present also using 
\PHOTOS{}. 
This algorithm 
is provided automatically in the files \tmtexttt{main-PYTHIA.f}, \tmtexttt{main-PYTHIA8.f} and 
\tmtexttt{scalupveto.f}. For the case of the interface to \PYTHIAEIGHT{}, the user can optionally 
adopt the internal matching algorithm of \PYTHIAEIGHT{}, which is switched 
on by setting the flag:
\\
\tmtexttt{py8veto 1   ! default 0}
\\
in the \tmtexttt{powheg.input} file.
\\
\\
A general issue is the matching between the NLO calculation and the 
(QCD and QED) higher order corrections given by the parton shower: 
due to the different definitions of $p_\perp$ in \POWHEG{} and 
\PYTHIAEIGHT{}/\PYTHIASIX{}, some double counting or dead zone 
can arise. In \PYTHIAEIGHT{} the default is to generate all QCD/QED 
shower emissions up to the kinematical limit and then veto emissions 
harder than the \POWHEG{} emission, according to the \POWHEG{} 
$p_\perp$ definition. This is done, as default, by means of the provided 
\tmtexttt{class PowhegHooks}. With the provided \PYTHIAEIGHT{} interface 
the user can optionally choose an alternative scheme, where 
the shower starting scale is fixed to {\tt scalup} and no veto 
is performed. This choice can be activated by setting 
\\
\tmtexttt{veto1 1   ! default 0}
\\
in the \tmtexttt{powheg.input} file.
\\
In this case, if the QED higher order radiation is handled by \PYTHIAEIGHT{}, 
also the QED starting scale is set to {\tt scalup} through the 
\tmtexttt{class MyUserHooks}. 
\\ 
\\
Additional flags available for the \PYTHIAEIGHT{} interface 
are the following: 
\\
\tmtexttt{noQEDq 1   ! default 0}, which allows to switch off QED radiation 
from quarks; 
\\
\tmtexttt{pytune xx  ! default 5}, which allows to change the tune; 
\\
\tmtexttt{nohad 1  ! default 0}, which allows to switch off the hadronization. 
\\
In the provided interface to \PYTHIAEIGHT{} 
the decay of hadronic resonances which can proceed radiatively has been 
suppressed. In order to let the resonances decay, the user should 
open the file \tmtexttt{pythia8F77.cc} and comment the relevant lines 
in {\tt pythia\_init}.
\\
\\
For further customization of the settings used by the shower interfaces, beyond the flags available in \tmtexttt{powheg.input}, the user can modify the following source code files:

\begin{itemize}
\item Interface \tmtexttt{main-PYTHIA-lhef}: Settings in files \tmtexttt{setup-PYTHIA-lhef.f} and \tmtexttt{photos.f}.
\item Interface \tmtexttt{main-PYTHIA8-lhef}: Settings in file \tmtexttt{pythia8F77.cc}.
\item Interface \tmtexttt{main-PHOTOS-lhef}: Settings in file \tmtexttt{photosCCF.cc}.
\end{itemize}


\section{Generation of a sample with $W^+$ and $W^-$
  events}\label{sec:merging}

In case the user is interested in the generation of a sample where
both $W^+$ and $W^-$ events appear, a script and a dedicated
executable have been included. The script is named
\tmtexttt{merge\_wp\_wm.sh} and can be found in the directory
\tmtexttt{W/testrun-merge}. It can be run in any subfolder of
\tmtexttt{W} however. Three inputs are mandatory: the first two are
the prefixes of the input files used to generate $W^+$ and $W^-$
events. The third input has to be an integer and correspond to the
total number of events that the final \emph{merged} sample will
contain. The script has to be run twice, using a positive integer
value at the first call and its opposite afterwards.  Therefore, for
example, to produce a sample of 10000 events, starting from the input
files \tmtexttt{wp-powheg.input} and \tmtexttt{wm-powheg.input}, the
invocation lines should be as follows:~\\~\\ \tmtexttt{\$ sh
  merge\_wp\_wm.sh wp wm 10000}\\~\\ and then~\\~\\ \tmtexttt{\$ sh
  merge\_wp\_wm.sh wp wm -10000}\\~\\

Few remarks are needed:
\begin{itemize}
\item it is responsibility of the user to check that the 2 input files
  are equal. The \tmtexttt{idvecbos}  and \tmtexttt{vdecaymode} tokens have to be different,
  obviously.
\item the two values of \tmtexttt{numevts} are not really used: the
  program re-calculate the needed values as a function of the $W^+$ and
  $W^-$ cross sections and of the total number of events to be
  generated.
\item the final event file is always named
  \tmtexttt{wp\_wm\_sample-events.lhe}. In the header section it also
  contains a copy of the two input files used to generate it, for
  cross-checking purposes
\end{itemize}


\begin{thebibliography}{10}

\bibitem{Nason:2004rx}
  P.~Nason,
  ``A new method for combining NLO QCD with shower Monte Carlo algorithms,''
  JHEP {\bf 0411} (2004) 040
  [arXiv:hep-ph/0409146].
  %%CITATION = JHEPA,0411,040;%%

%\cite{Frixione:2007vw}
\bibitem{Frixione:2007vw}
  S.~Frixione, P.~Nason and C.~Oleari,
``Matching NLO QCD computations with Parton Shower simulations: the POWHEG
method,''
  JHEP {\bf 0711} (2007) 070
  [arXiv:0709.2092 [hep-ph]].
  %%CITATION = JHEPA,0711,070;%%

%\cite{Alioli:2010xd}
\bibitem{Alioli:2010xd}
  S.~Alioli, P.~Nason, C.~Oleari and E.~Re,
``A general framework for implementing NLO calculations in shower Monte Carlo
  programs: the POWHEG BOX,''
  [arXiv:1002.2581 [hep-ph]].
  %%CITATION = ARXIV:1002.2581;%%

\bibitem{Barze:2012tt}
  L.~Barz\`e, G.~Montagna, P.~Nason, O.~Nicrosini and F.~Piccinini,
  ``Implementation of electroweak corrections in the POWHEG BOX: single W production,''
  [arXiv:1202.0465 [hep-ph]]

%\cite{Golonka:2005pn}
\bibitem{Golonka:2005pn}
  P.~Golonka and Z.~Was,
  %``PHOTOS Monte Carlo: A Precision tool for QED corrections in $Z$ and $W$ decays,''
  Eur.\ Phys.\ J.\ C {\bf 45} (2006) 97
  [hep-ph/0506026].
  %%CITATION = HEP-PH/0506026;%%

\end{thebibliography}

\end{document}





