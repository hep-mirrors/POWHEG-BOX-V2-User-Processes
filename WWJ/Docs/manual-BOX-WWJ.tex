\documentclass{article}
\usepackage[english]{babel}
\usepackage{amsmath,hyperref}

%%%%%%%%%% Start TeXmacs macros
\newcommand{\tmop}[1]{\ensuremath{\operatorname{#1}}}
\newcommand{\tmstrong}[1]{\textbf{#1}}
\newcommand{\tmtextbf}[1]{{\bfseries{#1}}}
\newcommand{\tmtextit}[1]{{\itshape{#1}}}
\newcommand{\tmtexttt}[1]{{\ttfamily{#1}}}
\newcommand{\tmverbatim}[1]{{\ttfamily{#1}}}
\newcommand{\noun}[1]{\textsc{#1}}


\newcommand{\WWJMINLO}{\noun{WWj-Minlo}}
\newcommand{\WWJ}{\noun{WWj}}
\newcommand{\MINLO}{\noun{Minlo}}

%%%%%%%%%% End TeXmacs macros


\begin{document}

\title{The POWHEG-BOX-V2/WWJ manual}

\maketitle

\section{Introduction}

The POWHEG-BOX \WWJ{} program {\cite{Hamilton:2016bfu}} can be used to generate the
QCD production of $W^+ W^-$ + 1~jets events in hadronic collisions,
with the $W$-bosons decaying into leptons or hadrons, to NLO accuracy
in QCD, in such a way that matching with a shower program is
possible. In case of decays into hadrons, NLO corrections to the decay
processes are not included. This is unlikely to be necessary: most
shower Monte Carlo already do a good job in dressing the $W$ decay
with QCD radiation, since $W$ hadronic decays have been fit to LEP2
data. The effect of off-shell singly resonant graphs is fully
included. The CKM matrix is by default the Cabibbo matrix. The
calculation is performed in the four-flavour scheme. Therefore it's
mandatory to use a four-flavour PDF, as reminded in the template input
cards.

If the $W$-bosons decay into leptons of the same flavour (e.g. $e^+ e^- \nu_e
\bar{\nu}_e$), then the $\tmop{ZZJ}$ production of this signal should be
considered separately. Interference between
these two processes is negligible (see ref.~\cite{Melia:2011tj}, where the same interference is considered without the extra jet) and is not included. \
This document describes the input parameters that are specific to this
implementation. The parameters that are common to all \tmtexttt{POWHEG BOX}
implementation are given in the \tmtexttt{POWHEG-BOX-V2/Docs} directory. 

When the \MINLO{} option is switched on, the \WWJ{} generator becomes
NLO accurate also for inclusive $W^+ W^-$ production. 


\section{Generation of events}

In the WWJ directory do\\~\\
\tmtexttt{\$ make pwhg\_main}\\~\\
Then do (for example)\\~\\
\tmtexttt{\$ cd testrun-lhc}\\
\tmtexttt{\$ ../pwhg\_main}\\~\\
At the end of the run, the file \tmtexttt{pwgevents.lhe} will contain events
for $W$-pair production in association with one jet in the Les Houches format. In order to shower them
with \tmtexttt{PYTHIA}:\\~\\
\tmtexttt{\$ make main-PYTHIA-lhef}\\
\tmtexttt{\$ cd test}\\
\tmtexttt{\$ ../main-PYTHIA-lhef}\\~

However, because the program is numerically intensive, we do not
recommend to run it without the \tmtexttt{POWHEG} parallel-feature
version switched on, as described in detail in ref.~\cite{POWHEG}.  A
template input card can be found in the
\tmtexttt{testrun-wwj-parallel} directory. In the
\tmtexttt{testrun-minlo-parallel} directory, instead, we provide a
template input card to perform a run with the \MINLO{} option
activated.

\section{Input parameters}

Parameters in \tmtexttt{powheg.input} that are specific to $W W$ pair
production in association with one jet:\\~

\begin{tabular}{lll}
%\tmtexttt{bornsuppfact 0 } & \tmtexttt{!}  & \tmtexttt{(default 0), if 1 include a born}\\
%& \tmtexttt{!} & \tmtexttt{suppression factor}\\
\hspace{-0.5cm}\tmtexttt{runningscale 0 } & \tmtexttt{!} & \tmtexttt{(default 0), 0 = fixed scale} \\
\hspace{-0.5cm}\tmtexttt{minlo 0 } &\tmtexttt{!}  & \tmtexttt{(default 0) if 1 turn on the \MINLO{} option} \\~\\
%\tmtexttt{ph\_Wmass2low 20 } & \tmtexttt{!}  & \tmtexttt{(default 20) minimal  mass of each $W$ system}  \\
%\tmtexttt{ph\_Wmass2high 200 } & \tmtexttt{!}  & \tmtexttt{(default 200) maximal  mass of each $W$ system } \\
\end{tabular}

%!OLD
%!\noindent
%!\tmtexttt{bornsuppfact \ \ \ \ \ \ 0 \ \ \ \ \ ! (default 0), if 1 include a born \\
%!\phantom{1}\hspace{3.3cm}  !  suppression factor {\bf tested, code needs fixing??}} \\
%!\tmtexttt{runningscale \ \ 0 \ \ \ \ \ ! (default 0, i.e. set the scale to 2$M_W$) if 1 use mu =M\_(WW)}, if \MINLO{} is turned one, the scale is set with the \MINLO{} prescription\\
%!\tmtexttt{minlo \ \ \ \ \ \ \ 0 \ \ \ \ \ ! (default 0) if 1 turn on the \MINLO{} option} \\
%!
%!
%!\tmtexttt{dronly \ \ \ \ \ \ 0 \ \ \ \ \ ! (default 0), if 1 include only double\\
%!\phantom{1}\hspace{3.3cm} ! resonant contributions\\
%!zerowidth \ \ \ 0 \ \ \ \ \ ! (default 0), if 1 use on-shell $W$-bosons only}
%!\\
%!\tmtexttt{runningwidth 0 \ \ \ \ \ ! (default 0), if 1 use running width} \ \
%!\\

%If \tmtexttt{zerowidth} is absent or equal to zero, the $W$-boson's are given
%finite width. Singly resonant graphs are also included by default, unless the
%\tmtexttt{dronly} flag is set to 1. If zerowidth is set to true, dronly is set
%to true regardless of what is in the powheg.input file. Dynamic widths can be
%used by setting the \tmtexttt{runningwidth} flag to 1. 

%If the flag \tmtexttt{fixedscale} is set equal to 1, then the factorization
%and renormalization scales are fixed at $M_W$. Otherwise, a dynamic scale of
%the mass of the $W$-boson pair will be used.

\noindent Several decay modes can be selected by an approriate flag in the
\tmtexttt{powheg.input} file:\\~\\
\begin{tabular}{lll}
%\tmtexttt{semileptonic 1 ! one W goes to hadrons, one goes to leptons}\\
\tmtexttt{e+e- 1} &\tmtexttt{!}& \tmtexttt{only electrons}\\
\tmtexttt{mu+mu- 1}&\tmtexttt{!} &\tmtexttt{only muons}\\
\tmtexttt{tau+tau- 1} &\tmtexttt{!} &\tmtexttt{only taus}\\
\tmtexttt{leptonic\_notau 1 } &\tmtexttt{!} &\tmtexttt{both W's go into leptons (but not $\tau$)}\\
\tmtexttt{leptonic 1} &\tmtexttt{! } &\tmtexttt{both W's go into leptons}\\
\tmtexttt{hadronic 1 } &\tmtexttt{!} &\tmtexttt{both W's go into hadrons}\\
\tmtexttt{semileptonic 1 } &\tmtexttt{!} &\tmtexttt{one W goes into hadrons, one into leptons}\\
\tmtexttt{semileptonic\_notau 1} &\tmtexttt{!} &\tmtexttt{one W goes into hadrons, one into leptons (but not $\tau$)}\\
\tmtexttt{e+mu- 1} &\tmtexttt{!} &\tmtexttt{$W^+$ decays to electrons, $W^-$ to muons}\\
\tmtexttt{mu+e- 1 } &\tmtexttt{!} &\tmtexttt{$W^-$ decays to electrons, $W^+$ to muons}\\~\\
\end{tabular}

More conditions can be easily added, by editing the
\tmtexttt{alloweddec} function in the \tmtexttt{init\_processes.f}
file. If no condition is specified in the input card, the default
decay channel is assumed, namely \tmtexttt{e+mu-}.

In the case of leptonic final states, we release a template analysis
\tmtexttt{pwhg\_analysis-WW-template.f} with a set of differential
distributions.

As a final remark, we note that in ref.~\cite{Hamilton:2016bfu} we
found that closed fermion loops slow down the calculation
considerably, yet provide no sizable effect in any distribution that
we considered (within our numerical accuracy). Hence, we also provide
the possibility to run the code without including closed fermion
loops. This can be achieved by setting the variable
\tmtexttt{GOSAMDIR} to \tmtexttt{GoSamlib\_nofboxes} in the Makefile,
and recompiling the code from scratch.

\begin{thebibliography}{1}

%\cite{Hamilton:2016bfu}
\bibitem{Hamilton:2016bfu}
  K.~Hamilton, T.~Melia, P.~F.~Monni, E.~Re and G.~Zanderighi,
  %``Merging WW and WW+jet with MINLO,''
  arXiv:1606.07062 [hep-ph].
  %%CITATION = ARXIV:1606.07062;%%

%\cite{Melia:2011tj}
\bibitem{Melia:2011tj}
  T.~Melia, P.~Nason, R.~Rontsch and G.~Zanderighi,
  %``W+W-, WZ and ZZ production in the POWHEG BOX,''
  JHEP {\bf 1111} (2011) 078
  doi:10.1007/JHEP11(2011)078
  [arXiv:1107.5051 [hep-ph]].
  %%CITATION = doi:10.1007/JHEP11(2011)078;%%
  %171 citations counted in INSPIRE as of 17 Jun 2016
  
\bibitem{POWHEG}
POWHEG-BOX-V2/Docs/V2-paper.pdf
\end{thebibliography}

\end{document}
