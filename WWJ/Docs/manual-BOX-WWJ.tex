\documentclass{paper}
\usepackage[english]{babel}
\usepackage{amsmath,hyperref}

%%%%%%%%%% Start TeXmacs macros
\newcommand{\tmop}[1]{\ensuremath{\operatorname{#1}}}
\newcommand{\tmstrong}[1]{\textbf{#1}}
\newcommand{\tmtextbf}[1]{{\bfseries{#1}}}
\newcommand{\tmtextit}[1]{{\itshape{#1}}}
\newcommand{\tmtexttt}[1]{{\ttfamily{#1}}}
\newcommand{\tmverbatim}[1]{{\ttfamily{#1}}}
\newcommand{\noun}[1]{\textsc{#1}}

\newcommand{\WWJMINLO}{\noun{WWj-Minlo}}
\newcommand{\WW}{\noun{WW}}
\newcommand{\WWJ}{\noun{WWj}}
\newcommand{\MINLO}{\noun{Minlo}}
\newcommand{\NNLOPS}{\noun{NNLOPS}}

%%%%%%%%%% End TeXmacs macros


\begin{document}

\title{The POWHEG-BOX-V2/WWJ manual}

\maketitle

\section{Introduction}

The POWHEG-BOX \WWJ{} program {\cite{Hamilton:2016bfu}} can be used to
generate the QCD production of $W^+ W^-$ + 1~jets events in hadronic
collisions, with the $W$-bosons decaying into leptons or hadrons, to
NLO accuracy in QCD, in such a way that matching with a shower program
is possible. As in the \WW{} process, in the case of decays into
hadrons, NLO corrections to the decay processes are not included. This
is unlikely to be necessary: most shower Monte Carlo already do a good
job in dressing the $W$ decay with QCD radiation, since $W$ hadronic
decays have been fit to LEP2 data. The effect of off-shell singly
resonant graphs is fully included. The CKM matrix is by default the
Cabibbo matrix. The calculation is performed in the four-flavour
scheme. Therefore it is mandatory to use a four-flavour PDF, as
reminded in the template input cards.

If the $W$-bosons decay into leptons of the same flavour (e.g. $e^+
e^- \nu_e \bar{\nu}_e$), then the $\tmop{ZZJ}$ production of this
signal should be considered separately. Interference between these two
processes is negligible (see ref.~\cite{Melia:2011tj}, where the same
interference is considered without the extra jet) and is not
included. \ This document describes the input parameters that are
specific to this implementation. The parameters that are common to all
\tmtexttt{POWHEG BOX} implementation are given in the
\tmtexttt{POWHEG-BOX-V2/Docs} directory.

When the \MINLO{} option is switched on, the \WWJ{} generator becomes
NLO accurate also for inclusive $W^+ W^-$ production. Furthermore,
when the \NNLOPS{} option is also switched on, an on-the-fly
reweighting to NNLO (fully differential in the Born phase space) is
performed.

\section{Generation of events}

In the WWJ directory do\\~\\
\tmtexttt{\$ make pwhg\_main}\\~\\
Then do (for example)\\~\\
\tmtexttt{\$ cd testrun-lhc}\\
\tmtexttt{\$ ../pwhg\_main}\\~\\
At the end of the run, the file \tmtexttt{pwgevents.lhe} will contain events
for $W$-pair production in association with one jet in the Les Houches format. In order to shower them
with \tmtexttt{PYTHIA}:\\~\\
\tmtexttt{\$ make main-PYTHIA-lhef}\\
\tmtexttt{\$ cd test}\\
\tmtexttt{\$ ../main-PYTHIA-lhef}\\~

However, because the program is numerically intensive, we do not
recommend to run it without the \tmtexttt{POWHEG} parallel-feature
version switched on, as described in detail in ref.~\cite{POWHEG}.  A
template input card can be found in the \tmtexttt{testrun-lhc}
directory.

In the \tmtexttt{testrun-nnlops} directory, instead, we provide a
template input card to perform runs in parallel with the \MINLO{} (and
the \NNLOPS{}) option activated, and to obtain events with multiple
(scale variation) weights without the need of performing a reweighting
on a pre-existing event file.

\section{Input parameters}

Parameters in \tmtexttt{powheg.input} that are specific to $W W$ pair
production in association with one jet are listed in the following:\\~

\begin{tabular}{lll}
%\tmtexttt{bornsuppfact 0 } & \tmtexttt{!}  & \tmtexttt{(default 0), if 1 include a born}\\
%& \tmtexttt{!} & \tmtexttt{suppression factor}\\
  \hspace{-0.5cm}\tmtexttt{runningscale 0 } & \tmtexttt{!} & \tmtexttt{(default 0), 0 = fixed scale 2$M_W$,} \\
  & \tmtexttt{!} &  \tmtexttt{1 == $M(WW)$, 2 == $M_{T, W^+}+M_{T, W^-}$ } \\
\hspace{-0.5cm}\tmtexttt{minlo 0 } &\tmtexttt{!}  & \tmtexttt{(default 0) if 1 turn on the \MINLO{} option} \\~\\
%\tmtexttt{ph\_Wmass2low 20 } & \tmtexttt{!}  & \tmtexttt{(default 20) minimal  mass of each $W$ system}  \\
%\tmtexttt{ph\_Wmass2high 200 } & \tmtexttt{!}  & \tmtexttt{(default 200) maximal  mass of each $W$ system } \\
\hspace{-0.5cm}\tmtexttt{nnlops 0 } &\tmtexttt{!}  & \tmtexttt{(default 0) if 1 turn on the \NNLOPS{} option} \\~\\
\end{tabular}

Note that \MINLO{} overwrites any other running scale choice.  The
\NNLOPS{} works only if the \MINLO{} option is on.  Finally, if
\MINLO{} is switched off, one needs to use a Born suppression factor,
or, alternatively, a cut in the phase-space generation, in order to
make the cross-section finite. The latter cut can be set through the
token \tmtexttt{bornktmin}.

The \NNLOPS{} option is new in the \WWJ{} code and it allows to
reweight to the NNLO result on the fly. This option is not yet present
in other \NNLOPS{} codes (Higgs, Drell Yan, associated Higgs
production). The on-the-fly reweighting to the \WW{} NNLO computation
is performed using precomputed NNLO tables obtained with the MATRIX
code~\cite{Grazzini:2017mhc} and with
\WWJ-\MINLO{}~\cite{Hamilton:2016bfu}. The setup used to generate
these (NNLO and \MINLO{}) tables is described in detail in
ref.~\cite{Re:2018vac}. If one wishes to reweight using a different
PDF set, or a different collider energy, one might want to generate
new \MINLO{} and NNLO reweighting tables. Note however that the
reweighing factor should be relatively insensitive to input
parameters.

Note that, if one runs in \NNLOPS{} mode, then the directory where the
run is performed should contain the file \tmtexttt{binvalues-WW.top}
and the directories \tmtexttt{WW-MATRIX} and \tmtexttt{WW-MINLO}.

It is possible to obtain directly events with multiple weights
associated to scale variation without the need to reweight a
posteriori. For instance, if the \tmtexttt{powheg.input} file contains
the following lines:\\~

\tmtexttt{
rwl\_file '-'\\
<initrwgt>\\
<weightgroup name='First-Weights'>\\
<weight id='11'> renscfact=1.0 facscfact=1.0 </weight>\\
<weight id='12'> renscfact=1.0 facscfact=2.0 </weight>\\
<weight id='21'> renscfact=2.0 facscfact=1.0 </weight>\\
<weight id='22'> renscfact=2.0 facscfact=2.0 </weight>\\
<weight id='1H'> renscfact=1.0 facscfact=0.5 </weight>\\
<weight id='H1'> renscfact=0.5 facscfact=1.0 </weight>\\
<weight id='HH'> renscfact=0.5 facscfact=0.5 </weight>\\
</initrwgt>
}\\~

then the usual 7-point scale variation weights are produced. On top of
this, if the \NNLOPS{} option is on, the weights are doubled and the
last 7 weights are the \NNLOPS{} ones.


%!OLD
%!\noindent
%!\tmtexttt{bornsuppfact \ \ \ \ \ \ 0 \ \ \ \ \ ! (default 0), if 1 include a born \\
%!\phantom{1}\hspace{3.3cm}  !  suppression factor {\bf tested, code needs fixing??}} \\
%!\tmtexttt{runningscale \ \ 0 \ \ \ \ \ ! (default 0, i.e. set the scale to 2$M_W$) if 1 use mu =M\_(WW)}, if \MINLO{} is turned one, the scale is set with the \MINLO{} prescription\\
%!\tmtexttt{minlo \ \ \ \ \ \ \ 0 \ \ \ \ \ ! (default 0) if 1 turn on the \MINLO{} option} \\
%!
%!
%!\tmtexttt{dronly \ \ \ \ \ \ 0 \ \ \ \ \ ! (default 0), if 1 include only double\\
%!\phantom{1}\hspace{3.3cm} ! resonant contributions\\
%!zerowidth \ \ \ 0 \ \ \ \ \ ! (default 0), if 1 use on-shell $W$-bosons only}
%!\\
%!\tmtexttt{runningwidth 0 \ \ \ \ \ ! (default 0), if 1 use running width} \ \
%!\\

%If \tmtexttt{zerowidth} is absent or equal to zero, the $W$-boson's are given
%finite width. Singly resonant graphs are also included by default, unless the
%\tmtexttt{dronly} flag is set to 1. If zerowidth is set to true, dronly is set
%to true regardless of what is in the powheg.input file. Dynamic widths can be
%used by setting the \tmtexttt{runningwidth} flag to 1. 

%If the flag \tmtexttt{fixedscale} is set equal to 1, then the factorization
%and renormalization scales are fixed at $M_W$. Otherwise, a dynamic scale of
%the mass of the $W$-boson pair will be used.

~\\
Several decay modes can be selected by an appropriate flag
in the \tmtexttt{powheg.input} file:\\~\\
\begin{tabular}{lll}
%\tmtexttt{semileptonic 1 ! one W goes to hadrons, one goes to leptons}\\
\tmtexttt{e+e- 1} &\tmtexttt{!}& \tmtexttt{only electrons}\\
\tmtexttt{mu+mu- 1}&\tmtexttt{!} &\tmtexttt{only muons}\\
\tmtexttt{tau+tau- 1} &\tmtexttt{!} &\tmtexttt{only taus}\\
\tmtexttt{leptonic\_notau 1 } &\tmtexttt{!} &\tmtexttt{both W's go into leptons (but not $\tau$)}\\
\tmtexttt{leptonic 1} &\tmtexttt{! } &\tmtexttt{both W's go into leptons}\\
\tmtexttt{hadronic 1 } &\tmtexttt{!} &\tmtexttt{both W's go into hadrons}\\
\tmtexttt{semileptonic 1 } &\tmtexttt{!} &\tmtexttt{one W goes into hadrons, one into leptons}\\
\tmtexttt{semileptonic\_notau 1} &\tmtexttt{!} &\tmtexttt{one W goes into hadrons, one into leptons (but not $\tau$)}\\
\tmtexttt{e+mu- 1} &\tmtexttt{!} &\tmtexttt{$W^+$ decays to electrons, $W^-$ to muons}\\
\tmtexttt{mu+e- 1 } &\tmtexttt{!} &\tmtexttt{$W^-$ decays to electrons, $W^+$ to muons}\\~\\
\end{tabular}

More conditions can be easily added, by editing the
\tmtexttt{alloweddec} function in the \tmtexttt{init\_processes.f}
file. If no condition is specified in the input card, the default
decay channel is assumed, namely \tmtexttt{e+mu-}.


As a final remark, we note that in ref.~\cite{Hamilton:2016bfu} we
found that closed fermion loops slow down the calculation
considerably, yet provide no sizable effect in any distribution that
we considered (within our numerical accuracy). Hence, we also provide
the possibility to run the code without including closed fermion
loops. This can be achieved by setting the variable
\tmtexttt{GOSAMDIR} to \tmtexttt{GoSamlib\_nofboxes} in the Makefile,
and recompiling the code from scratch.

\begin{thebibliography}{1}

%\cite{Hamilton:2016bfu}
\bibitem{Hamilton:2016bfu}
  K.~Hamilton, T.~Melia, P.~F.~Monni, E.~Re and G.~Zanderighi,
  %``Merging WW and WW+jet with MINLO,''
  arXiv:1606.07062 [hep-ph].
  %%CITATION = ARXIV:1606.07062;%%

%\cite{Melia:2011tj}
\bibitem{Melia:2011tj}
  T.~Melia, P.~Nason, R.~Rontsch and G.~Zanderighi,
  %``W+W-, WZ and ZZ production in the POWHEG BOX,''
  JHEP {\bf 1111} (2011) 078
  doi:10.1007/JHEP11(2011)078
  [arXiv:1107.5051 [hep-ph]].
  %%CITATION = doi:10.1007/JHEP11(2011)078;%%
  %171 citations counted in INSPIRE as of 17 Jun 2016
  
\bibitem{POWHEG}
  POWHEG-BOX-V2/Docs/V2-paper.pdf

%\cite{Re:2018vac}
\bibitem{Re:2018vac}
  E.~Re, M.~Wiesemann and G.~Zanderighi,
  %``NNLOPS accurate predictions for $W^+W^-$ production,''
  arXiv:1805.09857 [hep-ph].
  %%CITATION = ARXIV:1805.09857;%%
  %1 citations counted in INSPIRE as of 29 Jun 2018

%\cite{Grazzini:2017mhc}
\bibitem{Grazzini:2017mhc}
  M.~Grazzini, S.~Kallweit and M.~Wiesemann,
  %``Fully differential NNLO computations with MATRIX,''
  arXiv:1711.06631 [hep-ph].
  %%CITATION = ARXIV:1711.06631;%%
  %14 citations counted in INSPIRE as of 29 Jun 2018

  
\end{thebibliography}

\end{document}
