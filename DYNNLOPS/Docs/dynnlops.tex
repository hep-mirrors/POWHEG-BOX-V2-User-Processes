\documentclass[11pt,a4paper]{article}
\pdfoutput=1 % if your are submitting a pdflatex (i.e. if you have
             % images in pdf, png or jpg format)

\usepackage{jheppub}

\usepackage[T1]{fontenc}

\usepackage[latin9]{inputenc}
\usepackage{esint}


\newcommand{\noun}[1]{{\tt #1}}
\newcommand{\POWHEG}{\noun{POWHEG}}
\newcommand{\POWHEGBOX}{\noun{POWHEG BOX}}
\newcommand{\MINLO}{\noun{MiNLO}}
\newcommand{\ZJMINLO}{\noun{ZJ-MiNLO}}
\newcommand{\WJMINLO}{\noun{WJ-MiNLO}}
\newcommand{\VJMINLO}{\noun{VJ-MiNLO}}
\newcommand{\DYNNLO}{\noun{DYNNLO}}
\newcommand{\NNLOPS}{\noun{NNLOPS}}
\newcommand{\NLOPS}{\noun{NLOPS}}
\newcommand{\PYTHIA}{\noun{Pythia}}
\newcommand{\HERWIG}{\noun{Herwig}}
\newcommand{\MCatNLO}{\noun{MC@NLO}}
\newcommand{\MEPS}{\noun{MEPS}}
\newcommand{\JETVHETO}{\noun{JetVHeto}}
\newcommand{\FASTJET}{\noun{FastJet}}
\newcommand{\as}{\alpha_{\scriptscriptstyle \mathrm{S}}}
\newcommand{\Kr}{K_{\scriptscriptstyle \mathrm{R}}}
\newcommand{\Kf}{K_{\scriptscriptstyle \mathrm{F}}}
\newcommand{\mur}{\mu_{\scriptscriptstyle \mathrm{R}}}
\newcommand{\muf}{\mu_{\scriptscriptstyle \mathrm{F}}}
\newcommand{\pt}{p_{\scriptscriptstyle \mathrm{T}}}
\newcommand{\pth}{p_{\scriptscriptstyle \mathrm{T}}^{\scriptscriptstyle \mathrm{H}}}
\newcommand{\ptjone}{p_{\scriptscriptstyle \mathrm{T}}^{\scriptscriptstyle \mathrm{j_{1}}}}
\newcommand{\kt}{k_{\scriptscriptstyle \mathrm{T}}}
\newcommand{\mh}{m_{\scriptscriptstyle \mathrm{H}}}
\newcommand{\mH}{mh}

\newcommand{\ptjo}{\pt^{{\scriptscriptstyle \mathrm{j}_{1}}}}
\newcommand{\hc}{\beta}
\newcommand{\hgam}{\gamma}
\newcommand{\smallk}{\kappa}
\newcommand{\comment}[1]{{\bf [#1] }}

\usepackage[mathscr]{euscript}

% continuation line arrows
\newcommand{\continueend}[0]{\raisebox{2.5mm}%
{\rotatebox{-90}{$\curvearrowright$}}}
\newcommand{\continuebeg}[0]{\raisebox{-0mm}%
{\rotatebox{90}{$\curvearrowleft$}}}



\title{{NNLOPS accurate Drell-Yan production}}

% more complex case: 4 authors, 3 institutions, 2 footnotes
\author[a]{Alexander Karlberg,}
\author[a]{Emanuele Re,}
\author[a,b]{Giulia Zanderighi}

% The "\note" macro will give a warning: "Ignoring empty anchor..."
% you can safely ignore it.

\affiliation[a]{Rudolf Peierls Centre for Theoretical Physics, University of Oxford\\1 Keble Road, UK}
\affiliation[b]{Theory
Division, CERN,\\CH--1211, Geneva 23, Switzerland}
% e-mail addresses: one for each author, in the same order as the authors
\emailAdd{a.karlber1g@physics.ox.ac.uk}
\emailAdd{e.re1@physics.ox.ac.uk}
\emailAdd{g.zanderighi1@physics.ox.ac.uk}

\abstract{This document contains instructions for the promotion of \VJMINLO{} Les Houches event
  files from NLO to NNLO accuracy in the description of inclusive Drell-Yan 
  production observables, by a reweighting procedure making use of the
  \DYNNLO{} program~\cite{Catani:2009sm}.}


%\makeatother

\begin{document}
\maketitle
\flushbottom

\section{Introduction}

This manual explains how to run the code (that makes use of the {\tt
  MiNLO} procedure) in the {\tt{Zj/DYNNLOPS}} and {\tt{Wj/DYNNLOPS}}
directories, in conjunction with the \DYNNLO{} program, to get
\NNLOPS{} accuracy for inclusive Drell-Yan production.

In order to obtain the needed source code, the user has to download the {\tt DYNNLOPS} directory from the {\tt Zj} or {\tt Wj} main directory, using the following command

{\tt svn co [---revision n] ---username anonymous ---password anonymous \textbackslash

\hspace{1cm} svn://powhegbox.mib.infn.it/trunk/User-Processes-V2/DYNNLOPS} . 


The {\tt DYNNLOPS} directory contains five folders
\begin{itemize}
\item {\tt aux}: a directory containing auxiliary files (e.g. to combine histograms, make plots, etc.); 
\item {\tt Docs}: a directory containing the manual; 
\item {\tt COMMON}: a directory containing all files common to W and Z production; 
\item {\tt WNNLOPS} and {\tt ZNNLOPS}: directories containing files specific of W or Z production. 
\end{itemize}

As there is no fundamental difference between obtaining \NNLOPS{}
accurate results for Z and W production using this code, in the
following we will give instructions for the case of Z production. For
the W case, the commands listed in the rest of this document need to be
changed straightforwardly.


\section{NNLO input\label{sec:NNLO-ingredients}}
A fundamental  ingredient required to obtain NNLO accurate
event samples is a triple differential distribution for the Z boson Born kinematics.\footnote{
In this case the Born kinematics is fully specified in terms of the
rapidity of the boson, an (arbitrary) angle describing the final state leptons and the invariant mass of the lepton pair.} We here give instructions on how to obtain
such distributions, in a format suitable for upgrading the \ZJMINLO{}
events via the {\tt{Zj/DYNNLOPS}} reweighting code.

\begin{enumerate}

\item Make sure the {\noun{LHAPDF}} package is installed:\\
\hspace*{5mm}
{\url{https://lhapdf.hepforge.org}}\\
In particular, for the installation of \DYNNLO{},
the command {\tt{lhapdf-config ---libdir}} should return the 
location of the installed {\noun{LHAPDF}} libraries.

\item Download \DYNNLO{} from the following URL:\\
\hspace*{5mm}
{\url{http://theory.fi.infn.it/grazzini/codes/dynnlo-v1.4.tgz}}

\item Unpack the tarball in a convenient location\\
\hspace*{5mm}
{\tt{\$ cp ~/Downloads/dynnlo-v1.4.tgz ./}}\\
\hspace*{5mm}
{\tt{\$ tar -xzvf dynnlo-v1.4.tgz}}\\
\hspace*{5mm}
{\tt{\$ ls dynnlo-v1.4}}\\
Under the parent directory {\tt{dynnlo-v1.4}} one should find subdirectories
{\tt{bin}}, {\tt{doc}}, {\tt{obj}}, {\tt{src}}, and a {\tt{makefile}}.

\item Enter the \DYNNLO{} parent directory\\
\hspace*{5mm}
{\tt{\$ cd dynnlo-v1.4}}

\item  Replace (or link) the \DYNNLO{} default {\tt{makefile}} with the one
from the \\ {\tt{Zj/DYNNLOPS/ZNNLOPS/dynnlo-patches}} directory\\
\hspace*{5mm}
{\tt{\$ cp /path/to/Zj/DYNNLOPS/ZNNLOPS/dynnlo-patches/dynnlo.makefile ./makefile}}

\item  Link (or copy using the {\tt -L} option) the \DYNNLO{} patches directory into the parent directory\\
\hspace*{5mm}
{\tt{\$ cp -r -L /path/to/Zj/DYNNLOPS/ZNNLOPS/dynnlo-patches ./}}

\item  Build the code\\
\hspace*{5mm}{\tt{\$ make}}\\
A message {\tt{----> DYNNLO compiled with LHAPDF routines <----}}
indicates success. You might need to set the enviromental variable {\ttfamily LD\_LIBRARY\_PATH} to the output of
{\ttfamily lhapdf-config ---libdir}, although the patched {\ttfamily dynnlo.makefile} contains commands that should take care
of this automatically.

\item  Copy over the template input file\\
\hspace*{5mm}{\tt{\$ cd bin}}\\
\hspace*{5mm}{\tt{\$ cp /path/to/Zj/DYNNLOPS/ZNNLOPS/dynnlo-patches/dynnlo.infile ./}}\\
%\hspace*{5mm}{\tt{work\$ cp /path/to/HJ/NNLOPS/HNNLO-mur-muf-scan.sh ./}}\\

\item  Edit the input file as desired.
The input file is nothing but a standard \DYNNLO{} input file and may be
run simply by typing\\
\hspace*{5mm}{\tt{\$ ./dynnlo < dynnlo.infile  >\,> my.log}}\\ 
yielding a {\tt{nnlo.top}}  and {\tt{nnlo3D.top}} histogram file.
\vspace{3mm}\\
The first file contains one-dimensional distributions which can be used for phenomenological studies. The second file is the one of interest here, as it contains a set of three-dimensional distributions which can be used for reweighting \ZJMINLO{} events. As each histogram consists of $25�=15625$ bins, very high statistics are required to properly populate the tails of the distributions. For the analysis presented in \cite{Karlberg:2014qua} we ran 100 instances of \DYNNLO{} with the following parameters:\\
\hspace*{5mm}{\tt 15 10000000         ! itmx1, ncall1}\\
\hspace*{5mm}{\tt 30 10000000         ! itmx2, ncall2}
\vspace{3mm}\\
%Alternatively the {\tt{HNNLO-mur-muf-scan.sh}} script loops over rescaling
%factors ${\frac{1}{4}}$, ${\frac{1}{2}}$, 1, 2, independently, for $\mur$ and
%$\muf$, and, moreover, a number of sequentially ordered random seeds,
%{\tt{njobs}}, for each $\mur$, $\muf$ value; the intention being that runs
%different only by the value of the random seed may be combined to increase
%the statistical precision. Statistically equivalent histogram files may be
%combined with the mergedata program:\\
Changing the random seed {\tt rseed} for each run. To combine the runs we provide a fortran program, {\tt merge3ddata.f}, in the {\tt aux} folder.  
As it can use big amounts of memory, we recommend compiling with the following flags

\hspace*{5mm}{\tt{\$ gfortran -mcmodel=medium -o merge3ddata merge3ddata.f }}

To then merge the {\tt nnlo3D.top} files into a file called {\tt dynnlo.top} do the following

\hspace*{5mm}{\tt{\$ ./merge3ddata 1 file\_1.top ... file\_N.top}}\\
\hspace*{5mm}{\tt{\$ mv fort.12 dynnlo.top}}\vspace{3mm}\\
%
We would like to note here that, due to the way \DYNNLO{} computes the real contribution of the cross-section, the errors of the {\tt nnlo3D.top} files (and {\tt nnlo.top}) are not reliable. In order to asses the precision on the three dimensional distributions, we provide a small program, {\tt 3dto1d.f}, which will integrate the three-dimensional distributions into three one-dimensional distributions. After compilation it is run by

\hspace*{5mm}{\tt{\$ ./3dto1d dynnlo.top}}

which will produce a file fort.12 with one-dimensional distributions. 

In the case where a user might not be interested in high rapidity events, it is presumably reasonable to run with fewer points  than suggested here. 

\item Finally, make sure that all input parameters (width, mass, couplings, pdfs) should be the same in \DYNNLO{}  and \ZJMINLO{}. (The patched \DYNNLO{} program outputs all relevant physical parameters). 

\end{enumerate}


\section{{\noun{Zj-MiNLO}} events\label{sub:HJ-MINLO-ingredients}}

The other fundamental ingredient needed for the NNLO reweighting procedure are  \ZJMINLO{} Les Houches events. A few modifications of the standard code are needed to obtain three-dimensional distributions and  to maintain
consistency of the physical parameters in \DYNNLO{} and \ZJMINLO{}.  In order to include them, it is sufficient to replace the default {\ttfamily POWHEG} makefile with the patched one:\vspace{3mm}\\
\hspace*{5mm} {\tt{\$ cp DYNNLOPS/ZNNLOPS/powheg-patches/powheg.makefile ./makefile}}\vspace{3mm}\\
In {\tt{DYNNLOPS/ZNNLOPS/powheg-patches/powheg.input}} a template input file can also be found. If the code is run with the patches out of the box and the above input file, consistency between \DYNNLO{} and \ZJMINLO{} should be achieved. 
However, we stress again that it is up to the user to make sure this is the case.


Few comments are helpful:
\begin{enumerate}

\item  A large number of physical parameters used by the \ZJMINLO{} code,
  such as the Fermi constant, are assigned by the subroutine
  {\tt{init\_couplings}}
  defined in the file\linebreak {\tt{Zj/init\_couplings.f}}. As in the case of
  the \DYNNLO{}\ \  {\tt{mdata.f}} file, some of the parameters in this file are
  irrelevant. Nevertheless, in {\tt{Zj/DYNNLOPS/ZNNLOPS/powheg.input}} all these parameters can be changed. The version we provide is consistent with our patched 
  version of {\tt mdata.f}

 \item For detailed instructions
  on setting up the \ZJMINLO{} program to perform numerous runs in parallel
  see section 4.1 in
  {\tt{/path/to/W2jet/Docs/manual-BOX-WZ2jet.pdf}}
  
  \item We add that although
  it is not strictly necessary to generate the \ZJMINLO{} events using a
  NNLO PDF, in the limited studies that we have carried out to date we used the same (NNLO) PDF set in \ZJMINLO{} and \DYNNLO{}.
  
\item As was the case when obtaining NNLO distributions, a large number of events is needed to populate the tails of the three-dimensional distributions. For the study carried out in \cite{Karlberg:2014qua} we used 20 million events to compute denominators.  


\end{enumerate}


\section{Reweighting\label{sub:Reweighting}}

The NLO-to-NNLO weight factor assigned to the \VJMINLO{} events is
differential in the vector boson phase space  ($\Phi_B$) and the transverse
momentum of the leading jet ($\pt$), at the NNLO level in the \DYNNLO{}
case, and at the level of the hardest emission cross section (Les Houches
events) in \VJMINLO{}. It is given by 
\begin{eqnarray} \mathcal{W}\left(\Phi_B,\, p_{{\scriptscriptstyle
\mathrm{T}}}\right)&=&h\left(\pt\right)\,\frac{\smallint
d\sigma^{{\scriptscriptstyle
\mathrm{NNLO\phantom{i}}}}\,\delta\left(\Phi_B-\Phi_B\left(\Phi\right)\right)-\smallint
d\sigma_{B}^{{\scriptscriptstyle
\mathrm{MINLO}}}\,\delta\left(\Phi_B-\Phi_B\left(\Phi\right)\right)}{\smallint
d\sigma_{A}^{{\scriptscriptstyle
\mathrm{MINLO}}}\,\delta\left(\Phi_B-\Phi_B\left(\Phi\right)\right)}\nonumber\\
&+&\left(1-h\left(\pt\right)\right)\,,\label{eq:NNLOPS-overall-rwgt-factor-1}
\label{eq:Wdef}
\end{eqnarray} 
where
\begin{eqnarray}
d\sigma_{A} & = & d\sigma\, h\left(\pt\right)\,,\label{eq:NNLOPS-dsig0}\\
d\sigma_{B} & = & d\sigma\,\left(1-h\left(\pt\right)\right)\,,\label{eq:NNLOPS-dsig1}
\end{eqnarray}
with $h$ a monotonic profile function 
\begin{equation}
h(\pt)=\frac{(\hc\, m_{{\scriptscriptstyle \mathrm{V}}})^{2}}{(\hc\, m_{{\scriptscriptstyle \mathrm{V}}})^{2}+\pt^{2}},\label{eq:NNLOPS-hpT-fn-defn}
\end{equation}
and $\hc$ a constant parameter. 
On convoluting $\mathcal{W}\left(\Phi_B,\, \pt\right)$ with the \VJMINLO{} differential cross section
and integrating over $\pt$ one finds, exactly,
\begin{eqnarray}
\left(\frac{d\sigma}{d\Phi_B}\right)^{{\scriptscriptstyle \mathrm{NNLOPS}}} & = & \left(\frac{d\sigma}{d\Phi_B}\right)^{{\scriptscriptstyle \mathrm{NNLO}}}\,.\label{eq:NNLOPS-NNLOPS-eq-NNLO_0+MINLO_1-1}
\end{eqnarray}
For a proof of why such a reweighting procedure leads to NNLO accuracy in
general, we refer the
reader to \cite{Hamilton:2013fea,Karlberg:2014qua}. 

The role of the profile function $h$ and, in particular, the $\hc$
parameter within it, is, roughly speaking, to determine how to spread out
the NLO-to-NNLO corrections along the $\pt$ axis. For
$\hc=\infty$ the corrections are spread uniformly in $\pt$ (see e.g.
fig.~3 of ref.~\cite{Hamilton:2013fea}), while for $\hc=1$
they are concentrated in the region $\pt<\hc\,m_V$. In the latter
respect the $\hc$ parameter plays a similar role to the resummation
scale in dedicated resummation calculations, and as such we favour 
that $\hc$ be set consistently with the preferred values in those.
Thus we recommend $\hc=1$ in carrying out the reweighting.
Indeed for $\hc=1$ we find good agreement with dedicated
NNLL+NNLO calculations of the Z boson transverse momentum and the
0-jet veto efficiency where the resummation scale was set to
$m_Z$.

We should emphasise that, while our NNLOPS simulation is formally NNLO
accurate for inclusive quantities, the accuracy of its resummation of
all-orders large logarithms is, formally, categorically not at the
NNLL+NNLO level. We recommend that the predictions of such calculations
be used to `tune' the \NNLOPS{} simulation (in particular the $\hc$
parameter) to approximate the yet higher order, large logarithmic, terms
which it does not take into account. Setting $\hc=1$ appears
to do a very good job in this respect, thus it is the default, recommended,
value in the following. If the reweighting is performed using eq.~\eqref{eq:Wdef}, the value of $\beta$ can be changed by just 
modifying in a straightforward way the file {\tt get\_zdamp.f} ({\tt get\_wdamp.f} for W production). 


\subsection{A simple run\label{sec:simple-reweighting}} 

The code to be used in order to produce an NNLO-reweighted event file
can be found in the {\tt{DYNNLOPS/ZNNLOPS/Reweighter}} directory. After
compiling it (\FASTJET{} needs to be linked), the program has to
be run with the following command line:
\vspace{3mm}\\
\hspace*{5mm}{\tt \$ ./minnlo <ZjMiNLO-eventfile> <nr-DYNNLOfiles> <DYNNLO-file1>} \\ 
\hspace*{10mm}{\tt [ <DYNNLO-file2>...]  [<ZjMiNLO-file1>  <ZjMiNLO-file2>  .... ] }
\vspace{3mm}\\
%

%\vspace{3mm}\\
%\hspace*{5mm}{\tt{\$ ./minnlo <ZjMiNLO-eventfile> <DYNNLO-identifier> <DYNNLO-outputfile> \{<ZjMiNLO-denominators>\}}}\vspace{3mm}\\
%


where {\tt ZjMiNLO-eventfile} is a LH file (suffix {\tt .lhe})
containing events produced with \ZJMINLO{}, obtained as described in
sec.~\ref{sub:HJ-MINLO-ingredients}, 
{\tt nr-DYNNLOfiles} denotes the number of {\tt DYNNLO} files containing the 3D distributions (more then one is needed for instance when computing uncertainty bands using scale variation), {\tt DYNNLO-file*} are output files of {\tt DYNNLO},  computed as described in sec~\ref{sec:NNLO-ingredients}. The number of files given in the input line \emph{must} correspond to {\tt nr-DYNNLOfiles}. Finally, {\tt ZjMiNLO-file*} contain three-dimensional histograms  computed using {\tt ZJ-MiNLO} events. These last arguments are optional. If these files are not present in the command line, the reweighing program computes them and stops. We call this stage one. When the {\tt ZjMiNLO-file*} are computed by the reweighing program using the {\tt ZjMiNLO-eventfile}, the first weight present in the LHE file will produce a file called {\tt MINLO-W1-denom.top}, the second weight a file called {\tt MINLO-W2-denom.top} and so on. 

Rather than producing just one huge event file, it is also possible to compute the {\tt MINLO-W*-denom.top} files on each event sample and then combine the resulting distributions using the program {\tt merge3ddata.f} (in {\tt aux}).\footnote{In this case, care has to be take to produce files in different directories.} 
With the generated {\tt MINLO-W*-denom.top} files one can then go on and perform the actual NNLO reweighing including in the command line these files. We call this stage two. The files \emph{have} to be given in the same order as the corresponding weights in the LH file. Otherwise the reweighting will be incorrect. 

Here follows an example on how to go through both stages, assuming that the user has a \ZJMINLO{} event file called {\tt pwgevents.lhe} containing 7 weights and 3 \DYNNLO{} files called {\tt dynnlo-mur0.5-muf0.5.top}, {\tt dynnlo-mur1.0-muf1.0.top} and {\tt dynnlo-mur2.0-muf2.0.top}. The program should be called as\vspace{3mm}\\
\hspace*{5mm}{\tt \$./minnlo pwgevents.lhe 3 dynnlo-mur0.5-muf0.5.top  \textbackslash }

\hspace*{5mm} {\tt  dynnlo-mur1.0-muf1.0.top dynnlo-mur2.0-muf2.0.top}\vspace{3mm}\\
This will result in an output of {\tt MINLO-W1-denom.top}, {\tt MINLO-W2-denom.top}, {\tt MINLO-W3-denom.top}, {\tt MINLO-W4-denom.top}, {\tt MINLO-W5-denom.top}, {\tt MINLO-W6-denom.top}, {\tt MINLO-W7-denom.top}.\\

To perform stage two the user would run \vspace{3mm}\\
\hspace*{5mm}{\tt \$./minnlo pwgevents.lhe 3  dynnlo-mur0.5-muf0.5.top  \textbackslash }

\hspace*{5mm} {\tt dynnlo-mur1.0-muf1.0.top dynnlo-mur2.0-muf2.0.top MINLO-W*-denom.top}\vspace{3mm}\\
which will produce a file called {\tt pwgevents.lhe-nnlo}.\footnote{As usual, the shell will match and expand {\tt MINLO-W*-denom.top} to the list of files {\tt MINLO-W1-denom.top}, {\tt MINLO-W2-denom.top},..., {\tt MINLO-W7-denom.top}, which is the actual input read by the {\tt minnlo} program.} This LH file is the final output of the {\tt minnlo} NLO-to-NNLO reweighter program, and can now be read and
showered by \PYTHIA{} or \HERWIG{}, as it is usually done with LH
event files generated by the \POWHEGBOX{}.  {It is important to notice that
the new NNLO weight attached to each event is written \emph{after} the
event record, in the line comprised between the partonic momenta and
the end-of-event tag ({\tt </event>})}. Similarly to the format of a
\POWHEGBOX{} output when the reweighting machinery is used, this line
starts with the text
\vspace{3mm}\\
\hspace*{5mm}{\tt \#new
  weight,renfact,facfact,pdf1,pdf2}\vspace{3mm}\\
%
followed by a weight and other information. In a LH file obtained
with the {\tt minnlo} reweighter, the names of the {\tt DYNNLO-output} files are
appended at the end of this line, and the NNLO weight to be used is
the first number appearing, which will obviously be different from the
original weight contained in {\tt ZJMiNLO-eventfile}. This different
weight, together with the appended identifier at the end of each line
starting with {\tt \#new weight}, should be the only differences
between the input and output LH files. Notice that the NNLO-reweighted
output file will always contain the {\tt \#new weight} line(s), even
if the original LH file {\tt ZJMiNLO-eventfile} didn't.

The program takes few hours to reweight a LH file produced by
\ZJMINLO{} containing 20 million events. The output printed on the
terminal at run time is self explanatory. We have also included a
template script ({\tt runminnlo\_template.sh}) to help the user.

\subsection{Estimating uncertainties\label{sec:Estimating-uncertainties}}

The conservative ansatz we recommend in estimating errors (the one
employed in ref.~\cite{Hamilton:2013fea,Karlberg:2014qua}) is that the  $\mur$ and
$\muf$ scales in the NNLO and NLO inputs should be varied in a fully
independent way. In doing so we regard the uncertainties in the
normalisations of distributions as being independent of the respective
uncertainties in the shapes, at least in the region covered by the
profile function $h(\pt)$, i.e. in the low $\pt$
domain. Normalisation uncertainties are determined
by the \DYNNLO{} program, while shape uncertainties are due to
\ZJMINLO{}. Outside of the low $\pt$ region, the uncertainty is given by the
standard \ZJMINLO{} computation (which uses in that region $\mur=\muf=\pt$ as central scale choice).

In order to compute an uncertainty band, one first needs to have
obtained multiple outputs from \DYNNLO{} and \ZJMINLO{} with different $\mur$
and $\muf$. For the sake of simplicity, we will now describe a case
where $\mur$ and $\muf$ are kept equal both when running \DYNNLO{} and
\ZJMINLO{}. We call this situation a ``3x3 pts'' scale variation
study: for each event we will obtain 9 NNLOPS weights, associated to
each of the 9 possible combinations among 3 results from \DYNNLO{} and
3 from \ZJMINLO{}.  This procedure is general enough to be
straightforwardly adapted to the more general case of a ``7x7 pts''
scale variation, or variation thereof.

In the ``3x3 pts'' case, the needed inputs are 3 histogram files from
\DYNNLO{}, which we will call {\tt DYNNLO-outputfile-HH.top}, {\tt
  DYNNLO-outputfile-11.top} and {\tt DYNNLO-outputfile-22.top}, for
values of $\mur=\muf=\{m_Z/2,m_Z,2m_Z\}$ respectively.  Similarly,
the user needs to have obtained a LH file from \ZJMINLO{} where 3
weights are associated to each event. This file has to be obtained
with the \POWHEGBOX{} reweighting machinery: the 3 lines in between
the last line of each event record and the {\tt </event>} tag should
have the format
\vspace{3mm}\\
\hspace*{5mm}{\tt \#new weight,renfact,facfact,pdf1,pdf2 <weight>
  <ren.scale factor> \continueend}\\
\hspace*{5mm} {\tt \continuebeg \ <fact.scale fact> 
  <pdf number for hadron 1> \continueend}\\
\hspace*{5mm} {\tt \continuebeg \ <pdf number for hadron 2> <'mlm' or 'lha'>}\vspace{3mm}\\
%
The NLO-to-NNLO reweighter program should now be invoked in a first steep as:
\vspace{3mm}\\
\hspace*{5mm}{\tt{\$ ./minnlo ZJMiNLO-events.lhe 3
    DYNNLO-outputfile-HH.top \textbackslash }}\\
%
 \hspace*{5mm}{\tt {   DYNNLO-outputfile-11.top \
%
    DYNNLO-outputfile-22.top}}\vspace{3mm}\\
%
This will produce the three {\tt MiNLO} files with three-dimensional distributions. 

In a second step, one calls 
\vspace{3mm}\\
\hspace*{5mm}{\tt{\$ ./minnlo ZJMiNLO-events.lhe 3
    DYNNLO-outputfile-HH.top \textbackslash }}\\
%
 \hspace*{5mm}{\tt {   DYNNLO-outputfile-11.top \
%
    DYNNLO-outputfile-22.top \textbackslash}}\\
 \hspace*{5mm}{\tt { MINLO-W1-denom.top MINLO-W2-denom.top  MINLO-W3-denom.top }}\vspace{3mm}\\
At the end
of the run, 9 lines will be present after each event record, each one
containing the NNLO weight associated to the \ZJMINLO{} result
labelled by the values of the pair ({\tt <ren.scale factor>}, {\tt
  <fact.scale fact>}) and the \DYNNLO{} result identified by the name of the file appended to each line.

\section{Example analysis\label{sec:Analysis}}

The analysis used for the phenomenological study in
ref.~\cite{Karlberg:2014qua} can be found in
{\tt{pwhg\_analysis-minlo.f}}. The analysis is compiled in the {\tt Zj} directory in the usual way by \\
\hspace*{5mm}{\tt make lhef\_analysis}\\

It can also be used to analyse events after the
showering stage. We provide drivers for both Pythia6 (version 6.4.28) and Pythia8 (version 8.185) which can be compiled by\\
\hspace*{5mm}{\tt make main-PYTHIA-lhef}\\
\hspace*{5mm}{\tt make main-PYTHIA8-lhef}\\

Note that \POWHEGBOX{} ships with Pythia6.4.25 as standard. It is straightforward to use any other version, by downloading it, and putting it in the main \POWHEGBOX{} directory. 
%
For Pythia8 the user has to compile it himself and then set the {\tt PYTHIA8LOCATION} appropriately. 

\section{Suggested citations}
If you use this code please cite the inclusive Drell-Yan production
\NNLOPS{} paper \cite{Karlberg:2014qua} and the \VJMINLO{} paper which
precedes and lays much of the foundations for it \cite{Hamilton:2012rf}.
The \NNLOPS{} simulation fundamentally relies on the NNLO Drell-Yan boson
production calculation of refs.~\cite{Catani:2009sm},
and so these works should also be cited.


\appendix


\section{Code description}
In this section we briefly record the various additions and modifications
to existing \DYNNLO{} and \POWHEGBOX{} code used to produce \NNLOPS{} events.

\subsection{DYNNLO\label{sec:DYNNLOcode}}
\begin{itemize}

\item {\tt{makefile}} ({\tt{DYNNLO-makefile}})
  \begin{itemize}
  \item The \DYNNLO{} {\tt{makefile}} is modified by
    prepending {\tt{\$(DYNNLOHOME)/dynnlo-patches}} to
    the {\tt{\$DIRS}} variable, introducing a variable {\tt{\$PATCHES}}
    equal to the concatenation of the following object files in this list,
    plus the removal of those elements from the other {\tt{Makefile}}
    variables (avoiding duplication). In this way the modified \DYNNLO{}
    files below are compiled and linked from {\tt{dynnlo-patches}} instead
    of the default versions in the default locations.
  \end{itemize}

\item {\tt{mbook.f}}
  \begin{itemize}
  \item The {\tt{mtop}} subroutine, which outputs the \DYNNLO{} histograms
    to a text file, has undergone a minor modification so as to have the
    same format as the \VJMINLO{} histogram output, to ease comparisons
    and for use in the reweighting program.
  \end{itemize}

\item {\tt{mdata.f}}
  \begin{itemize}
  \item This file contains the values of numerous physical constants in
    \DYNNLO{} e.g. Fermi's constant $G_{F}$, etc.  We have edited all
    parameters in this file to have agreement with the corresponding
    \VJMINLO{} default settings.
  \end{itemize}

\item {\tt{plotter.f}}
  \begin{itemize}
  \item This file contains an example analysis for \DYNNLO{} by
    default.  We have modified this analysis  to produce
    three-dimensional histograms. Also a number of distributions relevant for phenomenology is being filled.
    The three-dimensional histogram range has been set to
    $-5<y<5$, $-\pi/2<a_{mll}<\pi/2$ and $0<\theta_l<\pi$. The distributions have 25 bins in each direction. These values may be altered by
    the user as desired, by editing the relevant booking
    subroutine call. However, in this case one must take care to
    modify the relevant \POWHEGBOX{} analysis file ({\tt{pwhg\_analysis-release.f}}) used by the
    reweighting code, under the {\tt{Vj/DYNNLOPS/VNNLOPS/Reweighter}} directory.
  \end{itemize}

\item {\tt{setup.f}}
  \begin{itemize}
  \item This file was modified to use the $k_T$-algorithm instead of the anti-$k_T$.
  \end{itemize}

\item {\tt{writeinfo.f}}
  \begin{itemize}
  \item Originally the {\tt{writeinfo}} subroutine in this file copied
    the contents of the input file used in running the program, plus the
    cross section, as a series of comments to the top of the histogram
    output file. In order to have a simple uniform format for the \VJMINLO{}
    and \DYNNLO{} we removed these comments (the bulk of which was simply
    a copy of the input file used to run the program).
  \end{itemize}

\item {\tt{histofinLH.f}}
  \begin{itemize}
  \item This file takes care of finalising the histograms. As we are using the \POWHEGBOX{} routines to fill our histograms (and a modified version to fill three-dimensional histograms) this file had to be modified to fill the correct histograms and finalise them.
  \end{itemize}

\item {\tt{coupling.f}}
  \begin{itemize}
  \item We added a printout to the screen of various parameters, to make it easier to compare with the output of \VJMINLO{}.  
  \end{itemize}

\item {\tt{boost.f}}
  \begin{itemize}
  \item Added a routine to go from the lab frame to the CM frame.
  \end{itemize}

\item {\tt{auxiliary.f,pwhg\_bookhist-multi.f,pwhg\_bookhist-new.f,pwhg\_book-multi.h}, {\tt pwhg\_bookhist-new.h},pwhg\_math.h}
  \begin{itemize}
  \item These files contains the routines to fill histograms, to have a consistent output between \DYNNLO{} and \VJMINLO{}. 
  \end{itemize}


\end{itemize}

\subsection{\ZJMINLO{}\label{sec:ZJMINLOcode}}

\begin{itemize}

\item {\tt{lhapdfif.f}}
  \begin{itemize}
  \item When using NNLO PDFs $\Lambda_5$ is now computed at NLL and not NNLL. 
  \end{itemize}

\item {\tt{setlocalscales2.f}}
  \begin{itemize}
  \item This file is used by the Makefile instead of the default {\tt setlocalscales.f}. We have changed the routine which computes $\alpha_S$ to be the default \POWHEGBOX{} one, instead of a locally defined prescription. The two prescriptions agree far away from $\Lambda_5$ but significant differences show up close to the Landau pole. 
  \end{itemize}

\item {\tt{powheg.input}} / {\tt{powheg.input-save}}
  \begin{itemize}
  \item Here all values should be set consistently with the values used by \DYNNLO{}.
  \end{itemize}

\end{itemize}

The rest of the files found in this folder are either a driver for Pythia (as used in \cite{Karlberg:2014qua} or files already described in the section above)
\subsection{NNLO reweighting\label{sec:HNNLO code}}

\begin{itemize}
  
\item {\tt{Makefile}}
  \begin{itemize}
  \item The default value for the {\tt ANALYSIS} variable should be
    {\tt release}. \FASTJET{}
    must be linked properly too, and as usual it is recommended to let
    the {\tt Makefile} call the {\tt fastjet-config} command.
  \end{itemize}
  
\item {\tt{minnlo.f}}
  \begin{itemize}
  \item This file contains the main program to perform the NLO-to-NNLO
    reweighting. Some parameters useful for debugging purposes can be
    found here, as described in the commented section at the beginning
    of the file. However, the user is recommended not to change them.
  \end{itemize}
  
\item {{\tt opencount.f}, {\tt auxilliary.f}, {\tt lhef\_readwrite.f},
    {\tt get\_zdamp.f}, {\tt genclust\_kt.f}, \\ {\tt swapjet.f},
    {\tt miscclust.f}, {\tt ptyrap.f}, {\tt r.f}}
  \begin{itemize}
  \item These files contain several functions and routines needed by
    {\tt minnlo.f} and/or by the analysis subroutine used to process
    the \ZJMINLO{} LH file and compute
    $d\sigma_{A/B}^{{\scriptscriptstyle
        \mathrm{MINLO}}}\,\delta\left(\Phi_B-\Phi_B\left(\Phi\right)\right)$.
  
  [{\tt jetlabel.f}, {\tt jetcuts.f}, {\tt mxpart.f} and {\tt npart.f}
  contain common blocks used in the source files.]
\end{itemize}

\item {{\tt pwhg\_analysis-release.f}, {\tt jet\_finder-release.f}}
  \begin{itemize}
  \item These files contain the minimal analysis needed to extract
    $d\sigma_{A/B}^{{\scriptscriptstyle
        \mathrm{MINLO}}}\,\delta\left(\Phi_B-\Phi_B\left(\Phi\right)\right)$
  from the LH input file. They are compiled if {\tt ANALYSIS=release}
  is set, which is the default option.
\end{itemize}
  
\end{itemize}

\bibliographystyle{JHEP}
\bibliography{dynnlops}


\end{document}
