\documentclass[11pt,a4paper]{article}\pdfoutput=1

%\usepackage{jheppub}
\usepackage{hyperref}
\usepackage{booktabs}
\usepackage{xspace}
\usepackage[T1]{fontenc}
\usepackage[utf8]{inputenc}
\usepackage{xcolor}
\usepackage[toc,page]{appendix}
\usepackage{mathtools}
\usepackage{slashed}
\usepackage{listings}

%---------------------------------------------------------------------------------------------------
\hypersetup{
    colorlinks=true,
    linkcolor=blue,
    citecolor=blue,
    urlcolor=magenta,
}

%---------------------------------------------------------------------------------------------------
\newcommand{\POWHEG}{{\tt{POWHEG}}}
\newcommand{\POWHEGBOX}{{\tt{POWHEG-BOX}}}
\newcommand{\POWHEGBOXVTWO}{{\tt{POWHEG-BOX-V2}}}
\newcommand{\MINLO}{{\tt{MiNLO}}}
\newcommand{\MINNLO}{{\tt{MiNNLO}}}
\newcommand{\hbbg}{{\tt{h\_bbg}}}
\newcommand{\pythia}{{\texttt{Pythia8}}}
%
\newcommand{\Hbb}[1]{\ensuremath{H\to b\bar{b}{#1}}}
\newcommand{\gev}{\textrm{GeV}}

%---------------------------------------------------------------------------------------------------
\definecolor{xcode}{rgb}{0.25,0.25,0.75}

\textwidth 15.5cm
\hoffset -1.5cm 
%---------------------------------------------------------------------------------------------------
\title{The \POWHEGBOXVTWO{} user manual:\\ Higgs boson decay into pair of $b$-quarks at NNLOPS}
\author{Wojciech Bizo\'n, Emanuele Re, Giulia Zanderighi}

%%>>>
%\author[a,b]{Wojciech Bizo\'n,}
%\author[c,d]{
%\author[e]{
%
%%>>>
%\affiliation[a]{Institute for Theoretical Particle Physics, KIT, Karlsruhe, Germany}
%\affiliation[b]{Institut f\"ur Kernphysik, KIT, 76344 Eggenstein-Leopoldshafen, Germany}
%\affiliation[c]{Theoretical Physics Department, CERN, 1211 Geneva 23, Switzerland}
%\affiliation[d]{LAPTh, CNRS, Universit\'e Savoie Mont Blanc, 74940 Annecy, France}
%\affiliation[e]{Max-Planck-Institut fu\"ur Physik, D-80805 Munich, Germany}
%
%%>>>
%\emailAdd{wojciech.bizon@kit.edu}
%\emailAdd{emanuele.re@lapth.cnrs.fr}
%\emailAdd{zanderi@mpp.mpg.de}
%
%\abstract{FIXME: xxxxxxxxxxxxxx
%}
%
%
%\keywords{FIXME: xxxxxxxxxxxxx}

\begin{document}
\maketitle \flushbottom
\lstset{
  language=bash,
  basicstyle=\small\ttfamily\color{xcode},
  basewidth=0.5em,
  backgroundcolor=\color{blue!3}
}

%###################################################################################################
\section{Introduction}
\label{sec:intro}

%%
In this manual, we describe the \POWHEGBOXVTWO{} implementation of the
Higgs boson decay into a pair of $b$-quarks accompanied by a gluon,
\begin{equation}
  H \longrightarrow b + \bar{b}~ + g \,,
\end{equation}
at NLO QCD accuracy.
%
The program is equipped with the \MINLO{} procedure, introduced in
Refs.~\cite{Hamilton:2012np,Hamilton:2012rf}, that ensures that one
retains NLO QCD accuracy for \Hbb{} observables upon integrating out
the additional real radiation.
%
Furthermore, the accuracy for the \Hbb{} observables can by upgraded
to NNLO QCD by a suitable reweighting of the Les Houches events.
%
The details of the \MINLO{} procedure as well as NNLO reweighting are
investigated in the Ref.~\cite{hbbg-paper}.

%%
This document is organised as follows: in Sec.~\ref{sec:generation} we
briefly describe how to generate and reweight the \hbbg{} event
samples; in Sec.~\ref{sec:interface} we explain how to interface the
\hbbg{} event sample with a specific Higgs production event
sample, where the Higgs is undecayed. Finally, in Sec.~\ref{sec:parton-shower} we describe how to
match the produced event sample with parton shower. We list all
physical and technical input parameters used in the \hbbg{} code in
App.~\ref{app:input} and App.~\ref{app:tech} respectively.



%###################################################################################################
\section{Generation of \hbbg{} events}
\label{sec:generation}

%---------------------------------------------------------------------------------------------------
\subsection{Generation of event sample with \POWHEG{}}
\label{sec:generation_1}
\begin{itemize}
  \renewcommand{\labelitemi}{$\rightarrow$}
  %%
\item Download the \hbbg{} user process into the \POWHEGBOXVTWO{} directory.
  %%
\item Navigate to the user process directory and build the executable
  \begin{lstlisting}
    $> cd POWHEG-BOX-V2/h_bbg
    $> make pwhg_main
  \end{lstlisting}
  %%
\item Run the code (for example)
  \begin{lstlisting}
    $> cd testrun
    $> ../pwhg_main
  \end{lstlisting}
\end{itemize}
Successful run should produce the file {\tt{pwgevents.lhe}} containing
1000 \hbbg{} events in the Les Houches format and should terminate in
only a few seconds. The physical and technical input parameters can
be changed by editing the {\tt{powheg.input}} card, which are listed
in App.~\ref{app:input} and App.~\ref{app:tech}. It is also possible
to run the code in parallel mode - for this a {\tt{powheg.input-save}}
card and relevant scripts from {\tt{testrun-parallel}} can be used as
an example.

Such event sample(s) can be already interfaced with a user's Higgs
production process. However, in order to achieve NNLOPS QCD accuracy,
an NNLO reweighting is required, see Sec.~\ref{sec:nnlo-rwgt}.


%---------------------------------------------------------------------------------------------------
\subsection{NNLO reweighting of the event sample}
\label{sec:nnlo-rwgt}
The NNLO reweighting is performed along the lines of the standard
\MINNLO{} technique, outlined in Sec.~5 of
Ref.~\cite{Hamilton:2012rf}.
%
In this particular case, due to simplicity of the \Hbb{} process, the
\MINNLO{} procedure boils down to a 0-dimensional NNLO reweighting that
is driven only by the total decay width ($\Gamma_{\rm{NNLO}}$) and its
\MINLO{} counter-part ($\Gamma_{\MINLO{}}$). Weights of all events
should be adjusted using a reweighting factor
\begin{align}
  \mathcal{W} = \frac{\Gamma_{\rm{NNLO}}}{\Gamma_{\MINLO{}}}\,.
\end{align}
%
In a more refined scheme, described in Sec.~2.2 of
Ref.~\cite{hbbg-paper}, which we adopt as our default,
the reweighting also depends on the three-jet
resolution parameter $y_3$ through a function
\begin{align}
  \mathcal{W}(y_3) =
  h(y_3) \cdot
  \frac{ \Gamma_{\rm{NNLO}} - \Gamma_{\MINLO{},\rm{B}} }{ \Gamma_{\MINLO{},\rm{A}} }
  +
  \left(  1 - h(y_3) \right),
\end{align}
where we have introduced a function $h(y_3)$,
\begin{align}
  \label{eq:hval}
  h(y_3) = \frac{1}{1 + \left(y_3 / y_{3,\rm{ref}}\right)^{\alpha} }
  \qquad\rm{with}\qquad
  \begin{cases}
    &\alpha > 0 \\
    &y_{3,\rm{ref}} > 0
    \end{cases},
\end{align}
that is used to classify the hardness of an event, i.e. $h(y_3)$
approaches one for $y_3\to 0$ and $h(y_3)\ll 1$ for hard events, so that for hard events
$\mathcal{W} = 1$. As a
default values we choose $y_{3,\rm{ref}} = e^{-4}$ and $\alpha = 2$.

~\\~
\noindent
In order to perform the NNLO reweighting of an event sample {\tt{pwgevents.lhe}}:
\begin{itemize}
  \renewcommand{\labelitemi}{$\rightarrow$}
  %%
\item Navigate to the user process directory \POWHEGBOXVTWO{}{\tt{/}}\hbbg{}
  %%
\item Compile the program {\tt{hdec}} which will drive the reweighting process
  \begin{lstlisting}
    $> make hdec -f makefile.hdec
  \end{lstlisting}
  After program is compiled, syntax can be checked by executing
  \begin{lstlisting}
    $> ./hdec
  \end{lstlisting}
  %%
\item Analyse existing event samples in order to calculate values
  $\Gamma_{\MINLO{},\rm{A/B}}$
  \begin{lstlisting}
    $> cd testrun
    $> ../hdec -histprep -events pwgevents.lhe -out hist.dat
  \end{lstlisting}
  This will produce file {\tt{hist.dat}} which contains necessary
  information used later for the NNLO reweighting. Note that the
  values of parameters $y_{3,\rm{ref}}$ and $\alpha$,
  cf. Eq.~\eqref{eq:hval}, can be altered using optional arguments
  {\tt{-y3ref}} and {\tt{y3pow}}.

  The output file {\tt{hist.dat}} contains results for a different
  renormalization scale variation, as specified in the {\tt
    powheg.input file}.  In the following, the number of weights at
  the \MINLO{} level will be denoted as {\tt num-lhef-wgt}. For
  example, if the original \MINLO{} events were obtained with a
  reweighting that performs a standard 3-point variation ($\mu_R =
  K_R^{\rm{MiNLO}} \mu_0$, with $K_R^{\rm{MiNLO}}=\{1.0,\, 0.5,\,
  2.0\}$), then {\tt num-lhef-wgt}=3.
  
  In case of parallel run ({\tt{testrun-parallel}}), remember to
  analyse all produced event files and merge the obtained histograms.
  The analysis of multiple event files can be performed either by
  specifying a number of event files following keyword {\tt{-events}}
  in the above example. Alternatively, analyse files separately and
  direct the output to various output files which can be specified
  using {\tt{-out}} keyword, for example
  \begin{lstlisting}
    $> cd testrun
    $> ../hdec -histprep -events pwgevents-0001.lhe -out hist-0001.dat
    $> ../hdec -histprep -events pwgevents-0002.lhe -out hist-0002.dat
    $> ...
    $> ...
  \end{lstlisting}
  Various output files {\tt{hist-????.dat}} should be merged by running
  \begin{lstlisting}
    $> ../hdec -histmerge -files hist-????.lhe -out hist.dat
  \end{lstlisting}
  which will combine results saved in all histograms into a single
  output file {\tt{hist.dat}} which will be used later for the NNLO
  reweighting.

\item Having obtained the histogram file {\tt{hist.dat}}, the NNLO
  reweighting can be performed by running
  \begin{lstlisting}
    $> ../hdec -minnlo -hist hist.dat -events pwgevents.lhe
  \end{lstlisting}
  which will produce a reweighted event file
  {\tt{pwgevents.lhe-minnlo}}, where the NNLO weights were appended
  after the \MINLO{} ones. Multiple files can be reweighted at once
  simply by specifying a number of arguments after the keyword
  {\tt{-events}}; each file name will be supplied with the
  ``{\tt{-minnlo}}'' suffix after reweighting.

  Note that, by default, the NNLO reweighting will prepare all
  possible combinations of weights, i.e. for each of \MINLO{} weights,
  the reweighted file will contain three weights related to scale
  variation in the computation of $\Gamma_{\rm{NNLO}}$ with
  $K_R^{\rm{NNLO}}=\{1.0,\, 0.5,\, 2.0\}$.

  The user can constrain the possible scale combinations by preparing
  a file {\tt{wgt.dat}} which contains a list of desired scale
  combinations. Each line of this file must contain the identifier of
  a given \MINLO{} weight ({\tt minlo\_wgt=1,2,...,num-lhef-wgt})
  followed by the factor to be used to perform the (renormalisation)
  scale variation in the computation of $\Gamma_{\rm{NNLO}}$ ({\tt
    nnlo\_renscfact}=$K_R^{\rm{NNLO}}$).
  
  For example, if one wants to perform a totally correlated scale
  variation (i.e. $K_R^{\rm{MiNLO}}=K_R^{\rm{NNLO}}$), then the
  {\tt{wgt.dat}} file should be
  \begin{lstlisting}[basicstyle=\small\ttfamily\color{black}]
    1   1.0
    2   0.5
    3   2.0
  \end{lstlisting}
  It is responsibility of the user to make sure that the correct order
  is used for the {\tt minlo\_wgt} values: in the above example, {\tt
    minlo\_wgt=1,2,3} must correspond to $K_R^{\rm{MiNLO}}=\{1.0,\,
  0.5,\, 2.0\}$, respectively.

  The file {\tt{wgt.dat}} that contains desired scale combinations
  should be passed to the reweighter programm using keyword
  {\tt{-weights}}, i.e.
  \begin{lstlisting}
    $> ../hdec -minnlo -hist hist.dat -events pwgevents.lhe -weights wgt.dat
  \end{lstlisting}
\end{itemize}

Note that while the $\Gamma_{\MINLO{}}$ is computed on the fly using
the event file generated, $\Gamma_{\rm NNLO}$ is computed internally
using the Higgs mass which is read off from the event file. The value
of the strong coupling used for calculation is by default set to
$\alpha_s(M_Z) = 0.1181$. This value can be changed using keyword
{\tt{-asmz}} followed by the desired value.



%###################################################################################################
\section{Interface of \hbbg{} events with a specific production mode}
\label{sec:interface}

In this section we describe how to combine a sample of \Hbb{} decays
(at \MINLO{} accuracy in the decay, or upgraded to NNLOPS) together
with a sample of events containing the production of an undecayed Higgs boson
and other final state particles (at NLOPS, \MINLO{}, or NNLOPS
accuracy in production).  The resulting event sample will contain the
full production kinematics, upgraded with the \Hbb{} decay at (N)NLO.

From now on we assume that the user has generated a LH event file
containing \Hbb{} decays following the steps described
in Sec.~\ref{sec:generation}. In the following we denote this file as {\tt
  pwgdecays.lhe} ({\tt pwgdecays-0001.lhe}, {\tt pwgdecays-0002.lhe},
..., if they were obtained through parallel runs). We also assume that
the user has generated events for Higgs production. We call this file
{\tt pwgproduction.lhe} ({\tt pwgproduction-0001.lhe}, {\tt
  pwgproduction-0002.lhe}, ..., if they were obtained through parallel
runs).

\begin{itemize}
\item Compile the {\tt{hdec}} program in the main \hbbg{} directory
  \begin{lstlisting}
    $> cd POWHEG-BOX-V2/h_bbg
    $> make hdec -f makefile.hdec
  \end{lstlisting}
  %%
\item The interface can be performed by running
  \begin{lstlisting}
    $> ../hdec -interface -decay pwgdecays.lhe -prod pwgproduction.lhe
  \end{lstlisting}
  The combined events will be saved into {\tt{pwgproduction.lhe-hbb}}
  file. The user should check that the number of decay events is
  greater or equal to the number of production events.  If more than
  one decay file is needed to satisfy this requirement, a number of
  files can be specified after the {\tt{-decay}} keyword.

  Note that, by default, program will prepare all possible
  combinations of the weights in the production and decay files. In
  order to constrain ththe possible scale combinations, the user
  should prepare a file {\tt{scales-comb.dat}} that lists desired
  combinations. Each line of this file should contain two integers,
  the first is the identifier of the production file weight and the
  second identifies the decay file weight. The file should be passed
  to the programm using {\tt{-weights}} keyword, i.e.
  \begin{lstlisting}
    $> ../hdec -interface -decay pwgdecays.lhe -prod pwgproduction.lhe
               -weights scale-comb.dat
  \end{lstlisting}

  For example, if the production file contains 7 weights,
  corresponding to renormalisation and factorisation scale variations
  \begin{align}
    (K_{R}^{\rm{prod}},K_{F}^{\rm{prod}}) = \{
    (1.0, 1.0),\,
    (1.0, 0.5),\,
    (1.0, 2.0),\,
    (0.5, 0.5),\,
    (0.5, 1.0),\,
    (2.0, 1.0),\,
    (2.0, 2.0)\,
    \}\,,
  \end{align}
  and the decay file contains 3 weights corresponding to
  \begin{align}
    K_{R}^{\rm{dec}} = \{ 1.0,\, 0.5,\, 2.0 \}\,,
  \end{align}
  and one wants to correlate the production and decay renormalisation
  scale variation, the file {\tt{scale-comb.dat}} should read
  \begin{lstlisting}[basicstyle=\small\ttfamily\color{black}]
    1   1
    2   1
    3   1
    4   2
    5   2
    6   3
    7   3
  \end{lstlisting}
  %%

  The default value of the branching ratio that is used for the \Hbb{}
  decay is set to $\rm{Br} = 0.5824$. This value can be modified by
  using the keyword {\tt{-BrHbb}} followed by the desired value of the
  branching ratio.
\end{itemize}



%###################################################################################################
\section{Matching with parton shower}
\label{sec:parton-shower}

In this section we describe how to parse \Hbb{} Les Houches events,
interfaced with a desired Higgs production channel, to a parton
shower.
%
We stress that since after the interface Les Houches record is a
compound prepared from two sources, production and decay of the Higgs
boson, and as such contains two separate veto scales:
\texttt{scalup\_prod} and \texttt{scalup\_dec} for production and
decay respectively.
%
Radiation generated by a parton shower, depending on its origin,
should respect the relevant bound.
%
We supply a parton shower driver that allows a user to process events
produced along the instructions of Sec.~\ref{sec:interface} with a
\pythia{} parton shower~\cite{Sjostrand:2014zea}.

In order to prepare a drive, one should
\begin{itemize}
  %--------------------
\item Prepare a desired phenomenological analysis by modifying a
  subroutines \texttt{init\_hists} and \texttt{analysis} of the
  \texttt{src/f77/pwhg\_analysis.f} file, by default it contains only
  an entry for a calculation of the total cross section.

  It is possible to use the default \POWHEG{} histogramming facility or
  link user's external analysis. The connection should be established
  withing the \texttt{src/f77/pwhg\_analysis.f} file.
  
  %--------------------
\item Compile the \pythia{} driver in the main \hbbg{}
  directory.
  \begin{lstlisting}
    $> cd POWHEG-BOX-V2/h_bbg
    $> make main-PYTHIA-lhef
  \end{lstlisting}
  %--------------------
\item To process parton shower, navigate to directory with Les Houches
  event files and run the shower driver
  \begin{lstlisting}
    $> cd testrun
    $> echo <event-file> | ../main-PYTHIA-lhef
  \end{lstlisting}
  where \texttt{<event-file>} is an event file to be
  showered. Successful run should produce histogram files with a
  \texttt{.top} extension.
\end{itemize}

We note that a state-of-the-art \pythia{} parton shower,
provides its own facility to veto emissions off various
resonances. For that reason it is possible to use the built-in tools
known as the \texttt{UserHooks}.
%
For more details please refer to the original \pythia{}
reference guide on this topic~\cite{Pythia8UserHooks,Pythia8POWHEG}.


\appendix
%###################################################################################################
\section{Physical input parameters}
\label{app:input}
A list of physical parameters that can be adjusted by the user in the
\texttt{powheg.input} card. Floating point numbers should be simply
specified as numbers, the logical variables are specified with 0/1 for
True/False, respectively.

\begin{itemize}
  %%%
\item \texttt{hmass} [float]: mass of the Higgs boson, in \gev{}.
  %%%
\item \texttt{hwidth} [float]: total width of the Higgs boson resonance, in
  \gev{}.
  %%%
\item \texttt{bmass} [float]: $b$-quark pole mass, in \gev{}, used for
  calculation of the Yukawa coupling (the dependence cancels out once
  an \Hbb{} branching fraction is applied for during the
  interface). The kinematics of the decay is always calculated with
  massless $b$-quarks.
  %%%
\item \texttt{mb\_running} [logical]: if False the pole $b$-quark
  Yukawa will not be transformed into $\overline{\mathrm{MS}}$ one
  calculated at the renormalization scale.

\item \texttt{massive\_b\_lhe} [logical]: if True the Les Houches
  event file will contain massive $b$-quarks (calculation is performed
  in the massless approximation, reshuffling is applied at the stage
  of generating of the Les Houches file)
\end{itemize}

Moreover we have
\begin{itemize}
  %%%
\item \texttt{renscfact} [float]: modification of a renormalization
  scale, $\mu_{R} = \texttt{renscfact} \cdot \mu_0$, with $\mu_0$
  equal to the Higgs mass.
  %%%
\item \texttt{xresummation} [float]: modification of the resummation
  scale, $Q_{\text{res}} = \texttt{xresummation}\cdot Q_{0}$, with
  $Q_0$ equal to the Higgs mass.
  %%%
\item \texttt{sudscalevar} [logical]: if True scale variation is also
  performed in the Sudakov radiator.
\end{itemize}


%###################################################################################################
\section{Technical parameters}
\label{app:tech}
All technical parameters in the \texttt{powheg.input} card can be modified by the user. The most relevant include: 
\begin{itemize}
  %%%
\item \texttt{itmx1}/\texttt{itmx2} [integer]: number of Monte Carlo
  subsequent runs at \texttt{stage1} and \texttt{stage2} of a
  \POWHEG{} run (if using parallel version of the run, \texttt{itmx1}
  is ignored and set to 1).
  %%%
\item \texttt{ncall1}/\texttt{ncall2} [integer]: number of Monte Carlo
  sampling points at \texttt{stage1} and \texttt{stage2} of a
  \POWHEG{} run.
  %%%
\item \texttt{nubound} [integer]: number of sampling points at
  \texttt{stage3} of a \POWHEG{} run.
  %%%
\item \texttt{numevts} [integer]: number of requested event records in
  a single Les Houches event file.
\end{itemize}




%###################################################################################################
\begin{thebibliography}{10}

  %\cite{hbbg-paper}
\bibitem{hbbg-paper}
  W.~Bizon, E.~Re, and G.~Zanderighi,
  ``NNLOPS description of the H$\to\! b\bar{b}$ decay with \MINLO{},''
  %JHEP {\bf 19xx} (2019) xxx
  %doi:10.1007/JHEPxx(2019)xxx
  %[arXiv:19xx.xxxxx [hep-ph]]
  .
  
  %\cite{Nason:2004rx}
\bibitem{Nason:2004rx}
  P.~Nason,
  ``A new method for combining NLO QCD with shower Monte Carlo algorithms,''
  JHEP {\bf 0411} (2004) 040
  [arXiv:hep-ph/0409146].
  %%CITATION = JHEPA,0411,040;%%

  %\cite{Frixione:2007vw}
\bibitem{Frixione:2007vw}
  S.~Frixione, P.~Nason and C.~Oleari,
  ``Matching NLO QCD computations with Parton Shower simulations: the POWHEG method,''
  JHEP {\bf 0711} (2007) 070
  [arXiv:0709.2092].
  %%CITATION = JHEPA,0711,070;%%

  %\cite{Alioli:2010xd}
\bibitem{Alioli:2010xd}
  S.~Alioli, P.~Nason, C.~Oleari and E.~Re,
  ``A general framework for implementing NLO calculations in shower Monte Carlo programs: the POWHEG BOX,''
  JHEP {\bf 1006} (2010) 043
  [arXiv:1002.2581].

  %\cite{Jezo:2015aia}
\bibitem{Jezo:2015aia}
  T.~Jezo and P.~Nason,
  ``On the Treatment of Resonances in Next-to-Leading Order Calculations Matched to a Parton Shower,''
  JHEP {\bf 1512} (2015) 065
  [arXiv:1509.09071].

  %\cite{Hamilton:2012np}
\bibitem{Hamilton:2012np}
  K.~Hamilton, P.~Nason and G.~Zanderighi,
  %``MINLO: Multi-Scale Improved NLO,''
  JHEP {\bf 1210} (2012) 155
  doi:10.1007/JHEP10(2012)155
  [arXiv:1206.3572 [hep-ph]].

  %\cite{Hamilton:2012rf}
\bibitem{Hamilton:2012rf}
  K.~Hamilton, P.~Nason, C.~Oleari and G.~Zanderighi,
  %``Merging H/W/Z + 0 and 1 jet at NLO with no merging scale: a path to parton shower + NNLO matching,''
  JHEP {\bf 1305} (2013) 082
  doi:10.1007/JHEP05(2013)082
  [arXiv:1212.4504 [hep-ph]].

  %\cite{Sjostrand:2014zea}
\bibitem{Sjostrand:2014zea}
  T.~Sjöstrand {\it et al.},
  %``An Introduction to PYTHIA 8.2,''
  Comput.\ Phys.\ Commun.\  {\bf 191} (2015) 159
  doi:10.1016/j.cpc.2015.01.024
  [arXiv:1410.3012 [hep-ph]].
  %%CITATION = doi:10.1016/j.cpc.2015.01.024;%%
  %1907 citations counted in INSPIRE as of 22 Oct 2019

\bibitem{Pythia8UserHooks}
  \url{http://home.thep.lu.se/Pythia/pythia82html/UserHooks.html}

\bibitem{Pythia8POWHEG}
  \url{http://home.thep.lu.se/Pythia/pythia82html/POWHEGMerging.html}
  
\end{thebibliography}


\end{document}
