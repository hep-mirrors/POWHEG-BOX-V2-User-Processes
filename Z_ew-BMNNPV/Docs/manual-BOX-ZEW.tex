\documentclass[paper]{JHEP3}
\usepackage{amsmath,amssymb,enumerate,url}
\bibliographystyle{JHEP}
\usepackage{lineno}
%\linenumbers

%%%%%%%%%% Start TeXmacs macros
\newcommand{\tmtextit}[1]{{\itshape{#1}}}
\newcommand{\tmtexttt}[1]{{\ttfamily{#1}}}
\newenvironment{enumeratenumeric}{\begin{enumerate}[1.] }{\end{enumerate}}
\newcommand\sss{\mathchoice%
{\displaystyle}%
{\scriptstyle}%
{\scriptscriptstyle}%
{\scriptscriptstyle}%
}

\newcommand\as{\alpha_{\sss\rm S}}
\newcommand\POWHEG{{\tt POWHEG}}
\newcommand\POWHEGBOX{{\tt POWHEG BOX}}
\newcommand\HERWIG{{\tt HERWIG}}
\newcommand\PYTHIASIX{{\tt PYTHIA6}}
\newcommand\PYTHIAEIGHT{{\tt PYTHIA8}}
\newcommand\PYTHIA{{\tt PYTHIA}}
\newcommand\PHOTOS{{\tt PHOTOS}}
\newcommand\PHOTOSPP{{\tt PHOTOS++}}
\newcommand\MCatNLO{{\tt MC@NLO}}
\newcommand\DIR{{\tt Z\_ew-BMNNPV}}

\newcommand\kt{k_{\sss\rm T}}

\newcommand\pt{p_{\sss\rm T}}
\newcommand\LambdaQCD{\Lambda_{\scriptscriptstyle QCD}}
%%%%%%%%%% End TeXmacs macros


\title{The $\boldsymbol{Z/\gamma^*}$ EW NLO \& QCD production \\
in the POWHEG-BOX-V2 user manual: svn v3371} \vfill
\author{Mauro Chiesa\\
  INFN, Sezione di Pavia, Via A. Bassi 6, 27100 Pavia, Italy\\
  E-mail: \email{Mauro.Chiesa@physik.uni-wuerzburg.de}}
\author{Homero Martinez\\
  Dipartimento di Fisica, Universit\`a di Pavia and INFN, 
  Sezione di Pavia, \\
  Via A. Bassi 6, 27100 Pavia, Italy\\
  E-mail: \email{Homero.Martinez@pv.infn.it}}
\author{Guido Montagna\\
  Dipartimento di Fisica, Universit\`a di Pavia and INFN, 
  Sezione di Pavia, \\
  Via A. Bassi 6, 27100 Pavia, Italy\\
  E-mail: \email{Guido.Montagna@pv.infn.it}}
\author{Paolo Nason\\
  INFN, Sezione di Milano-Bicocca,
  Piazza della Scienza 3, 20126 Milan, Italy\\
  E-mail: \email{Paolo.Nason@mib.infn.it}}
\author{Oreste Nicrosini\\
  INFN, Sezione di Pavia, Via A. Bassi 6, 27100 Pavia, Italy\\
  E-mail: \email{Oreste.Nicrosini@pv.infn.it}}
\author{Fulvio Piccinini\\
  INFN, Sezione di Pavia, Via A. Bassi 6, 27100 Pavia, Italy\\
  E-mail: \email{Fulvio.Piccinini@pv.infn.it}}
\author{Alessandro Vicini\\
  Dipartimento di Fisica, Universit\`a di Milano and INFN, 
  Sezione di Milano, \\
  Via Celoria 16, I-20133 Milano, Italy\\
  E-mail: \email{Alessandro.Vicini@mi.infn.it}}
\vskip -0.5truecm

\keywords{POWHEG, Shower Monte Carlo, NLO, Electroweak}

\abstract{This note documents the use of the package
  \tmtexttt{POWHEG-BOX-V2} for $Z/\gamma^*$ production processes including
  QCD and ElectroWeak NLO corrections.
  Results can be easily interfaced to shower Monte Carlo programs, in
  such a way that both NLO and shower accuracy are maintained.}
\preprint{\today\\ \tmtexttt{POWHEG-BOX-V2}}

\begin{document}


\section{Introduction}

The \tmtexttt{POWHEG BOX} program is a framework for implementing NLO
calculations in Shower Monte Carlo programs according to the
\POWHEG{} method. An explanation of the method and a discussion of
how the code is organized can be found in
Refs.~\cite{Nason:2004rx,Frixione:2007vw,Alioli:2010xd}. The code is
distributed according to the ``MCNET GUIDELINES for Event Generator
Authors and Users'' and can be found at the web page \\
%
\begin{center}
 \url{http://powhegbox.mib.infn.it}.
\end{center}
%
~\\
%
This program is an implementation of the Drell-Yan NLO cross sections
$pp\to Z/\gamma^* \to \ell^+ \ell^-$ including QCD and ElectroWeak (EW) 
radiative corrections. 
A detailed description of the implementation can be found in 
Ref.~\cite{Barze:2013}. Major improvements w.r.t. svn version 3370 
in Ref.~\cite{CarloniCalame:2016ouw}. 
%Please cite the paper when you use the program. \\
In order to run the \tmtexttt{POWHEG BOX} program, we recommend the
reader to start from the \tmtexttt{POWHEG BOX} user manual, which
contains all the information and settings that are common between
all subprocesses. In this note we focus on  the settings and
parameters specific to the $Z/\gamma^*$ implementation.

\section{Generation of events}
\label{sec:gen_events}

Build the executable\\

\noindent\tmtexttt{\$ cd POWHEG-BOX-V2/\DIR{} \\
\$ make pwhg\_main }\\

Then do (for example) \\

\noindent\tmtexttt{\$
cd runtest-lhc-8TeV\\
\$ ../pwhg\_main}\\

At the end of the run, the file \tmtexttt{pwgevents.lhe} will contain 1000 events for $Z/\gamma^* \to e^+ e^-$ in the Les Houches format (\tmtexttt{.lhe} file). 

We provide four executables, called ``interfaces", which process the events in the file \tmtexttt{pwgevents.lhe} and give them as inputs to parton shower programs, performing the required vetoes and setting the required flags in a manner consistent with the physical accuracy of the input events. \PYTHIA{} is used to perform the QCD shower. The QED shower can be done using \PYTHIA{} or the independent program \PHOTOS{}~\cite{Golonka:2005pn}. The four interfaces are:

\begin{itemize}
\item \tmtexttt{main-PYTHIA81-lhef} : In this interface, \PYTHIAEIGHT{} (versions 8.1xx) is used to perform the QCD shower, while the QED shower can be done using \PHOTOSPP{} (C++ version of \PHOTOS{}), or \PYTHIAEIGHT{}. It is designed to work with versions of \PYTHIAEIGHT{} up to 8.186. It has been tested with \PYTHIA{} version 8.185 and \PHOTOSPP{} version 3.56. The output of the shower can be analyzed looking at the histograms in the output \tmtexttt{.top} file. Optionally, the events after the shower can be saved in another \tmtexttt{.lhe} file (see section~\ref{sec:flags}).

Even if you do not use \PHOTOSPP{}, it has to be linked. The minimal procedure is to use the version 3.56 included in the \DIR{} package, doing:
\\~\\
\noindent\tmtexttt{\$ 
cd PHOTOS \\ 
\$./configure --without-hepmc  \\ 
\$ make}
\\

The flag \verb+PHOTOSCC_LOCATION+ must be set in the \tmtexttt{Makefile} to the path of installation of \PHOTOSPP{}. This path can be the folder \tmtexttt{\DIR{}/PHOTOS} if the user wants to use the included version of \PHOTOSPP{}, otherwise it should be set to the folder of an external installation.

\PYTHIAEIGHT{} has to be downloaded and compiled by the user. The script \tmtexttt{pythia8-config} should configure automatically the path to the \PYTHIAEIGHT{} installation in the \tmtexttt{Makefile}, if this is not the case, the user must set manually the \verb+PYTHIA8LOCATION+ flag. 

Once \PHOTOSPP{} and \PYTHIAEIGHT{} are compiled and the flags are set properly in the \tmtexttt{Makefile}, the interface can be compiled, doing:
\\~\\
\tmtexttt{
\$ cd POWHEG-BOX-V2/\DIR{} \\ 
\$ make main-PYTHIA81-lhef}
\\

Before running, the path to \PHOTOSPP{} libraries needs to be added to the list of dynamically linked libraries, and a variable pointing to the path of \PYTHIAEIGHT{} particle data (\tmtexttt{.xml}) files) needs to be set. In order to do this, the script \tmtexttt{setlibrarypaths.sh} must be edited to point to the correct paths, and then executed in the current shell, doing:
\\~\\
\tmtexttt{\$ source setlibrarypaths.sh }\\

To run the interface, do:
\\~\\
\tmtexttt{
\$ cd runtest-lhc-14TeV-wp \\
\$ ../main-PYTHIA81-lhef}
\\

\item \tmtexttt{main-PYTHIA82-lhef} : Similar interface to \tmtexttt{main-PYTHIA81-lhef}, but designed to work with newer versions of \PYTHIA{} (versions 8.2xx). It has been tested with version 8.223. The compilation / linking procedure is similar to the one described in the previous item.

\item \tmtexttt{main-PYTHIA6-lhef} : Interface to \PYTHIASIX{} and \PHOTOS{} (both codes written in Fortran). The source codes of \PYTHIASIX{} and \PHOTOS{} are included in the \POWHEGBOX{} and in the \DIR{} folder, respectively, so no flag or linking to external libraries need to be done. To compile and run, simply do:
\\~\\
\noindent\tmtexttt{
\$ cd POWHEG-BOX-V2/\DIR{} \\ 
\$ make main-PYTHIA6-lhef \\ 
\$ cd runtest-lhc-14TeV \\
\$ ../main-PYTHIA6-lhef}
\\

\item \tmtexttt{main-PHOTOS-lhef} : Interface to \PHOTOSPP{}, it processes the \POWHEG{} generated events, calls the QED final state shower implemented by \PHOTOSPP{} and generates a new event file in LHE format, that can be then interfaced to a QCD shower program, where QED radiation must be switched off to avoid double counting. The compilation and linking requires the \PHOTOS{} related steps described in item (1).

\end{itemize}

The customization of the histograms and cuts provided in the output \tmtexttt{.top} file (for interfaces \tmtexttt{main-PYTHIA81-lhef}, \tmtexttt{main-PYTHIA82-lhef} and \tmtexttt{main-PYTHIA6-lhef}), can be done modifying the code in \verb+pwhg_analysis.f+ (\tmtexttt{analysis} subroutine). In addition, C++ based analysis can be done accessing the events from the files \tmtexttt{pythia81F77.cc}, \tmtexttt{pythia82F77.cc} (function \verb+pythia_next_+) or in the file \tmtexttt{photosCCF.cc} (function \verb+photos_process_+), for the interfaces \tmtexttt{main-PYTHIA81-lhef}, \tmtexttt{main-PYTHIA82-lhef} and \tmtexttt{main-PHOTOS-lhef}, respectively.


\section{Process specific input parameters}
\label{sec:flags}

All the parameters and flags are set in the input card file \tmtexttt{powheg.input}. The mandatory parameters are those needed to select the final state
leptonic species coming from the vector-boson:
~\\
\tmtexttt{
  vdecaymode 11 !  code for selected Z decay\\
  \phantom{aaaaaaaaaaaaa} ! (11(-11): electronic; 13(-13): muonic; 15(-15): tauonic)}
\\
\\
The decay $Z \to \nu \bar \nu$ is not handled in the present version. 
\\
\\
In addition to the mandatory parameters, the \POWHEGBOX{} input allows for an easy 
setting of EW and run parameters, by explicitly adding the relevant 
lines to the input card. 
If one of the following entries is not present in the input card  
the reported default 
value is assumed. In any case, these parameters are printed 
in the output of the program, 
so their values can be easily tracked down. 
~\\~\\
 \tmtexttt{
   Wmass\phantom{aa}  80.398 \phantom{aaaaaaa} ! W mass in GeV\\
   Wwidth\phantom{a} 2.141 \phantom{aaaaaaaa} ! W width in GeV\\
   Zmass\phantom{aa}  91.1876 \phantom{aaaaaa}   ! Z mass in GeV\\
   Zwidth\phantom{a} 2.4952 \phantom{aaaaaaa} ! Z width in GeV\\
   alphaem 0.00729735254 \phantom{a}! em coupling alpha(0)\\
   gmu\phantom{aaaa} 1.16637d-5 \phantom{aaa} ! Fermi constant in GeV\^{}-2\\
   Hmass\phantom{aa} 120. \phantom{aaaaaaaa} ! Higgs mass in GeV\\
   Tmass\phantom{aa} 172.9 \phantom{aaaaaaaa} ! Top mass in GeV\\
   Bmass\phantom{aaaa} 4.6 \phantom{aaaaaaaa} ! B quark mass in GeV\\
   Cmass\phantom{aaaa} 1.2 \phantom{aaaaaaaa} ! C quark mass in GeV\\
   Smass\phantom{aaa} 0.15 \phantom{aaaaaaaa} ! S quark mass in GeV\\
   Umass\phantom{aaa} 0.06983 \phantom{aaaaa} ! U quark mass in GeV\\
   Dmass\phantom{aaa} 0.06984 \phantom{aaaaa} ! D quark mass in GeV\\
   Elmass\phantom{aa} 0.005109989 \phantom{a} ! Electron mass in GeV\\
   Mumass\phantom{aa} 0.105658369 \phantom{a} ! Mu mass in GeV\\
   Taumass\phantom{a} 1.77699 \phantom{aaaaa} ! Tau mass in GeV\\
}
\\
The following parameter limits from below the virtuality of the $Z$ boson:
\\
\tmtexttt{mass\_low 20 \phantom{aaaaaaaa}! Z virtuality > mass\_low in GeV}\\
\\
If absent, it is set to 30~GeV.
In order to avoid edge effects, the lower limit \tmtexttt{mass\_low} 
should be more inclusive w.r.t. cuts applied at the analysis level.
Notice that, if photons are generated, the $Z$ virtuality is not
necessarily the mass of the dilepton.
\\
\\
\tmtexttt{runningscale 0 \phantom{aaaa}! choice for ren and fac scales in Bbar integration}\\
\phantom{pppppppppppppppppppppppppppppp} \tmtexttt{ 0: fixed scale M\_Z}\\
\phantom{pppppppppppppppppppppppppppppp} 
\tmtexttt{ 1: running scale $\ell^+ \ell^- (\gamma)$ inv mass }\\ 
\phantom{pppppppppppppppppppppppppppppppppp} 
\tmtexttt{ $\gamma$ included with QED FSR }\\ 

With running scale, a minimum cutoff of 5 GeV is imposed on $m(\ell^+ \ell^-)$.
\\
\\
\tmtexttt{scheme 1! choice for EW NLO scheme calculation}\\
\phantom{pppppppppppppppppppppppppppppp} \tmtexttt{ 0: Alpha(0)}\\
\phantom{pppppppppppppppppppppppppppppp} \tmtexttt{ 1: Alpha(M\_Z)}\\
\phantom{pppppppppppppppppppppppppppppp} \tmtexttt{ 2: G\_$\mu$}\\
\\ 
The CKM mixing matrix is assumed diagonal in the EW NLO corrections.\\
\\
The EW radiative corrections can be calculated according to three 
different schemes: the $\alpha(0)$ scheme, where the input parameters 
are $\alpha(0)$, $M_W$ and $M_Z$; the $\alpha(M_Z^2)$ scheme, where the 
input parameters are $\alpha(M_Z^2)$, $M_W$ and $M_Z$ (with this 
scheme the value of the parameter \tmtexttt{alphaem\_z} should be specified); 
the $G_\mu$ scheme, where the 
input parameters are $G_\mu$, $M_W$ and $M_Z$. 
\\
The EW corrections can be switched off by setting
\\
\tmtexttt{no\_ew 1       ! default 0}
\\
and the strong corrections can be switched off by setting
\\
\tmtexttt{no\_strong 1       ! default 0}
\\
This last option is just to check EW corrections at the NLO level (i.e.,
the Les Houches events do not have much meaning).
\\
\\
\subsection{Flags used by the shower interfaces}

The flags starting with \verb+SI_+ configure the behavior of the shower interfaces. The following flags are used by the interfaces \tmtexttt{main-PYTHIA81-lhef} and \tmtexttt{main-PYTHIA82-lhef}:

\begin{itemize}
\item[] \verb+SI_inputfile+: Configure the input file for the shower interfaces (default: \tmtexttt{./pwgevents.lhe}).
\item[]\verb+SI_maxshowerevents+: Number of events to read from the input file (default: all events).
\item[]\verb+SI_pythiamatching+, explained below.
\item[]\verb+SI_pytune+: Set the \PYTHIA{} tune used in the QCD shower.
\item[]\verb+SI_dopythiaqed+: Turn ON / OFF \PYTHIA{} QED shower (default OFF).
\item[]\verb+SI_use_photos+: Turn ON / OFF the final state QED shower by \PHOTOS{} (default OFF). If it is ON, it automatically sets off the final state QED radiation from \PYTHIAEIGHT{}.
\item[]\verb+SI_kt2minqed+: Set value of photos low energy cut off (default is $10^{-6}$).
\item[]\verb+SI_usepy8veto+, explained below.
\item[]\verb+SI_nohad+: Allows to switch OFF the hadronization (the hadronization is ON by default).
\item[]\verb+SI_savelhe+: Turn ON / OFF the production of an output LHE file (default OFF).
\item[]\verb+SI_savehistos+: Turn ON / OFF the production of histograms in \tmtexttt{.top} file (default OFF).
\item[]\verb+SI_no_tworadevents+, explained below.
\end{itemize}

According to the \POWHEG{} method, the radiation by the shower has to be generated from a starting scale given by the hardest $\pt$ tried at the matrix element level. Traditionally, this scale is written in the variable {\tt scalup} in the {tt LHE} event file. However, for the DY process, when both QCD and EW NLO corrections are present (Ref.~\cite{Barze:2012tt}), it is necessary to keep track in the \POWHEG{} events of two scales, one for initial state radiation ({\tt scalup-isr}) and one for final state radiation ({\tt scalup-fsr}). These two scales are used as starting points for the IS and FS showers, respectively. \PYTHIAEIGHT{} and \PHOTOSPP{} do not use {\tt scalup-fsr} for the generation of QED final state radiation from the $W$. Hence, in order to avoid double counting of QED radiation, a veto algorithm is necessary. This veto is activated unless the flag \verb+no_ew+ is activated in \tmtexttt{powheg.input}.

A general issue is the matching between the NLO calculation and the (QCD and QED) higher order corrections given by the parton shower: due to the different definitions of $p_\perp$ in \POWHEG{} and \PYTHIAEIGHT{}, some double counting or dead zone can arise. By default (flag \verb+SI_pythiamatching+ equal to~2), the interface generates all QCD/QED shower emissions up to the kinematical limit and then veto emissions harder than the \POWHEG{} scales ({\tt scalup-isr} and {\tt scalup-fsr}), computed according to the \POWHEG{} $p_\perp$ definition. The user can optionally choose an alternative scheme, where the shower starting scales are fixed and no veto's are performed. This choice can be activated by setting the flag \verb+SI_pythiamatching+ equal to~1.

When FS QED radiation is implement by \PHOTOSPP{}, a Fortran-based function is used to veto QED emissions and perform the correct matching (implemented in the file \tmtexttt{main-PYTHIA-8.f}). If FS QED radiation is done by \PYTHIAEIGHT{}, the matching can be done using Fortran-based algorithms coded in the files \tmtexttt{main-PYTHIA-8.f} and \tmtexttt{scalupveto.f} (flag \verb+SI_usepy8veto+ equal to~0), or alternatively, use the matching algorithms provided by \PYTHIAEIGHT{} (flag \verb+SI_usepy8veto+ equal to~1). Those algorithms use the {\tt UserHooks} functions, and their behavior depend on the value of the flag \verb+SI_pythiamatching+.

In the case the interface is used to read events where the two scales {\tt scalup-isr} and {\tt scalup-fsr} are not present, and {\tt scalup} is to be used as starting scale for the showers, the flag \verb+SI_no_tworadevents+ should be set to~1.

The decay of hadronic resonances which can proceed radiatively has been 
suppressed. In order to let the resonances decay, the user should 
open the file \tmtexttt{pythia81F77.cc} (or \tmtexttt{pythia82F77.cc}) and comment the relevant lines in the function {\tt pythia\_init}.

The interface \tmtexttt{main-PYTHIA6-lhef} can be customized using the following flags: \verb+SI_inputfile+, \verb+SI_maxshowerevents+, \verb+SI_pytune+, 
\verb+SI_use_photos+, \verb+SI_kt2minqed+, \verb+SI_savehistos+, \verb+SI_no_tworadevents+, read also from the \tmtexttt{powheg.input} file. The other flags have no effect. Notice that in this case, the QED radiation from \PYTHIA{} is ON by default.

The interface \tmtexttt{main-PHOTOS-lhef} uses only the relevant flags, namely:  \verb+SI_inputfile+, \verb+SI_maxshowerevents+, \verb+SI_kt2minqed+.

For further customization of the settings used by the shower interfaces, beyond the flags available in \tmtexttt{powheg.input}, the user can modify the following source code files:

\begin{itemize}
\item Interface \tmtexttt{main-PYTHIA81-lhef}: Settings in file \tmtexttt{pythia81F77.cc}.
\item Interface \tmtexttt{main-PYTHIA82-lhef}: Settings in file \tmtexttt{pythia82F77.cc}.
\item Interface \tmtexttt{main-PYTHIA6-lhef}: Settings in file \tmtexttt{setup-PYTHIA6-lhef.f}
\item Interface \tmtexttt{main-PHOTOS-lhef}: Settings in file \tmtexttt{photosCCF.cc}.
\end{itemize}




\begin{thebibliography}{10}

\bibitem{Nason:2004rx}
  P.~Nason,
  ``A new method for combining NLO QCD with shower Monte Carlo algorithms,''
  JHEP {\bf 0411} (2004) 040
  [arXiv:hep-ph/0409146].
  %%CITATION = JHEPA,0411,040;%%

%\cite{Frixione:2007vw}
\bibitem{Frixione:2007vw}
  S.~Frixione, P.~Nason and C.~Oleari,
``Matching NLO QCD computations with Parton Shower simulations: the POWHEG
method,''
  JHEP {\bf 0711} (2007) 070
  [arXiv:0709.2092 [hep-ph]].
  %%CITATION = JHEPA,0711,070;%%

%\cite{Alioli:2010xd}
\bibitem{Alioli:2010xd}
  S.~Alioli, P.~Nason, C.~Oleari and E.~Re,
``A general framework for implementing NLO calculations in shower Monte Carlo
  programs: the POWHEG BOX,''
  [arXiv:1002.2581 [hep-ph]].
  %%CITATION = ARXIV:1002.2581;%%

\bibitem{Barze:2013}
  L.~Barz\`e, G.~Montagna, P.~Nason, O.~Nicrosini, F.~Piccinini and 
  A.~Vicini, 
  %``Neutral current Drell-Yan with combined QCD and
  %                      electroweak corrections in the POWHEG BOX,''
  Eur.\ Phys.\ J.\ C {\bf 73} (2013) 2474  
  [arXiv:1302.4606[hep-ph]].

\bibitem{CarloniCalame:2016ouw}
  C.M.~Carloni Calame, M.~Chiesa, H.~Martinez, 
  G.~Montagna, O.~Nicrosini, F.~Piccinini, and A.~Vicini, 
  %``Precision Measurement of the W-Boson Mass: Theoretical
  %                      Contributions and Uncertainties,''
  [arXiv:1612.02841[hep-ph]].

%\cite{Golonka:2005pn}
\bibitem{Golonka:2005pn}
  P.~Golonka and Z.~Was,
  %``PHOTOS Monte Carlo: A Precision tool for QED corrections in $Z$ and $W$ decays,''
  Eur.\ Phys.\ J.\ C {\bf 45} (2006) 97
  [hep-ph/0506026].
  %%CITATION = HEP-PH/0506026;%%

\end{thebibliography}

\end{document}





