\documentclass[paper]{JHEP3}
\usepackage{amsmath,amssymb,enumerate,url}
\bibliographystyle{JHEP}

%%%%%%%%%% Start TeXmacs macros
\newcommand{\tmtextit}[1]{{\itshape{#1}}}
\newcommand{\tmtexttt}[1]{{\ttfamily{#1}}}
\newenvironment{enumeratenumeric}{\begin{enumerate}[1.] }{\end{enumerate}}
\newcommand\sss{\mathchoice%
{\displaystyle}%
{\scriptstyle}%
{\scriptscriptstyle}%
{\scriptscriptstyle}%
}

\newcommand\as{\alpha_{\sss\rm S}}
\newcommand\POWHEG{{\tt POWHEG}}
\newcommand\POWHEGBOX{{\tt POWHEG BOX}}
\newcommand\HERWIG{{\tt HERWIG}}
\newcommand\PYTHIASIX{{\tt PYTHIA6}}
\newcommand\PYTHIAEIGHT{{\tt PYTHIA8}}
\newcommand\PHOTOS{{\tt PHOTOS}}
\newcommand\MCatNLO{{\tt MC@NLO}}

\newcommand\kt{k_{\sss\rm T}}

\newcommand\pt{p_{\sss\rm T}}
\newcommand\LambdaQCD{\Lambda_{\scriptscriptstyle QCD}}
%%%%%%%%%% End TeXmacs macros


\title{The $\boldsymbol{Z/\gamma^*}$ EW NLO \& QCD production \\
in the POWHEG-BOX-V2 user manual} \vfill
\author{Luca Barz\`e\\
  PH-TH Department, CERN, CH-1211 Geneva 23, Switzerland\\
  E-mail: \email{Luca.Barze@cern.ch}}
\author{Guido Montagna\\
  Dipartimento di Fisica, Universit\`a di Pavia and INFN, 
  Sezione di Pavia, \\
  Via A. Bassi 6, 27100 Pavia, Italy\\
  E-mail: \email{Guido.Montagna@pv.infn.it}}
\author{Paolo Nason\\
  INFN, Sezione di Milano-Bicocca,
  Piazza della Scienza 3, 20126 Milan, Italy\\
  E-mail: \email{Paolo.Nason@mib.infn.it}}
\author{Oreste Nicrosini\\
  INFN, Sezione di Pavia, Via A. Bassi 6, 27100 Pavia, Italy\\
  E-mail: \email{Oreste.Nicrosini@pv.infn.it}}
\author{Fulvio Piccinini\\
  INFN, Sezione di Pavia, Via A. Bassi 6, 27100 Pavia, Italy\\
  E-mail: \email{Fulvio.Piccinini@pv.infn.it}}
\author{Alessandro Vicini\\
  Dipartimento di Fisica, Universit\`a di Milano and INFN, 
  Sezione di Milano, \\
  Via Celoria 16, I-20133 Milano, Italy\\
  E-mail: \email{Alessandro.Vicini@mi.infn.it}}
\vskip -0.5truecm

\keywords{POWHEG, Shower Monte Carlo, NLO, Electroweak}

\abstract{This note documents the use of the package
  \tmtexttt{POWHEG-BOX-V2} for $Z/\gamma^*$ production processes including
  QCD and ElectroWeak NLO corrections.
  Results can be easily interfaced to shower Monte Carlo programs, in
  such a way that both NLO and shower accuracy are maintained.}
\preprint{\today\\ \tmtexttt{POWHEG-BOX-V2}}

\begin{document}


\section{Introduction}

The \tmtexttt{POWHEG BOX} program is a framework for implementing NLO
calculations in Shower Monte Carlo programs according to the
\POWHEG{} method. An explanation of the method and a discussion of
how the code is organized can be found in
Refs.~\cite{Nason:2004rx,Frixione:2007vw,Alioli:2010xd}. The code is
distributed according to the ``MCNET GUIDELINES for Event Generator
Authors and Users'' and can be found at the web page \\
%
\begin{center}
 \url{http://powhegbox.mib.infn.it}.
\end{center}
%
~\\
%
This program is an implementation of the Drell-Yan NLO cross sections
$pp\to Z/\gamma^* \to \ell^+ \ell^-$ including QCD and ElectroWeak (EW) 
radiative corrections. 
A detailed description of the implementation can be found in 
Ref.~\cite{Barze:2013}. 
%Please cite the paper when you use the program. \\
In order to run the \tmtexttt{POWHEG BOX} program, we recommend the
reader to start from the \tmtexttt{POWHEG BOX} user manual, which
contains all the information and settings that are common between
all subprocesses. In this note we focus on  the settings and
parameters specific to the $Z/\gamma^*$ implementation.

\section{Generation of events}

Build the executable\\
\tmtexttt{\$ cd POWHEG-BOX-V2/Z\_ew-BMNNPV \\
\$ make pwhg\_main }\\
Then do (for example) \\
\tmtexttt{\$
cd runtest-lhc-8TeV\\
\$ ../pwhg\_main}\\
At
the end of the run, the file \tmtexttt{pwgevents.lhe} will contain
100000 events for $Z/\gamma^* \to e^+ e^-$  in the Les Houches format. 
In order to shower them with \PYTHIASIX{} do\\~\\
\tmtexttt{\$
cd POWHEG-BOX-V2/Z\_ew-BMNNPV \\ \$ make
main-PYTHIA-lhef \\ \$ cd
runtest-lhc-8TeV \\
\$ ../main-PYTHIA-lhef}
\\
\\
If you prefer to shower the event with \PYTHIAEIGHT{} do\\~\\
\tmtexttt{\$
cd POWHEG-BOX-V2/Z\_ew-BMNNPV \\ \$ make
main-PYTHIA8-lhef \\ \$ cd
runtest-lhc-8TeV \\
\$ ../main-PYTHIA-lhef}


\section{Process specific input parameters}

Mandatory parameters are those needed to select the final state
leptonic species coming from the vector-boson:
~\\
\tmtexttt{
  vdecaymode 11 !  code for selected Z decay\\
  \phantom{aaaaaaaaaaaaa} ! (11(-11): electronic; 13(-13): muonic; 15(-15): tauonic)}
\\
\\
The decay $Z \to \nu \bar \nu$ is not handled in the present version. 
\\
\\
Together with the mandatory parameters, the \POWHEGBOX{} input facility 
allows for an easy 
setting of EW and run parameters, by explicitly adding the relevant 
lines to the input card. 
If one of the following entries is not present in the input card  
the reported default 
value is assumed. In any case, these parameters are printed 
in the output of the program, 
so their values can be easily tracked down. 
~\\~\\
 \tmtexttt{
   Wmass\phantom{aa}  80.398 \phantom{aaaaaaa} ! W mass in GeV\\
   Wwidth\phantom{a} 2.141 \phantom{aaaaaaaa} ! W width in GeV\\
   Zmass\phantom{aa}  91.1876 \phantom{aaaaaa}   ! Z mass in GeV\\
   Zwidth\phantom{a} 2.4952 \phantom{aaaaaaa} ! Z width in GeV\\
   alphaem 0.00729735254 \phantom{a}! em coupling alpha(0)\\
   gmu\phantom{aaaa} 1.16637d-5 \phantom{aaa} ! Fermi constant in GeV\^{}-2\\
   Hmass\phantom{aa} 120. \phantom{aaaaaaaa} ! Higgs mass in GeV\\
   Tmass\phantom{aa} 172.9 \phantom{aaaaaaaa} ! Top mass in GeV\\
   Bmass\phantom{aaaa} 4.6 \phantom{aaaaaaaa} ! B quark mass in GeV\\
   Cmass\phantom{aaaa} 1.2 \phantom{aaaaaaaa} ! C quark mass in GeV\\
   Smass\phantom{aaa} 0.15 \phantom{aaaaaaaa} ! S quark mass in GeV\\
   Umass\phantom{aaa} 0.06983 \phantom{aaaaa} ! U quark mass in GeV\\
   Dmass\phantom{aaa} 0.06984 \phantom{aaaaa} ! D quark mass in GeV\\
   Elmass\phantom{aa} 0.005109989 \phantom{a} ! Electron mass in GeV\\
   Mumass\phantom{aa} 0.105658369 \phantom{a} ! Mu mass in GeV\\
   Taumass\phantom{a} 1.77699 \phantom{aaaaa} ! Tau mass in GeV\\
}
\\
The following parameter limits from below the virtuality of the $Z$ boson:
\\
\tmtexttt{mass\_low 20 \phantom{aaaaaaaa}! Z virtuality > mass\_low in GeV}\\
\\
If absent, it is set to 30~GeV.
In order to avoid edge effects, the lower limit \tmtexttt{mass\_low} 
should be more inclusive w.r.t. cuts applied at the analysis level.
Notice that, if photons are generated, the $Z$ virtuality is not
necessarily the mass of the dilepton.
\\
\\
\tmtexttt{runningscale 0 \phantom{aaaa}! choice for ren and fac scales in Bbar integration}\\
\phantom{pppppppppppppppppppppppppppppp} \tmtexttt{ 0: fixed scale M\_Z}\\
\phantom{pppppppppppppppppppppppppppppp} 
\tmtexttt{ 1: running scale $\ell^+ \ell^-$ inv mass }\\ 
With running scale, a minimum cutoff of 5 GeV is imposed on $m(\ell^+ \ell^-)$.
\\
\\
\tmtexttt{scheme 1! choice for EW NLO scheme calculation}\\
\phantom{pppppppppppppppppppppppppppppp} \tmtexttt{ 0: Alpha(0)}\\
\phantom{pppppppppppppppppppppppppppppp} \tmtexttt{ 1: Alpha(M\_Z)}\\
\phantom{pppppppppppppppppppppppppppppp} \tmtexttt{ 2: G\_$\mu$}\\
\\ 
The CKM mixing matrix is assumed diagonal in the EW NLO corrections.\\
\\
The EW radiative corrections can be calculated according to three 
different schemes: the $\alpha(0)$ scheme, where the input parameters 
are $\alpha(0)$, $M_W$ and $M_Z$; the $\alpha(M_Z^2)$ scheme, where the 
input parameters are $\alpha(M_Z^2)$, $M_W$ and $M_Z$ (with this 
scheme the value of the parameter \tmtexttt{alphaem\_z} should be specified); 
the $G_\mu$ scheme, where the 
input parameters are $G_\mu$, $M_W$ and $M_Z$. 
\\
The EW corrections can be switched off by setting
\\
\tmtexttt{no\_ew 1       ! default 0}
\\
and the strong corrections can be switched off by setting
\\
\tmtexttt{no\_strong 1       ! default 0}
\\
This last option is just to check EW corrections at the NLO level (i.e.,
the Les Houches events do not have much meaning).
\\
\\
The program can be interfaced to both \PYTHIASIX{} and  \PYTHIAEIGHT{},
by doing
\\
\tmtexttt{make main-PYTHIA-lhef}
\\
for \PYTHIASIX{} and
\\
\tmtexttt{make main-PYTHIA8-lhef}.
\\
for \PYTHIAEIGHT{}.
\\
In the case of \PYTHIASIX{} one can also optionally switch off photon radiation
from  \PYTHIASIX{} and use \PHOTOS{}~\cite{Golonka:2005pn}
instead. This is done by setting:
\\
\tmtexttt{use\_photos 1   ! default 0}
\\
in the \tmtexttt{powheg.input} file.
\\
The \PHOTOS \, source code is included in the \POWHEGBOX. 
\\
\\
For photon final state radiation a comment is in order. According to 
the \POWHEG{} method, the radiation by the shower has to be generated 
from a starting scale given by the hardest $\pt$ tried at the matrix element 
level (the variable {\tt scalup} written in the event file 
\tmtexttt{pwgevents.lhe}). This is true also in the case that both 
QCD and QED radiation are present, as detailed in 
Ref.~\cite{Barze:2013} . Both \PYTHIASIX{} and \PYTHIAEIGHT{} 
do not use this starting scale for the generation of QED final state radiation 
from the $Z$. Hence, in order to avoid double counting of QED radiation, 
a veto algorithm is necessary. The same problem is present also using 
\PHOTOS{}. 
This algorithm 
is provided automatically in the files \tmtexttt{main-PYTHIA-lhef.f} and 
\tmtexttt{scalupveto.f}. The same algorithm is implemented for 
\PYTHIAEIGHT{}. In this case, however, the user can optionally 
adopt the internal algorithm of \PYTHIAEIGHT, which is switched 
on by setting:
\\
\tmtexttt{py8veto 1   ! default 0}
\\
in the \tmtexttt{powheg.input} file.



\begin{thebibliography}{10}

\bibitem{Nason:2004rx}
  P.~Nason,
  ``A new method for combining NLO QCD with shower Monte Carlo algorithms,''
  JHEP {\bf 0411} (2004) 040
  [arXiv:hep-ph/0409146].
  %%CITATION = JHEPA,0411,040;%%

%\cite{Frixione:2007vw}
\bibitem{Frixione:2007vw}
  S.~Frixione, P.~Nason and C.~Oleari,
``Matching NLO QCD computations with Parton Shower simulations: the POWHEG
method,''
  JHEP {\bf 0711} (2007) 070
  [arXiv:0709.2092 [hep-ph]].
  %%CITATION = JHEPA,0711,070;%%

%\cite{Alioli:2010xd}
\bibitem{Alioli:2010xd}
  S.~Alioli, P.~Nason, C.~Oleari and E.~Re,
``A general framework for implementing NLO calculations in shower Monte Carlo
  programs: the POWHEG BOX,''
  [arXiv:1002.2581 [hep-ph]].
  %%CITATION = ARXIV:1002.2581;%%

\bibitem{Barze:2013}
  L.~Barz\`e, G.~Montagna, P.~Nason, O.~Nicrosini, F.~Piccinini and 
  A.~Vicini, 
  ``paper in preparation''.

%\cite{Golonka:2005pn}
\bibitem{Golonka:2005pn}
  P.~Golonka and Z.~Was,
  %``PHOTOS Monte Carlo: A Precision tool for QED corrections in $Z$ and $W$ decays,''
  Eur.\ Phys.\ J.\ C {\bf 45} (2006) 97
  [hep-ph/0506026].
  %%CITATION = HEP-PH/0506026;%%

\end{thebibliography}

\end{document}





