\documentclass[paper]{JHEP3}
\usepackage{amsmath,amssymb,enumerate,url}
\bibliographystyle{JHEP}

%%%%%%%%%% Start TeXmacs macros
\newcommand{\tmtextit}[1]{{\itshape{#1}}}
\newcommand{\tmtexttt}[1]{{\ttfamily{#1}}}
\newenvironment{enumeratenumeric}{\begin{enumerate}[1.] }{\end{enumerate}}
\newcommand\sss{\mathchoice%
{\displaystyle}%
{\scriptstyle}%
{\scriptscriptstyle}%
{\scriptscriptstyle}%
}

\newcommand\as{\alpha_{\sss\rm S}}
\newcommand\POWHEG{{\tt POWHEG}}
\newcommand\POWHEGBOX{{\tt POWHEG BOX}}
\newcommand\HERWIG{{\tt HERWIG}}
\newcommand\PYTHIA{{\tt PYTHIA}}
\newcommand\MCatNLO{{\tt MC@NLO}}

\newcommand\kt{k_{\sss\rm T}}

\newcommand\pt{p_{\sss\rm T}}
\newcommand\LambdaQCD{\Lambda_{\scriptscriptstyle QCD}}
%%%%%%%%%% End TeXmacs macros


\title{The POWHEG BOX user manual:\\
  $\boldsymbol{H Z}$ production in the SM-EFT} \vfill
\author{Simone Alioli,\\
Universita' degli Studi di Milano-Bicocca,\\ 20126, Milano,
  Italy\\
E-mail: \email{simone.alioli@unimib.it}}
\author{Wouter Dekens,\\
Department of Physics, University of California at San Diego,\\ 
La Jolla, CA 92093-0319, USA \\
E-mail: \email{wdekens@ucsd.edu}}
\author{Michael Girard, \\
Ernest Orlando Lawrence Berkeley National Laboratory, University of California, \\
Berkeley, CA 94720, USA\\
E-mail: \email{michael.girard@berkeley.edu}}
\author{Emanuele Mereghetti,\\
Theoretical Division, Los Alamos National Laboratory,\\
Los Alamos, NM 87545, USA \\
E-mail: \email{emereghetti@lanl.gov}}




\vskip -0.5truecm



\abstract{This note documents the use of the package
  \tmtexttt{POWHEG BOX} for $H Z $ production process, in the presence of SM-EFT operators.
 }
\preprint{\today\\ \tmtexttt{POWHEG BOX} 1.0}

\begin{document}


\section{Introduction}

The \tmtexttt{POWHEG BOX} program is a framework for implementing NLO
calculations in Shower Monte Carlo programs according to the
\POWHEG{} method. An explanation of the method and a discussion of
how the code is organized can be found in
refs.~\cite{Nason:2004rx,Frixione:2007vw,Alioli:2010xd}.  The code is
distributed according to the ``MCNET GUIDELINES for Event Generator
Authors and Users'' and can be found at the web page \\
%
\begin{center}
 \url{http://powhegbox.mib.infn.it}.
\end{center}
%
~\\
%

This program is an implementation of the associated $HZ$ and $H \gamma^*$ production cross sections
induced by SM-EFT operators, with $Z/\gamma^*$  decaying leptonically. 
A detailed description of the
implementation can be found in ref. \cite{Alioli:2018ljm}. Please cite
the paper when you use the program. 
The implementation included in the 
\tmtexttt{POWHEG-BOX/HZ\_smeft} subdirectory package 
is based on the original work of Ref. \cite{Luisoni:2013kna}, to which we refer for more details.

In order to run the \tmtexttt{POWHEG BOX} program, we recommend the
reader to start from the \tmtexttt{POWHEG BOX} user manual, which
contains all the information and settings that are common among
all subprocesses. In this note we focus on  the settings and
parameters specific to $H Z/\gamma^*$ implementation.

\section{Generation of events}
Due to the large number of coefficients of dimension-six operators, 
before building the executables, the user should increase the value of the parameter $\texttt{maxnum}$ in the file \\$\texttt{POWHEG-BOX-V2/include/pwhg\_pwin.h}$.
$\texttt{maxnum} = 500$ is sufficient to run the program. 
Build the executables\\
\tmtexttt{\$ cd POWHEG-BOX-V2/HZ$\_$smeft \\
\$ make pwhg\_main  \\
\$ make main-PYTHIA8-lhef  \\
}\\
Then do (for example) \\
\tmtexttt{\$
cd testrun-lhc-smeft\\
\$ ../pwhg\_main}\\
At
the end of the run, the file \tmtexttt{pwgevents.lhe} will contain
100000 events for the process $H Z/\gamma^* \to H (e^+ e^-)$  in the Les Houches format. In
order to shower them with \PYTHIA{} do\\
\tmtexttt{\$ cd
testrun-lhc-smeft \\
\$ ../main-PYTHIA8-lhef}

\section{Process specific input parameters}

The process inherits all the parameters of the Standard Model $HZ$ production POWHEG process,
described in Ref. \cite{Luisoni:2013kna}.


\begin{table}
\center
\begin{small}
\begin{tabular}{||c|cc||c|cc||}
\hline
								  &  \multicolumn{2}{|c||}{Operator} 	 					&
								  &  \multicolumn{2}{c||}{Operator} 												 \\
\hline
$C_{\varphi W,\, \varphi B, \, \varphi W B}$, 			  & \multicolumn{2}{c||}{\texttt{CC\_ww}, \texttt{CC\_bb}, \texttt{CC\_wb} }      &   
$C_{\varphi \tilde{W},\varphi \tilde{B},\varphi \tilde{W} B}$  	  & \multicolumn{2}{c||}{ \texttt{CC\_wwt}, \texttt{CC\_bbt}, \texttt{CC\_wbt}   }     \\ 								  
\hline
$\Gamma^u_W$ 		& \texttt{ReGUw}\_\textit{ik} & \texttt{ImGUw}\_\textit{ik}  				  		&
$\Gamma^d_W$ 		& \texttt{ReGDw}\_\textit{jl} & \texttt{ImGDw}\_\textit{jl} 				 		\\%
 $\Gamma^u_\gamma$   	& \texttt{ReGUe}\_\textit{ik} & \texttt{ImGUe}\_\textit{ik} 				  		& 
 $\Gamma^d_\gamma$   	& \texttt{ReGDe}\_\textit{jl}	& \texttt{ImGDe}\_\textit{jl}   		\\
\hline
$c^{}_{Q\varphi,\, U}$ 	& \texttt{QphiU}\_\textit{ik}				&   $i \le k$ 	& 
$c^{}_{Q\varphi,\, D}$ 	& \texttt{QphiD}\_\textit{jl} 				&   $j \le l$  	\\
$c_{U\varphi}$ 		& \texttt{Uphi}\_\textit{ik}	 			&   $i \le k$ 	& 
$c_{D\varphi}$ 		& \texttt{Dphi}\_\textit{jl}	    			&   $j \le l$  	\\
\hline
$Y^\prime_u$ 		& \texttt{ReYu}\_\textit{ik}  				&    \texttt{ImYu}\_\textit{ik} 	&
$Y^\prime_d$ 		& \texttt{ReYd}\_\textit{jl}				&    \texttt{ImYd}\_\textit{jl}	\\ 
\hline
\end{tabular}
\end{small}
\caption{Notation for the coefficients of the dimension-six operators that can be set in \POWHEG{}.
The indices $i, k$ run over $u$-type quark flavors $\textit{i},\textit{k} \in \{\texttt{u,c,t}\}$, while $j, l$ run over $d$-type quark flavors $\textit{j},\textit{l} \in \{\texttt{d,s,b}\}$.
With the assumptions of Ref. \cite{Alioli:2018ljm}, $c^{}_{Q\varphi,\, U}$, $c^{}_{Q\varphi,\, D}$, $c^{}_{U\varphi}$, $c^{}_{D\varphi}$
are real symmetric matrices. The notation $i \le k$ and $j \le l$ indicates that the elements $\{\texttt{uu,uc,ut,cc,ct,tt}\}$ 
and $\{\texttt{dd,ds,db,ss,sb,bb}\}$ can be set by the user, while the remaining elements are not independent. }\label{TabA}
\end{table}


By default, all dimension-six corrections are switched off in the \POWHEG{} input card.
To investigate the effect of one or more dimension-six operators, the user  needs to set the flag \texttt{dim6} to 1 and specify the values of 
one or more coefficients of dimension-six operators in the input file. The definition of the SM-EFT operators are given in Ref.\ \cite{Alioli:2018ljm},
and their coefficients are defined to be dimensionless, and scaling as $(v/\Lambda)^2$, where $\Lambda$ is the scale of new physics.

The user can choose between two different assumptions for the flavor structures of the coefficients, as discussed in detail in Ref. \cite{Alioli:2018ljm}.

The first possibility is a generic flavor structure. The notation for the coefficients of SM-EFT operators in this scenario is given in Table \ref{TabA},
and the only restriction we impose is that the matrices $c^{}_{Q\varphi,\, U}$, $c^{}_{Q\varphi,\, D}$, $c^{}_{U\varphi}$, $c^{}_{D\varphi}$
are real symmetric matrices, instead of hermitian matrices (i.e.\ the phases of the off-diagonal 
entries are set to zero). In particular, the user can set flavor off-diagonal coefficients, which induce tree-level processes that are absent in the SM, such as $d \bar s \rightarrow h e^+ e^-$ 
or $c \bar u \rightarrow h e^+ e^-$. To generate these processes, the flag \texttt{fcnc} should be set to 1 in the \POWHEG{} input card.   
An example for the \texttt{powheg.input}  card in this scenario is given in the folder \texttt{testrun-lhc-smeft}.


As an alternative to considering generic flavor structures, the code also provides the option to assume the Minimal Flavor Violation (MFV) framework. In this scenario, the allowed forms of the couplings are constrained by  flavor symmetries, which significantly decreases the number of free parameters. 
The implications for the flavor structures of the couplings are discussed in Appendix B of Ref. \cite{Alioli:2018ljm}.
To choose this scenario, the user should set the flag $\texttt{mfv}$ to 1, and select some of the couplings in Table \ref{tab:mfv}.
An example for the \texttt{powheg.input}  card in this scenario is given in the folder \texttt{testrun-lhc-mfv}. The couplings of the Higgs-gauge operators
$C_{\varphi W,\, \varphi B, \, \varphi W B}$ and $C_{\varphi \tilde{W},\varphi \tilde{B},\varphi \tilde{W} B}$ are not affected by the assumption of MFV, and their notation is as in Table
\ref{TabA}.


\begin{table}
\center
\begin{tabular}{||c|c  c|| c | c c || }
\hline
	Operator    & $a_i$ & $b_i$ & Operator & $a_i$ & $b_i$\\\hline
$	c_{Q\varphi,U}$ & \texttt{A}\_\texttt{QphiU} & - &
$	c_{Q\varphi,D}$ & \texttt{A}\_\texttt{QphiD} & \texttt{B}\_\texttt{QphiD}  \\
$	c_{U\varphi }$  & \texttt{A}\_\texttt{Uphi} &-&
$	c_{D\varphi }$  & \texttt{A}\_\texttt{Dphi}&- \\
\hline
\end{tabular}
\caption{Notation for the coefficients of the different flavor structures in the MFV framework, discussed in Appendix B of Ref. \cite{Alioli:2018ljm}, which can be set in \texttt{POWHEG}.
The couplings $C_{\varphi W,\, \varphi B, \, \varphi W B}$ and $C_{\varphi \tilde{W},\varphi \tilde{B},\varphi \tilde{W} B}$ are not affected by the MFV assumption, and the notation is as in Table 
\ref{TabA}. Dipole and Yukawa couplings are set to zero.
}\label{tab:mfv}
\end{table}

In addition to the couplings of dimension-six operators, the user can input the elements of the CKM matrix by setting  $\texttt{CKM\_V}ij$, with $i \in \{\texttt{u,c,t}\}$ and $j \in \{ \texttt{d,s,b}\}$,
in the input file. By default the CKM matrix is assumed to be real, and terms of $\mathcal O(\lambda^2)$, where $\lambda \sim |V_{us}|$, and higher are neglected. 

The dipole, scalar and tensor operators in Table \ref{TabA} run in QCD, and therefore the user should specify the scale at which the coefficents are evaluated.
The new physics scale $\Lambda$, at which the coefficients are defined, can be specified 
by setting the flag $\texttt{LambdaNP}$ to the desired value. 
The coefficients are then run from $\Lambda$ to $\mu_R$, the renormalization scale of the  process. By default, $\texttt{LambdaNP} = 1$ TeV.
For the coefficients that do not evolve under QCD, the flag $\texttt{LambdaNP}$ is irrelevant.



\begin{thebibliography}{10}

\bibitem{Nason:2004rx}
  P.~Nason,
  ``A new method for combining NLO QCD with shower Monte Carlo algorithms,''
  JHEP {\bf 0411} (2004) 040
  [arXiv:hep-ph/0409146].
  %%CITATION = JHEPA,0411,040;%%

%\cite{Frixione:2007vw}
\bibitem{Frixione:2007vw}
  S.~Frixione, P.~Nason and C.~Oleari,
``Matching NLO QCD computations with Parton Shower simulations: the POWHEG
method,''
  JHEP {\bf 0711} (2007) 070
  [arXiv:0709.2092 [hep-ph]].
  %%CITATION = JHEPA,0711,070;%%

%\cite{Alioli:2010xd}
\bibitem{Alioli:2010xd}
  S.~Alioli, P.~Nason, C.~Oleari and E.~Re,
``A general framework for implementing NLO calculations in shower Monte Carlo
  programs: the POWHEG BOX,''
  [arXiv:1002.2581 [hep-ph]].
  %%CITATION = ARXIV:1002.2581;%%

%\cite{Alioli:2018ljm}
\bibitem{Alioli:2018ljm} 
  S.~Alioli, W.~Dekens, M.~Girard and E.~Mereghetti,
  %``NLO QCD corrections to SM-EFT dilepton and electroweak Higgs boson production, matched to parton shower in POWHEG,''
  JHEP {\bf 1808}, 205 (2018)
  doi:10.1007/JHEP08(2018)205
  [arXiv:1804.07407 [hep-ph]].
  %%CITATION = doi:10.1007/JHEP08(2018)205;%%
  %3 citations counted in INSPIRE as of 05 Oct 2018
  
  
%\cite{Luisoni:2013kna}
\bibitem{Luisoni:2013kna} 
  G.~Luisoni, P.~Nason, C.~Oleari and F.~Tramontano,
  %``$HW^{\pm}$/HZ + 0 and 1 jet at NLO with the POWHEG BOX interfaced to GoSam and their merging within MiNLO,''
  JHEP {\bf 1310}, 083 (2013)
  doi:10.1007/JHEP10(2013)083
  [arXiv:1306.2542 [hep-ph]].
  %%CITATION = doi:10.1007/JHEP10(2013)083;%%
  %113 citations counted in INSPIRE as of 10 Oct 2018


  
  

\end{thebibliography}

\end{document}





