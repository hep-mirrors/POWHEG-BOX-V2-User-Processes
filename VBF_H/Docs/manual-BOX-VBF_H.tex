\documentclass[paper]{JHEP3}
\usepackage{amsmath,amssymb,enumerate,url}
\bibliographystyle{JHEP}

%%%%%%%%%% Start TeXmacs macros
\newcommand{\tmtextit}[1]{{\itshape{#1}}}
\newcommand{\tmtexttt}[1]{{\ttfamily{#1}}}
\newenvironment{enumeratenumeric}{\begin{enumerate}[1.] }{\end{enumerate}}
\newcommand\sss{\mathchoice%
{\displaystyle}%
{\scriptstyle}%
{\scriptscriptstyle}%
{\scriptscriptstyle}%
}

\newcommand\as{\alpha_{\sss\rm S}}
\newcommand\POWHEG{{\tt POWHEG}}
\newcommand\POWHEGBOX{{\tt POWHEG BOX}}
\newcommand\HERWIG{{\tt HERWIG}}
\newcommand\PYTHIA{{\tt PYTHIA}}
\newcommand\MCatNLO{{\tt MC@NLO}}

\newcommand\kt{k_{\sss\rm T}}

\newcommand\pt{p_{\sss\rm T}}
\newcommand\LambdaQCD{\Lambda_{\scriptscriptstyle QCD}}
%%%%%%%%%% End TeXmacs macros


\title{The POWHEG BOX user manual:\\
  Higgs boson production in vector-boson fusion} \vfill
\author{Paolo Nason\\
  INFN, Sezione di Milano-Bicocca,
  Piazza della Scienza 3, 20126 Milan, Italy\\
  E-mail: \email{Paolo.Nason@mib.infn.it}}
\author{Carlo Oleari\\
  Universit\`a di Milano-Bicocca and INFN, Sezione di Milano-Bicocca\\
  Piazza della Scienza 3, 20126 Milan, Italy\\
  E-mail: \email{Carlo.Oleari@mib.infn.it}}
\vskip -0.5truecm

\keywords{POWHEG, Shower Monte Carlo, NLO}

\abstract{This note documents the use of the package \POWHEGBOX{}
  for Higgs boson production in vector-boson fusion processes.  Results can
  be easily interfaced to shower Monte Carlo programs, in such a way that
  both NLO and shower accuracy are maintained.}  \preprint{\today\\
  \tmtexttt{POWHEG BOX} 1.0}

\begin{document}


\section{Introduction}

The \POWHEGBOX{} program is a framework for implementing NLO
calculations in Shower Monte Carlo programs according to the
\POWHEG{} method. An explanation of the method and a discussion of
how the code is organized can be found in
refs.~\cite{Nason:2004rx,Frixione:2007vw,Alioli:2010xd}.  The code is
distributed according to the ``MCNET GUIDELINES for Event Generator
Authors and Users'' and can be found at the web page 
%
\begin{center}
 \url{http://powhegbox.mib.infn.it}.
\end{center}
%
In this manual, we describe the \POWHEG{} NLO implementation of Higgs boson
production in vector-boson fusion processes, as described in
ref.~\cite{Nason:2009ai}


\section{Generation of events}

Build the executable\\
\tmtexttt{\$ cd POWHEG-BOX/VBF\_H \\
\$ make pwhg\_main }\\

\noindent
Then do (for example) \\
\tmtexttt{\$
cd testrun-lhc\\
\$ ../pwhg\_main}\\

\noindent
At the end of the run, the file \tmtexttt{pwgevents.lhe} will contain 500000
events for Higgs boson production in vector-boson fusion processes in the Les
Houches format.  

\noindent
In order to shower them with \PYTHIA{} do\\
\tmtexttt{\$
cd POWHEG-BOX/VBF\_H \\ 
\$ make main-PYTHIA-lhef \\ 
\$ cd testrun-lhc \\
\$ ../main-PYTHIA-lhef}

\section{Process specific input parameters}

In order to improve the efficiency for the generation of events,
the optional parameter  \tmtexttt{withdamp} should be set to 1, i.e.~there
should be the following line in the input file \tmtexttt{powheg.input}

\tmtexttt{withdamp    1      ! (default 0, do not use) use Born-zero damping factor}
\\

\noindent
The dafault invariant Higgs boson mass is distributed according to a
Breit-Wigner~(BW) with a running width. In case one is interested in the
generation of a Higgs boson invariant mass with a fixed width, the following
line should be present in the \tmtexttt{powheg.input} file

\tmtexttt{
whiggsfixedwidth    1  ! if 1, use old behaviour with fixed width Breit-Wigner
\mbox{}\hspace{44mm} ! default is running width
}

\noindent
The complex-pole scheme according to Passarino et al. is activated by the
flag

\tmtexttt{ complexpolescheme    1   ! complex-pole scheme according to
  Passarino et al.}\\ \\



\noindent
In case the \POWHEGBOX{} is interfaced to \PYTHIA{} or \HERWIG{}, it is
possible to select the Higgs boson decay products by setting the optional
parameter \tmtexttt{hdecaymode} to one of the allowed values

~\\
\tmtexttt{
  hdecaymode -1 \phantom{aa} ! -1 no decay\\
  \phantom{aaaaaaaaaaaaaaaaa}!  0 all decay channels open\\
\phantom{aaaaaaaaaaaaaaaaa}!	               1-6 d dbar, u ubar,..., t tbar\\
\phantom{aaaaaaaaaaaaaaaaa}!	               7-9 e+ e-, mu+ mu-, tau+ tau-\\
\phantom{aaaaaaaaaaaaaaaaa}!		       10  W+W-\\
\phantom{aaaaaaaaaaaaaaaaa}!	               11  ZZ\\
\phantom{aaaaaaaaaaaaaaaaa}!	               12  gamma gamma\\     
}
\\



\begin{thebibliography}{10}

\bibitem{Nason:2004rx}
  P.~Nason,
  ``A new method for combining NLO QCD with shower Monte Carlo algorithms,''
  JHEP {\bf 0411} (2004) 040
  [arXiv:hep-ph/0409146].
  %%CITATION = JHEPA,0411,040;%%

%\cite{Frixione:2007vw}
\bibitem{Frixione:2007vw}
  S.~Frixione, P.~Nason and C.~Oleari,
``Matching NLO QCD computations with Parton Shower simulations: the POWHEG
method,''
  JHEP {\bf 0711} (2007) 070
  [arXiv:0709.2092 [hep-ph]].
  %%CITATION = JHEPA,0711,070;%%

%\cite{Alioli:2010xd}
\bibitem{Alioli:2010xd}
  S.~Alioli, P.~Nason, C.~Oleari and E.~Re,
``A general framework for implementing NLO calculations in shower Monte Carlo
  programs: the POWHEG BOX,''
  [arXiv:1002.2581 [hep-ph]].
  %%CITATION = ARXIV:1002.2581;%%

%\cite{Nason:2009ai}
\bibitem{Nason:2009ai}
  P.~Nason and C.~Oleari,
  ``NLO Higgs boson production via vector-boson fusion matched with shower in
  POWHEG,''
  JHEP {\bf 1002} (2010) 037
  [arXiv:0911.5299 [hep-ph]].
  %%CITATION = JHEPA,1002,037;%%



\end{thebibliography}

\end{document}





