\documentclass[a4paper,11pt]{article}
\usepackage{amssymb,enumerate}
\usepackage{amsmath}
\usepackage{bbm}
\usepackage{url}
\usepackage{cite}
\usepackage{graphics}
\usepackage{xspace}
\usepackage{epsfig}
\usepackage{subfigure}

\setlength\paperwidth  {210mm}%
\setlength\paperheight {300mm}%	

\textwidth 160mm%		% DEFAULT FOR LATEX209 IS a4
\textheight 235mm%

\voffset -1in
\topmargin   .05\paperheight	% FROM TOP OF PAGE TO TOP OF HEADING (0=1inch)
\headheight  .02\paperheight	% HEIGHT OF HEADING BOX.
\headsep     .03\paperheight	% VERT. SPACE BETWEEN HEAD AND TEXT.
\footskip    .07\paperheight	% FROM END OF TEX TO BASE OF FOOTER. (40pt)


\hoffset -1in				% TO ADJUST WITH PAPER:
	\oddsidemargin .13\paperwidth	% LEFT MARGIN FOR ODD PAGES (10)
	\evensidemargin .15\paperwidth	% LEFT MARGIN FOR EVEN PAGES (30)
	\marginparwidth .10\paperwidth	% TEXTWIDTH OF MARGINALNOTES
	\reversemarginpar		% BECAUSE OF TITLEPAGE.

%%%%%%%%%% Start TeXmacs macros
\newcommand{\tmtextit}[1]{{\itshape{#1}}}
\newcommand{\tmtexttt}[1]{{\ttfamily{#1}}}
\newenvironment{enumeratenumeric}{\begin{enumerate}[ 1.] }{\end{enumerate}}
\newcommand\sss{\mathchoice%
{\displaystyle}%
{\scriptstyle}%
{\scriptscriptstyle}%
{\scriptscriptstyle}%
}
\newcommand\PSn{\Phi_{n}}
\newcommand{\tmop}[1]{\ensuremath{\operatorname{#1}}}



\newcommand\Lum{{\cal L}}
\newcommand\matR{{\cal R}}
\newcommand\Kinnpo{{\bf \Phi}_{n+1}}
\newcommand\Kinn{{\bf \Phi}_n}
\newcommand\PSnpo{\Phi_{n+1}}
\newcommand\as{\alpha_{\sss\rm S}}
\newcommand\asotpi{\frac{\as}{2\pi}}

\newcommand\POWHEG{{\tt POWHEG}}
\newcommand\POWHEGBOX{{\tt POWHEG BOX}}
\newcommand\POWHEGBOXV{{\tt POWHEG-BOX-V2}}
\newcommand\PYTHIA{{\tt PYTHIA}}
\newcommand\POWHEGpPYTHIA{{\tt POWHEG+PYTHIA}}
\newcommand\HERWIG{{\tt HERWIG}}

\def\lq{\left[} 
\def\rq{\right]} 
\def\rg{\right\}} 
\def\lg{\left\{} 
\def\({\left(} 
\def\){\right)} 

\def\beq{\begin{equation}}
\def\beqn{\begin{eqnarray}}
\def\eeq{\end{equation}}
\def\eeqn{\end{eqnarray}}

\def\mr{\mathrm}
\def\vbfevmv{VBF $e^+\nu_e\mu^-\bar\nu_\mu jj$\;}
\def\vbfww{VBF $W^+W^-jj$\;}
\def\wpm{W^+W^-}
\def\evmv{e^+\nu_e\mu^+\nu_\mu}
\def\pbox{{\tt POWHEG BOX}}
\def\pwg{{\tt POWHEG}}
\def\mc{\mathcal}

%%%%%%%%%% End TeXmacs macros

\title{Manual for  Higgs boson production via vector-boson fusion in association with three jets in the \POWHEGBOXV{}}
\date{}

\begin{document}
\maketitle
%
\noindent
The {\tt VBF\_HJJJ} program in the \POWHEGBOXV{} is an implementation of Higgs boson production via vector-boson fusion in association with three jets. Details of the calculation are described in Ref.~\cite{JSZ}. If you use this program, please quote Refs.~\cite{JSZ,Figy:2007kv,Alioli:2010xd}.
%\\[2ex]
%
\section*{Running the program}
%
Download the \POWHEGBOXV{}, following the instructions at the web site 
\\[2ex]
{\tt http://powhegbox.mib.infn.it/}
\\[2ex] 
and go to the process directory by typing 
\\[2ex]
{\tt \$ cd \POWHEGBOXV/VBF\_HJJJ}  
\\[2ex]
Running is most conveniently done in a separate directory. Together with the code, we provide the directory {\tt testrun} that contains sample input and seed files. 
\\[2ex]
For your runs, generate your own directory, for instance by doing 
\\[2ex]
{\tt \$ mkdir myruns}
\\[2ex]
The directory must contain the {\tt powheg.input} file, a file {\tt vbfnlo.dat} where boson masses and electroweak input parameters are set, and, for
parallel running, a {\tt pwgseeds.dat} file (see {\tt manual-BOX.pdf}
and {\tt Manyseeds.pdf} in the  {\tt POWHEG-BOX-V2/Docs} directory).
\\[2ex]
Before compiling make sure that:
\begin{itemize}
\item 
{\tt fastjet} is installed and {\tt fastjet-config} is in the path,
\item 
{\tt lhapdf} is installed and {\tt lhapdf-config} is in the path,
\item
{\tt gfortran}, {\tt ifort} or {\tt g77} is in the path, and the
appropriate libraries are in the environment variable {\tt
  LD\_LIBRARY\_PATH}. 
\end{itemize}
%\\[2ex]
After compiling the executable {\tt pwhg\_main} in the {\tt VBF\_HJJJ} directory, enter the {\tt myruns} directory and perform all your runs there. 
\\[2ex]
We recommend running the program in a parallel mode in several consecutive steps by setting 
\\[2ex]
{\tt manyseeds   1}
\\[2ex]
in the file  {\tt powheg.input}. The four steps of grid generation, NLO calculation, upper bound generation, and event generation can then be performed in parallel, consecutively, as described, for instance, in the manual of the {\tt VBF\_Z\_Z} directory in the \POWHEGBOXV{}. 

If the default analysis is activated by setting the flag 
{\tt ANALYSIS=default} in the Makefile before compiling the code, after the completion of the NLO calculation for each parallel run a file {\tt  pwg-*-NLO.top} is generated (where the * denotes the integer
identifier of the run). These files contain histogram information at fixed-order accuracy for a representative experimental setup in gnuplot-friendly format. The default analysis routine can easily be replaced with a personalized one by the user.  

The events that are ultimately generated in Les Houches format can be processed by a generic Monte-Carlo program. We are providing an interface to \PYTHIA~{\tt 6.4.25} that can be used by generating the executable {\tt main-PYTHIA-lhef} and running it in the directory where the event files are stored. The program then generates an output file {\tt pwgPOWHEG+PYTHIA-output.top} that contains histograms at NLO+PS accuracy. The Monte-Carlo parameters can be modified by the user in the file {\tt setup-PYTHIA-lhef.f}. 

%%%%%%%%%%%%%%%%%%%%%%%%%%
%
\begin{thebibliography}{99}

\bibitem{JSZ} B.~J\"ager, F.~Schissler, D.~Zeppenfeld {\em Parton-shower effects on Higgs boson production via vector-boson fusion in association with three jets}, arXiv:1405.6950 [hep-ph]. 

\bibitem{Figy:2007kv}
  T.~Figy, V.~Hankele and D.~Zeppenfeld,
  {\em Next-to-leading order QCD corrections to Higgs plus three jet production in vector-boson fusion},
  JHEP {\bf 0802} (2008) 076
  [arXiv:0710.5621 [hep-ph]].

\bibitem{Alioli:2010xd} S.~Alioli, P.~Nason, C.~Oleari and E. Re, {\em
    A general framework for implementing NLO calculations in shower
    Monte Carlo programs: the POWHEG BOX}, JHEP {\bf 1006} (2010)
  043  [arXiv:1002.2581 [hep-ph]].

\end{thebibliography}
%%%%%%%%%%%%%%%%%%
\end{document}
