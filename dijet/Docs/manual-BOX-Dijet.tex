\documentclass[paper]{JHEP3}
\usepackage{amsmath,amssymb,enumerate,url}
\bibliographystyle{JHEP}

%%%%%%%%%% Start TeXmacs macros
\newcommand{\tmtextit}[1]{{\itshape{#1}}}
\newcommand{\tmtexttt}[1]{{\ttfamily{#1}}}
%\newcommand{\texttt}[1]{\tt #1}
\newenvironment{enumeratenumeric}{\begin{enumerate}[1.] }{\end{enumerate}}
\newcommand\sss{\mathchoice%
{\displaystyle}%
{\scriptstyle}%
{\scriptscriptstyle}%
{\scriptscriptstyle}%
}

\newcommand\as{\alpha_{\sss\rm S}}
\newcommand\POWHEG{{\tt POWHEG}}
\newcommand\POWHEGBOX{{\tt POWHEG BOX}}
\newcommand\HERWIG{{\tt HERWIG}}
\newcommand\PYTHIA{{\tt PYTHIA}}
\newcommand\MCatNLO{{\tt MC@NLO}}

\newcommand\kt{k_{\sss\rm T}}

\newcommand\pt{p_{\sss\rm T}}
\newcommand\LambdaQCD{\Lambda_{\scriptscriptstyle QCD}}
%%%%%%%%%% End TeXmacs macros


\title{The POWHEG BOX user manual:\\
  dijet production} \vfill
\author{Simone Alioli\\
  Deutsches Elektronen-Synchrotron DESY\\
  Platanenallee 6, D-15738 Zeuthen, Germany\\
  E-mail: \email{simone.alioli@desy.de}}
\author{Keith Hamilton\\
  INFN, Sezione di Milano-Bicocca,
  Piazza della Scienza 3, 20126 Milan, Italy\\
  E-mail: \email{Keith.Hamilton@mib.infn.it}}
\author{Paolo Nason\\
  INFN, Sezione di Milano-Bicocca,
  Piazza della Scienza 3, 20126 Milan, Italy\\
  E-mail: \email{Paolo.Nason@mib.infn.it}}
\author{Carlo Oleari\\
  Universit\`a di Milano-Bicocca and INFN, Sezione di Milano-Bicocca\\
  Piazza della Scienza 3, 20126 Milan, Italy\\
  E-mail: \email{Carlo.Oleari@mib.infn.it}}
\author{Emanuele Re\\
  Institute for Particle Physics Phenomenology, Department of Physics\\
  University of Durham, Durham, DH1 3LE, UK\\
  E-mail: \email{emanuele.re@durham.ac.uk}}

\vskip -0.5truecm

\keywords{POWHEG, Shower Monte Carlo, NLO}

\abstract{This note documents the use of the package
  \POWHEGBOX{} for jet pair production processes.
  Results can be easily interfaced to shower Monte Carlo programs, in
  such a way that both NLO and shower accuracy are maintained.}
\preprint{\today\\ \tmtexttt{POWHEG BOX} 1.0}

\begin{document}


\section{Introduction}

The \POWHEGBOX{} program is a framework for implementing NLO
calculations in Shower Monte Carlo programs according to the
\POWHEG{} method. An explanation of the method and a discussion of
how the code is organized can be found in
refs.~\cite{Nason:2004rx,Frixione:2007vw,Alioli:2010xd}.  The code is
distributed according to the ``MCNET GUIDELINES for Event Generator
Authors and Users'' and can be found at the web page 
%
\begin{center}
 \url{http://powhegbox.mib.infn.it}.
\end{center}
%
In this manual, we describe the \POWHEG{} NLO implementation of jet pair
hadroproduction, as described in ref.~\cite{Alioli:2010xa}.


\section{Producing \POWHEG{} events out-of-the-box}

The \POWHEGBOX{} dijet program may be run directly
from the downloaded code as follows:
\begin{itemize}
\item Enter the\texttt{ POWHEG-BOX/Dijet }directory.
\item Type \texttt{make} here to produce an executable \texttt{pwhg\_main}
for producing \POWHEG{} events.%
\footnote{The compiler will relay numerous warnings due to our use of the \texttt{-Wall}
compilation flag; as with most High Energy Physics software these
can be safely ignored.%
}
\item Enter either directory \texttt{testrun-tev} or \texttt{testrun-lhc}. 
\item Type \texttt{../pwhg\_main }to run the program using the local \texttt{powheg.input}
input file to produce a Les Houches format event file, \texttt{pwgevents.lhe},
containing 50000 \POWHEG{} events.
\end{itemize}
The run-time for producing such an event sample should be in the region
of three hours. The number of events to be generated may be freely
changed and is set by the \texttt{numevts} flag in \texttt{powheg.input}.
Such events are distributed according to the so-called \POWHEG{}
hardest emission cross section, essentially a leading-log resummed
version of the next-to-leading order cross section, thus they contain
three-parton final-states with very few additional, `non-radiative',
two-parton final-state events.

The user is free to analyse the \POWHEG{} predictions to give
NLO accurate, resummed, predictions for observables using their own
analysis code, or, to fully develop them into life-like hadron level
events by passing them to a shower Monte Carlo program. In any case,
we have included within the \POWHEGBOX{} package programs
which can carry out all of these tasks with a minimum of effort, as
described in the following sections. 


\section{NLO and \POWHEG{}-level analysis with \textsc{FastJet}}

Besides producing the \POWHEG{} events the program can also
analyse them on-the-fly whilst they are being generated. At least
for jet production, to carry out such analysis the user should have
installed a copy of the \textsc{FastJet} jet-clustering package%
\footnote{FastJet is available at\texttt{ http://www.lpthe.jussieu.fr/\textasciitilde{}salam/fastjet/}%
} and the directory of the \texttt{fastjet\_config} script included
in his / her \texttt{PATH} environment variable. With \textsc{FastJet}
set up one can produce some basic \POWHEG{}-level topdrawer
histogram output by replacing, at the beginning of the \texttt{Dijet/Makefile}, 

\begin{center}
\texttt{\medskip{}
ANALYSIS=none $\Rightarrow$ ANALYSIS=default\medskip{}
}
\par\end{center}

\noindent which simply swaps\texttt{ pwhg\_analysis-dummy.f} by the
\textsc{FastJet} dependent \texttt{pwhg\_analysis.f} code. Now, repeating
the same steps given in Sect.\,1, yields an additional topdrawer
file, \texttt{pwgpwhgalone-output.top}, containing the analysis.

The \texttt{pwhg\_analysis.f} file contains very basic code passing particle
momenta to \textsc{FastJet} as well as booking and filling histograms. The
pre-packaged analysis was used, in part, to produce a number of the results
in our related publication~\cite{Alioli:2010xa} but can be overwritten by
users as they wish.

As well as producing analysis at the level of the \POWHEG{}
hardest emission cross section, the software is also capable of producing
the corresponding fixed order NLO analysis, since the NLO cross section
must be throughly explored prior to event generation to determine
an upper bound for the $\widetilde{B}\left(\Phi_{B}\right)$ function
--- the distribution of the underlying-Born kinematic configurations.
This analysis proceeds through the exact same \texttt{pwhg\_analysis.f}
code and maybe activated by replacing

\begin{center}
\texttt{\medskip{}
\#testplots 1 $\Rightarrow$ testplots 1\medskip{}
}
\par\end{center}

\noindent in \texttt{powheg.input}. Hopefully it is obvious that enabling
this feature depends on having set \texttt{ANALYSIS = default} in
the \texttt{Makefile}. If one is only interested in obtaining fixed
order NLO results it is not necessary to generate any \POWHEG{}
events, hence, the process is expedited by then setting \texttt{numevts}
to zero.

Of course the Les Houches file content may be analysed `off-line',
independently of the way in which it was generated, using standalone
code. Users can readily write their own programs to this end, however,
for convenience we include a program to do the task using, again,
the \texttt{pwhg\_analysis.f} code. To build this program, \texttt{lhef\_analysis},
\begin{itemize}
\item Enter\texttt{ POWHEG-BOX/Dijet/} .
\item Type \texttt{make} \texttt{lhef\_analysis} .
\end{itemize}
\noindent Two inputs are required in order to run the analysis: the
input file used to create the Les Houches events,\texttt{ foo-powheg.input},
and the event file \texttt{bar.lhe}. If the input file has been misplaced
it can be easily recovered from the head of the Les Houches event
file. Assuming these two files are in the directory \texttt{POWHEG-BOX/Dijet/foo-bar}
\begin{itemize}
\item Enter\texttt{ POWHEG-BOX/Dijet/foo-bar/} .
\item Type \texttt{../lhef\_analysis} .
\end{itemize}
When prompted enter the prefix of the input file \texttt{`foo}' and,
on further prompting, the full name of the event file. Note that the
program assumes that the name of the input file is formatted as \texttt{{*}-powheg.input}.
The program will run to produce a topdrawer histogram file, \texttt{LHEF\_analysis.top}.

If run on the same event file created by \texttt{pwhg\_main}, with
no modifications to \texttt{pwhg\_analysis.f}, one should find \texttt{LHEF\_analysis.top}
and \texttt{pwgpwhgalone-output.top} are basically identical, making
this seem like a redundant feature. However, if one is only interested
in the \POWHEG{} predictions and not the (inferior) fixed order
NLO ones, it is of course hugely faster to modify \texttt{pwhg\_analysis.f}
to your tastes and then run \texttt{lhef\_analysis}, as opposed to
running the event generation code over again from the beginning. Moreover,
if one is ultimately interested in generating large samples of events
it will be necessary to generate them independently on a CPU cluster
whereupon they may be combined at the end and analysed \emph{en} \emph{masse}
with \texttt{lhef\_analysis}.%
\footnote{For more details on parallel generation of event samples see \texttt{POWHEG-BOX/Docs/Manyseeds.pdf}.%
}


\section{Evolving \POWHEG{} events to the hadron level}

Given a Les Houches file of \POWHEG{} events one may evolve
these to the hadron level with \textsc{Herwig} as follows 
\begin{itemize}
\item Enter\texttt{ POWHEG-BOX/Dijet/} .
\item Type \texttt{make main-HERWIG-lhef} .
\end{itemize}
\noindent As in Sect.\,2, to run the program we require the input
file used to create the Les Houches event file,\texttt{ foo-powheg.input},
and the event file itself \texttt{bar.lhe} in the same directory.
Assuming these are present in \texttt{POWHEG-BOX/Dijet/foo-bar}
\begin{itemize}
\item Enter\texttt{ POWHEG-BOX/Dijet/foo-bar/} .
\item Type \texttt{../main-HERWIG-lhef}.
\end{itemize}
When prompted enter the prefix of the input file \texttt{`foo}' and
on further prompting \texttt{`bar.lhe}'. Assuming that \texttt{ANALYSIS
= default} in the \texttt{Makefile} the program will run to produce
a topdrawer histogram file, \texttt{foo-POWHEG+HERWIG-output.top},
based on the analysis of the hadron-level events with the (freely
changeable) \texttt{pwhg\_analysis.f} code. Although we have not included
the facility explicitly it is of course possible to store the showered
events in a binary format for direct off-line analysis and / or detector
simulation with little effort.

The file \texttt{main-HERWIG.f} in the main {\tt POWHEG-BOX}
directory contains a short generic program driving the \textsc{Herwig}
run, while \texttt{setup-HERWIG-lhef.f}, in the \texttt{Dijet} directory,
contains an equally brief user-serviceable subroutine, \texttt{setup\_HERWIG\_parameters},
allowing one to set options affecting the behaviour of \textsc{Herwig}. 

All of the discussion and instructions above here hold true for the
\textsc{Pythia}\texttt{ }event generator, up to replacing \texttt{HERWIG
$\rightarrow$ PYTHIA} everywhere it appears in the shell commands.
As in the analogous case of \textsc{Herwig}, \texttt{setup\_PYTHIA\_parameters}
isolates the setting of \textsc{Pythia} options; here one can, for
example, disable the multiple interactions / underlying event model,
substantially quickening studies in their preliminary stages. One
minor difference with respect to the \textsc{Herwig} driver code is
the routine \texttt{setup\_PYTHIA\_tune} routine in\texttt{ setup-PYTHIA-lhef.f}
file. This routine simply calls \texttt{PYTUNE} with an index to a
particular \textsc{Pythia} tuning, other tunings may be freely selected
by simply changing the argument of \texttt{PYTUNE}.


\section{Running with \textsc{Lhapdf}}

The \POWHEGBOX{} includes, for convenience, its own
internal PDF evolution code and a limited number of PDF data sets
(see the \texttt{POWHEG-BOX/pdfdata}/). It is, however, easy to use
instead the popular \textsc{Lhapdf} package for this purpose. To do
this edit the beginning of the \texttt{Makefile} by simply replacing
the line

\begin{center}
\texttt{PDF=native $\Rightarrow$ PDF=lhapdf}
\par\end{center}

\noindent and in the \texttt{foo-powheg.input} file

\begin{center}
\texttt{ndns1 131 $\Rightarrow$ ! ndns1 131}
\par\end{center}

\begin{center}
\texttt{ndns2 131 $\Rightarrow$ ! ndns2 131}
\par\end{center}

\begin{center}
\texttt{! lhans1 10050 $\Rightarrow$ lhans1 10050}
\par\end{center}

\begin{center}
\texttt{! lhans2 10050 $\Rightarrow$ lhans2 10050}
\par\end{center}

\noindent Of course, for this to work, \textsc{Lhapdf} should be correctly
installed and the directory containing the \texttt{lhapdf-config}
script should be included in the \texttt{PATH} environment variable.
Note also that the \texttt{Makefile} we include assumes the latest
version of \textsc{Lhapdf} is installed, if this is not the case one
may need to adjust the setting of the \texttt{LIBSLHAPDF} variable
in it, replacing the two instances of \texttt{-{}-libdir) }on line
58 by \texttt{-{}-prefix)/lib}. Lastly, with performance in mind we
recommend building \textsc{Lhapdf} with the \texttt{-{}-enable-low-memory}
configure option.%
\footnote{For more details please see \texttt{http://projects.hepforge.org/lhapdf/configure}%
}


\section{\texttt{bornktmin}}

As with all Monte Carlo simulations of processes such as this, which
are already divergent at the leading order, a generation cut
is required in the simulation process to avoid the singular, low $p_{T}$,
region of phase space. The presence of such cuts is nothing new and
will be well familiar to users who have simulated processes of the
type $X+\mathrm{jet}$ using, for example, \textsc{Pythia }and\textsc{
Herwig}. In the \POWHEGBOX{} framework, as in \textsc{Pythia
}and\textsc{ Herwig}, this amounts to a lower bound on the $p_{T}$
of the $2\rightarrow2$ underlying Born configuration, from which
radiation is subsequently generated. The value of this cut is set
by \texttt{bornktmi}n in \texttt{powheg.input}.

Since this cut has no physical origins and is merely a crude device
to avoid the singular regions of the leading order configurations,
it is essential that the observables which one is analysing include
sufficiently high $p_{T}\,/\, E_{T}$ cuts that they are unaffected
by a decrease in it. Re-iterating, it is fundamental that the observables
one is studying exhibit little or no sensitivity to the `missing',
unpopulated, low $p_{T}$ region of phase space below the cut. This
requirement must be more rigidly enforced in the \POWHEGBOX{}
case than \textsc{Pythia }or\textsc{ Herwig} since the latter only
aim at leading order accuracy.

To give some idea of what may be considered appropriate, in our related
article,\texttt{ arXiv:1012.3380v1}, we were able to reliably use
a generation cut, \texttt{bornktmi}n, of 10 GeV to study observables
for which the two leading jets were above as little as 20 GeV in transverse
momentum (at the Tevatron), while a \texttt{bornktmi}n cut of 20 GeV
proved adequate for the study of the same observables but with the
two leading jets above 50 GeV (at the LHC). In terms of what constitutes
`best practice' we recommend that serious studies involve cross checks
to verify that the dependence on \texttt{bornktmin} does not modify
predictions by more than 1-2\%.


\section{Scale choices}

In the \POWHEG{} algorithm, each event is built by first producing
what is referred to as an underlying Born configuration, here a QCD
$2\rightarrow2$ scattering, before proceeding to generate the hardest
branching in the event. We have elected to use the $p_{T}$ of the
underlying-Born configuration as the renormalization and factorization
scale in obtaining the fixed-order NLO predictions, effectively resumming
virtual corrections to the associated t-channel gluon propagator ---
see the subroutine \texttt{set\_fac\_ren\_scales} in \texttt{Dijet/Born\_phsp.}f.
This same scale choice is used in generating the underlying Born kinematics,
of the \POWHEG{} events, while the component of the hardest-emission
cross section, responsible for the subsequent generation of the hardest
branching kinematics, uses the transverse momentum of the branching,
both in the evaluation of the strong coupling constant and the PDF. 



\begin{thebibliography}{10}


\bibitem{Nason:2004rx}
  P.~Nason,
  ``A new method for combining NLO QCD with shower Monte Carlo algorithms,''
  JHEP {\bf 0411} (2004) 040
  [arXiv:hep-ph/0409146].
  %%CITATION = JHEPA,0411,040;%%

%\cite{Frixione:2007vw}
\bibitem{Frixione:2007vw}
  S.~Frixione, P.~Nason and C.~Oleari,
``Matching NLO QCD computations with Parton Shower simulations: the POWHEG
method,''
  JHEP {\bf 0711} (2007) 070
  [arXiv:0709.2092 [hep-ph]].
  %%CITATION = JHEPA,0711,070;%%

%\cite{Alioli:2010xd}
\bibitem{Alioli:2010xd}
  S.~Alioli, P.~Nason, C.~Oleari and E.~Re,
``A general framework for implementing NLO calculations in shower Monte Carlo
  programs: the POWHEG BOX,''
  [arXiv:1002.2581 [hep-ph]].
  %%CITATION = ARXIV:1002.2581;%%

%\cite{Alioli:2010xa}
\bibitem{Alioli:2010xa}
  S.~Alioli, K.~Hamilton, P.~Nason, C.~Oleari and E.~Re,
  ``Jet pair production in POWHEG,''
  arXiv:1012.3380 [hep-ph].
  %%CITATION = ARXIV:1012.3380;%%


\end{thebibliography}

\end{document}
