\documentclass[paper]{JHEP3}
\usepackage{amssymb,enumerate,url}
\bibliographystyle{JHEP}

%%%%%%%%%% Start TeXmacs macros
\newcommand{\tmtextit}[1]{{\itshape{#1}}}
\newcommand{\tmtexttt}[1]{{\ttfamily{#1}}}
\newenvironment{enumeratenumeric}{\begin{enumerate}[1.] }{\end{enumerate}}
\newcommand\sss{\mathchoice%
{\displaystyle}%
{\scriptstyle}%
{\scriptscriptstyle}%
{\scriptscriptstyle}%
}

\newcommand\as{\alpha_{\sss\rm S}}
\newcommand\POWHEG{{\tt POWHEG}}
\newcommand\HERWIG{{\tt HERWIG}}
\newcommand\PYTHIA{{\tt PYTHIA}}
\newcommand\MCatNLO{{\tt MC@NLO}}

\newcommand\kt{k_{\sss\rm T}}

\newcommand\pt{p_{\sss\rm T}}
\newcommand\LambdaQCD{\Lambda_{\scriptscriptstyle QCD}}
%%%%%%%%%% End TeXmacs macros


\title{The \mbox{POWHEG BOX} user manual:\\
  Single-top Wt-channel process (DR and DS)} \vfill
\author{Emanuele Re\\
  Institute for Particle Physics Phenomenology, Department of Physics\\
  University of Durham, Durham, DH1 3LE, UK\\
  E-mail: \email{emanuele.re@durham.ac.uk}\\
  New e-mail: \email{emanuele.re@lapth.cnrs.fr}}


\vskip -0.5truecm

\keywords{POWHEG, Shower Monte Carlo, NLO}

\abstract{This note documents the use of the package \tmtexttt{POWHEG BOX}
  for the single-top $Wt$-channel production process.
  Results can be easily interfaced to shower Monte Carlo programs, in
  such a way that both NLO and shower accuracy are maintained. Two
  alternative prescriptions to quantify interference effects with
  $t\bar{t}$ are available.
  Please read carefully the manual before using this code.}
\preprint{\today\\ \tmtexttt{POWHEG BOX} 2.0}

\begin{document}

The code has been migrated to \tmtexttt{POWHEG-BOX-V2} and validated.
It can be downloaded from here:
\begin{center}
  \url{svn://powhegbox.mib.infn.it/trunk/User-Processes-V2/ST_wtch_DR}
\end{center}
Apart from obvious changes, the content of this manual is still valid.

\section{Introduction}

The \tmtexttt{POWHEG BOX} program is a framework for implementing NLO
calculations in Shower Monte Carlo programs according to the
\POWHEG{} method. An explanation of the method and a discussion of
how the code is organized can be found in
refs.~\cite{Nason:2004rx,Frixione:2007vw,Alioli:2010xd}.  The code is
distributed according to the ``MCNET GUIDELINES for Event Generator
Authors and Users'' and can be found at the web page \\
%
\begin{center}
 \url{http://powhegbox.mib.infn.it}.
\end{center}
%
~\\
%
In the following we will focus on the implementation of single-top
$Wt$-channel production, whose source files can be found in
the \tmtexttt{POWHEG-BOX/ST\_wtch\_DR} and
\tmtexttt{POWHEG-BOX/ST\_wtch\_DS} subdirectories.

This program is an implementation of the NLO cross section calculated
in~\cite{Frixione:2008yi} in the \POWHEG{} formalism of
refs.~\cite{Nason:2004rx,Frixione:2007vw}.  A detailed description of
the implementation can be found in ref.~\cite{Re:2010bp}. Spin
correlations of the top-quark decay products are included with a
method analogous to the one described in~\cite{Frixione:2007zp}, and
the relevant matrix elements for the full decayed amplitudes were
obtained using MadGraph~\cite{Alwall:2007st}. To evaluate scalar
integrals, the program needs to be linked with
QCDloop~\cite{Ellis:2007qk}, that can be downloaded from\\
%
\begin{center}
 \url{http://qcdloop.fnal.gov}.
\end{center}
%
~\\
%
In this note we give all the necessary information to run the program.
A quick description on how to install and properly link QCDloop is also
given in the following.


\section{Installation}
Before describing the installation procedure, we want to recall that
single-top $Wt$-channel production is not well defined beyond LO,
since there are interference effects with $t\bar{t}$ production. To
deal with this problem, we used an approach very similar to the one
described in ref.~\cite{Frixione:2008yi}.  A description of the
problem and of the \POWHEG{} implementation used in this code can be
found in~\cite{Re:2010bp}. As described in the aforementioned
references, the size of interference effects with $t\bar{t}$ (and all
the related issues) can be quantified essentially by comparing results
obtained with two different prescriptions: DR (\emph{Diagram Removal})
and DS (\emph{Diagram Subtraction}). Hence, the two directories.\\

In order to run the \tmtexttt{POWHEG BOX} program, we recommend the
reader to start from the \tmtexttt{POWHEG BOX} user manual, which
contains all the information and settings that are common between
all subprocesses. In this note we focus on the settings and
parameters specific to the single-top $Wt$-channel implementation.

In the following, we will describe how to run the DR code. Same
considerations hold also for DS: in fact, the structure of the two
codes (and the relevant files) is similar.\footnote{We recall that in
  the DS case a slightly modified version of the file
  \tmtexttt{sigreal.f} has to be used with respect to the same file
  present in the main folder. The \tmtexttt{Makefile} is build such
  that the proper file will be compiled and linked.} Explicit
instructions will be given when there are relevant differences.

As anticipated before, the program needs to be linked with QCDloop: to
this end, the \tmtexttt{Makefile} variables \tmtexttt{VPATH} and
\tmtexttt{LIBSQCDLOOP} have to be set properly.

\subsection{Installation of QCDloop}
The procedure to install QCDloop can be found in its webpage. Here
we summarize it:
\begin{enumerate}
\item Download and decompress the QCDloop tarball.
\item Edit the \tmtexttt{QCDloop-1.9/ff/ffinit\_mine.f} file by
  setting the variable 'path' from its current value to the absolute
  path of the folder where \tmtexttt{ff} will be located.  Therefore,
  the user needs to change the line 786 from

  \tmtexttt{path =
    '/Users/ellis/QCDLoop-1.9/ff/'}

  to something like

  \tmtexttt{path =
    '/home/username/path\_to\_QCDLoop-1.9/QCDLoop-1.9/ff/'}

\item Compile QCDloop by typing make in the QCDLoop-1.9 directory. If
  the files \tmtexttt{libqcdloop.a} and \tmtexttt{libff.a} are now
  present in the directories \tmtexttt{ql} and \tmtexttt{ff}, the
  QCDloop library has been compiled and installed properly. Compiling
  QCDloop with the same compiler used for \tmtexttt{POWHEG} may be
  necessary in order to run the two programs together successfully. In
  this case, remember to change the compiler line in the QCDloop
  \tmtexttt{Makefile} (the default is \tmtexttt{g77}).
\end{enumerate}

\section{Generation of events and showering}

After having downloaded and placed the QCDloop library properly,
the executable is built with the following commands\\~\\
\tmtexttt{
%
  \$ cd POWHEG-BOX/ST\_wtch\_DR\\
% 
  \$ make pwhg\_main\\
%
}
\\
In the \tmtexttt{testrun} folder, there are several examples of input files.
For example, you can start a run doing\\~\\
\tmtexttt{
%
  \$ cd testrun\\
% 
  \$ ../pwhg\_main\\
%
}
\\
The input file read in this case is \tmtexttt{powheg.input} and
at the end a file named \tmtexttt{pwgevents.lhe} will contain 100000
events for $tW^-$ production at the LHC, in the Les Houches format.
To shower them with \PYTHIA{} do\\~\\
\tmtexttt{
%
\$ cd POWHEG-BOX/ST\_wtch\_DR\\
%
\$ make main-PYTHIA-lhef \\
%
\$ cd testrun \\
%
\$ ../main-PYTHIA-lhef\\
%
}
\\
Similar commands will run the \HERWIG{} shower.

\section{Process specific input parameters}
\tmtexttt{
  facscfact 1 ! factorization scale factor: mufact=muref*facscfact\\
  renscfact 1 ! renormalization scale factor: muren=muref*renscfact\\
}
\\
Factorization and renormalization scale factors appearing here have to
do with the computation of the inclusive cross section (i.e.~the
$\bar{B}$ function~\cite{Nason:2004rx,Frixione:2007vw,Alioli:2010xd}),
and can be varied by a factor of order 1 to study scale dependence.
The natural choice for this process is the mass of the top-quark. We
choose to perform the NLO calculation keeping these scales fixed.  The
experienced user can change this setting modifying the
\tmtexttt{set\_fac\_ren\_scales} routine.
\\~\\
It follows a description of parameters which are relevant for
this production process:
\begin{itemize}
\item As discussed in ref.~\cite{Re:2010bp} when the DS procedure is
  used, negative weights can appear. To keep track of them properly,
  in the input files for DS we decided to activate the
  \tmtexttt{withnegweights} flag. The fraction of negative-weighted
  events can be reduced by setting the \emph{folding} variables
  properly, as discussed in the next item.
\item If the fraction of negative weights is large, one may increase
  \tmtexttt{foldcsi}, \tmtexttt{foldy}, \tmtexttt{foldphi}.  Allowed
  values are 1, 2, 5, 10, 25, 50. The speed of the program is
  inversely proportional to the product of these numbers, so that a
  reasonable compromise should be found.  Our experiences tell us
  that, even at LHC energies, the fraction of negative weights in
  $\bar B$ calculation is such that the numbers provided in the
  examples need not to be changed. In particular, in the DS case,
  it is recommended to leave the default foldings on the csi and y
  variables, in order to generate an event sample with a low
  number of negative-weighted events.
\item In the $Wt$-channel case, it is not needed to activate the
  \tmtexttt{withdamp} option. Therefore, this token is set to
  zero. The same setting can be obtained by commenting or deleting the
  corresponding line, which was left as a reminder.
\end{itemize}

Other parameters are those specifically related to the single-top
$Wt$-channel processes: from revision 1.0, some of these parameters
are mandatory (the program stops if they are missing), other are
optional (default values are assigned in \tmtexttt{init\_couplings.f},
but are overwritten if the token is found uncommented in the input
file, as in previous versions).


For the production step, the relevant
parameters are:\\
~\\
\tmtexttt{
! mandatory production parameters\\
ttype   1 \phantom{aaaaaaa} ! 1 for t, -1 for tbar\\
topmass 175.0 \phantom{a}   ! top mass\\
}
%
~\\
where the value of \tmtexttt{ttype} is used to decide if top or
antitop quarks will be produced and \tmtexttt{topmass} set the
top-quark mass.

In the current released version, top-quark decay products are always
generated by \POWHEG{}, accordingly to a procedure very similar to the
one of ref.~\cite{Frixione:2007zp}. Therefore, the following
parameters are mandatory too:\\
~\\
\tmtexttt{
! mandatory parameters used in decay generation\\
topdecaymode 10000           ! decay mode: the 5 digits correspond to the following\\
\phantom{topdecaymode 10000} ! top-decay channels (l,mu,tau,u,c) \\
\phantom{topdecaymode 10000} ! 0 means close, 1 open\\
wdecaymode 10000\phantom{aa} ! decay mode: the 5 digits correspond to the following\\
\phantom{topdecaymode 10000} ! primary-w-decay channels (l,mu,tau,u,c) \\
\phantom{topdecaymode 10000} ! 0 means close, 1 open\\
tdec/elbranching 0.108  ! W electronic branching fraction\\
}
%
~\\
where the value of the \tmtexttt{topdecaymode} token is formed by five
digits, each representing the maximum number of the following
particles at the (parton level) decay of the $t$ ($\bar{t}$) quark:
$e^{\pm}$, $\mu^{\pm}$, $\tau^{\pm}$,
$\stackrel{\scriptscriptstyle{(-)}}{u}$,
$\stackrel{\scriptscriptstyle{(-)}}{c}$.  Thus, for example, 10000
means $t \rightarrow e^+ \nu_e b$, 11100 means all semileptonic
decays, 00011 means fully hadronic. The same syntax has to be used to
set the \tmtexttt{wdecaymode} token, which controls the decay of the
$W$-boson produced in the hard process (i.e.~not the $W$-boson present
in the decay chain of the top quark).

At the end of the event generation, just before writing the partonic
event on the LHEF, a reshuffling procedure is called, in order to put
outgoing charged leptons and quarks on mass shell. In revisions before
r213, for Wt-channel these particles were left massless. To reshuffle
momenta, default values are used for masses, unless the entries
\tmtexttt{lhfm/***mass} are present in the input card.

The optional parameters are listed below. Their meaning is
self-explanatory. We remind that it is not allowed to set any entry of
the CKM matrix exactly equal to zero.\\
~\\
\tmtexttt{
! optional production parameters \\
! (defaults defined in init\_couplings.f) \\
\#wmass 80.4                 \phantom{aaaaaaaaaaaaa} ! w mass \\
\#sthw2 0.23113              \phantom{aaaaaaaaaa} ! (sin(theta\_W))**2 \\
\#alphaem\_inv  127.011989   \phantom{a} ! 1/alphaem \\
\#CKM\_Vud 0.9740            \phantom{aaaaaaaaa} ! CKM matrix entries ...\\
\#CKM\_Vus 0.2225 \\
\#CKM\_Vub 0.000001 \\
\#CKM\_Vcd 0.2225 \\
\#CKM\_Vcs 0.9740 \\
\#CKM\_Vcb 0.000001 \\
\#CKM\_Vtd 0.000001 \\
\#CKM\_Vts 0.000001 \\
\#CKM\_Vtb 1.0\\
~\\
! optional parameters used in decay generation\\
! (defaults defined in init\_couplings.f)\\
\#topwidth         1.7      \phantom{aaaaaaa} ! top width \\
\#wwidth           2.141    \phantom{aaaaaaa} ! w width \\
\#lhfm/cmass       1.5      \phantom{aaaaa} ! c mass \\
\#lhfm/bmass       5.0      \phantom{aaaaa} ! b mass \\
\#lhfm/emass       0.000511 \phantom{} ! e mass \\
\#lhfm/mumass      0.1056   \phantom{a} ! mu mass \\
\#lhfm/taumass     1.777    \phantom{a} ! tau mass \\
}

\section{Generation of a sample with $t$ and $\bar{t}$
  events}\label{sec:merging}

The user can be interested in the generation of a sample where both
top and antitop events appear. To this purpose, a script and a
dedicated executable have been included. The script is named
\tmtexttt{merge\_ttb.sh} and can be found in the directory
\tmtexttt{testrun}.  It can be run in any subfolder of
\tmtexttt{ST\_wtch\_DR}. Three inputs are mandatory: the first two are
the prefixes of the input files used to generate $t$ and $\bar{t}$
events. The third input has to be an integer and correspond to the
total number of events that the final \emph{merged} sample will
contain. The script has to be run twice, using a positive integer
value at the first call and its opposite afterward.  Therefore, for
example, to produce a sample of 10000 events at the LHC, starting
from the input files \tmtexttt{lhc\_wt\_t-powheg.input} and \tmtexttt{lhc\_wt\_tb-powheg.input}, the invocation lines should be as follows:~\\~\\
\tmtexttt{\$ sh merge\_ttb.sh lhc\_wt\_t lhc\_wt\_tb 10000}\\~\\
and then~\\~\\
\tmtexttt{\$ sh merge\_ttb.sh lhc\_wt\_t lhc\_wt\_tb -10000}\\~

Few remarks are needed:
\begin{itemize}
\item it is responsibility of the user to check that the 2 input files
  are equal. The \tmtexttt{ttype} tokens have to be different,
  obviously.
\item the two values of \tmtexttt{numevts} are not really used: the
  program re-calculate the needed values as a function of the $t$ and
  $\bar{t}$ cross sections and of the total number of events to be
  generated.
\item the final event file is always named
  \tmtexttt{t\_tb\_sample-events.lhe}. In the header section it also
  contains a copy of the two input files used to generate it, for
  cross-checking purposes
\end{itemize}


\begin{thebibliography}{10}

\bibitem{Nason:2004rx}
  P.~Nason,
  ``A new method for combining NLO QCD with shower Monte Carlo algorithms,''
  JHEP {\bf 0411} (2004) 040
  [arXiv:hep-ph/0409146].
  %%CITATION = JHEPA,0411,040;%%

%\cite{Frixione:2007vw}
\bibitem{Frixione:2007vw}
  S.~Frixione, P.~Nason and C.~Oleari,
``Matching NLO QCD computations with Parton Shower simulations: the POWHEG
method,''
  JHEP {\bf 0711} (2007) 070
  [arXiv:0709.2092 [hep-ph]].
  %%CITATION = JHEPA,0711,070;%%

%\cite{Alioli:2010xd}
\bibitem{Alioli:2010xd}
  S.~Alioli, P.~Nason, C.~Oleari and E.~Re,
``A general framework for implementing NLO calculations in shower Monte Carlo
programs: the POWHEG BOX,''
  JHEP {\bf 1006}, 043 (2010)
  [arXiv:1002.2581 [hep-ph]].
  %%CITATION = JHEPA,1006,043;%%

%\cite{Frixione:2008yi}
\bibitem{Frixione:2008yi}
  S.~Frixione, E.~Laenen, P.~Motylinski, B.~R.~Webber and C.~D.~White,
  ``Single-top hadroproduction in association with a W boson,''
  JHEP {\bf 0807}, 029 (2008)
  [arXiv:0805.3067 [hep-ph]].
  %%CITATION = JHEPA,0807,029;%%

%\cite{Re:2010bp}
\bibitem{Re:2010bp}
  E.~Re,
  ``Single-top Wt-channel production matched with parton showers using the
  POWHEG method,''
  Eur.\ Phys.\ J.\  C {\bf 71}, 1547 (2011)
  [arXiv:1009.2450 [hep-ph]].
  %%CITATION = EPHJA,C71,1547;%%

% %\cite{Alioli:2009je}
% \bibitem{Alioli:2009je}
%   S.~Alioli, P.~Nason, C.~Oleari and E.~Re,
%   ``NLO single-top production matched with shower in POWHEG: s- and t-channel
%   contributions,''
%   JHEP {\bf 0909}, 111 (2009)
%   [arXiv:0907.4076 [hep-ph]].
%   %%CITATION = JHEPA,0909,111;%%

%\cite{Frixione:2007zp}
\bibitem{Frixione:2007zp}
  S.~Frixione, E.~Laenen, P.~Motylinski and B.~R.~Webber,
  ``Angular correlations of lepton pairs from vector boson and top quark decays
  in Monte Carlo simulations,''
  JHEP {\bf 0704}, 081 (2007)
  [arXiv:hep-ph/0702198].
  %%CITATION = JHEPA,0704,081;%%

%\cite{Alwall:2007st}
\bibitem{Alwall:2007st}
  J.~Alwall {\it et al.},
  ``MadGraph/MadEvent v4: The New Web Generation,''
  JHEP {\bf 0709}, 028 (2007)
  [arXiv:0706.2334 [hep-ph]].
  %%CITATION = JHEPA,0709,028;%%

%\cite{Ellis:2007qk}
\bibitem{Ellis:2007qk}
  R.~K.~Ellis and G.~Zanderighi,
  ``Scalar one-loop integrals for QCD,''
  JHEP {\bf 0802}, 002 (2008)
  [arXiv:0712.1851 [hep-ph]].
  %%CITATION = JHEPA,0802,002;%%

% %\cite{Cacciari:2005hq}
% \bibitem{Cacciari:2005hq}
%   M.~Cacciari and G.~P.~Salam,
%   ``Dispelling the $N^{3}$ myth for the $k_t$ jet-finder,''
%   Phys.\ Lett.\  B {\bf 641}, 57 (2006)
%   [arXiv:hep-ph/0512210].
%   %%CITATION = PHLTA,B641,57;%%

% %\cite{Boos:2001cv}
% \bibitem{Boos:2001cv}
%   E.~Boos {\it et al.},
%   ``Generic user process interface for event generators,''
%   [arXiv:hep-ph/0109068].
%   %%CITATION = HEP-PH/0109068;%%

% %\cite{Alwall:2006yp}
% \bibitem{Alwall:2006yp}
%   J.~Alwall {\it et al.},
%   ``A standard format for Les Houches event files,''
%   Comput.\ Phys.\ Commun.\  {\bf 176} (2007) 300
%   [arXiv:hep-ph/0609017].
%   %%CITATION = CPHCB,176,300;%%

% %\cite{Altarelli:1989wu}
% \bibitem{Altarelli:1989wu} T. Sj\"ostrand et~al., in
%   ``Z physics at LEP1: Event generators and software,'',  eds.
%   G.~Altarelli, R.~Kleiss and C.~Verzegnassi, Vol 3, pg. 327.
%   %%CITATION = CERN-89-08-V-3;%%

% %\cite{Whalley:2005nh}
% \bibitem{Whalley:2005nh}
%   M.~R.~Whalley, D.~Bourilkov and R.~C.~Group,
%   ``The Les Houches accord PDFs (LHAPDF) and LHAGLUE,''
%   [arXiv:hep-ph/0508110].
%   %%CITATION = HEP-PH/0508110;%%

% % %\cite{Yao:2006px}
% % \bibitem{Yao:2006px}
% %   W.~M.~Yao {\it et al.}  [Particle Data Group],
% %   %``Review of particle physics,''
% %   J.\ Phys.\ G {\bf 33}, 1 (2006).
% %   %%CITATION = JPHGB,G33,1;%%

\end{thebibliography}

\end{document}





