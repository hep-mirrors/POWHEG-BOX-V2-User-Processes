\documentclass[a4paper,11pt]{article}
\usepackage{amssymb,enumerate}
\usepackage{amsmath}
\usepackage{bbm}
\usepackage{url}
\usepackage{cite}
\usepackage{graphics}
\usepackage{xspace}
\usepackage{epsfig}
\usepackage{subfigure}

\setlength\paperwidth  {210mm}%
\setlength\paperheight {300mm}%	

\textwidth 160mm%		% DEFAULT FOR LATEX209 IS a4
\textheight 230mm%

\voffset -1in
\topmargin   .05\paperheight	% FROM TOP OF PAGE TO TOP OF HEADING (0=1inch)
\headheight  .02\paperheight	% HEIGHT OF HEADING BOX.
\headsep     .03\paperheight	% VERT. SPACE BETWEEN HEAD AND TEXT.
\footskip    .07\paperheight	% FROM END OF TEX TO BASE OF FOOTER. (40pt)


\hoffset -1in				% TO ADJUST WITH PAPER:
	\oddsidemargin .13\paperwidth	% LEFT MARGIN FOR ODD PAGES (10)
	\evensidemargin .15\paperwidth	% LEFT MARGIN FOR EVEN PAGES (30)
	\marginparwidth .10\paperwidth	% TEXTWIDTH OF MARGINALNOTES
	\reversemarginpar		% BECAUSE OF TITLEPAGE.

%%%%%%%%%% Start TeXmacs macros
\newcommand{\tmtextit}[1]{{\itshape{#1}}}
\newcommand{\tmtexttt}[1]{{\ttfamily{#1}}}
\newenvironment{enumeratenumeric}{\begin{enumerate}[ 1.] }{\end{enumerate}}
\newcommand\sss{\mathchoice%
{\displaystyle}%
{\scriptstyle}%
{\scriptscriptstyle}%
{\scriptscriptstyle}%
}
\newcommand\PSn{\Phi_{n}}
\newcommand{\tmop}[1]{\ensuremath{\operatorname{#1}}}



\newcommand\Lum{{\cal L}}
\newcommand\matR{{\cal R}}
\newcommand\Kinnpo{{\bf \Phi}_{n+1}}
\newcommand\Kinn{{\bf \Phi}_n}
\newcommand\PSnpo{\Phi_{n+1}}
\newcommand\as{\alpha_{\sss\rm S}}
\newcommand\asotpi{\frac{\as}{2\pi}}

\newcommand\POWHEG{{\tt POWHEG}}
\newcommand\POWHEGBOX{{\tt POWHEG BOX}}
\newcommand\POWHEGBOXV{{\tt POWHEG-BOX-V2}}
\newcommand\PYTHIA{{\tt PYTHIA}}
\newcommand\POWHEGpPYTHIA{{\tt POWHEG+PYTHIA}}
\newcommand\HERWIG{{\tt HERWIG}}

\def\lq{\left[} 
\def\rq{\right]} 
\def\rg{\right\}} 
\def\lg{\left\{} 
\def\({\left(} 
\def\){\right)} 

\def\beq{\begin{equation}}
\def\beqn{\begin{eqnarray}}
\def\eeq{\end{equation}}
\def\eeqn{\end{eqnarray}}

\def\mr{\mathrm}
\def\vbfevmv{VBF $e^+\nu_e\mu^+\nu_\mu jj$\;}
\def\vbfww{VBF $W^+W^+jj$\;}
\def\wpp{W^+W^+}
\def\evmv{e^+\nu_e\mu^+\nu_\mu}
\def\pbox{{\tt POWHEG BOX}}
\def\pwg{{\tt POWHEG}}
\def\mc{\mathcal}

%%%%%%%%%% End TeXmacs macros

\title{Manual for electroweak $W^+W^+ jj$ and $W^-W^- jj$ production in the \POWHEGBOXV{}}
\date{}
%\author{}
%\keywords{}
%\abstract{}
%\preprint{}


\begin{document}
\maketitle
%
\noindent
The {\tt VBF\_Wp\_Wp} program is an implementation of the electroweak
$W^+W^+ jj$ and $W^-W^- jj$ production cross sections within the \pbox{} framework. It
complements the {\tt Wp\_Wp\_j\_j} program, which provides the
QCD-induced contributions to same-sign weak-boson production in hadronic
collisions.
\\[2ex]
This document describes the input parameters that are specific to the
implementation of the EW channel in the \POWHEGBOXV. Note that the recommended settings
are very similar to those for the QCD production mode.  The parameters
that are common to all \POWHEGBOXV{} implementations are described in the {\tt POWHEG-BOX-V2/Docs}
directory.
\\[2ex]
If you use this program, please quote
Refs.~\cite{Jager:2009xx,JZ,Alioli:2010xd}.


\section*{Running the program}
%
Download the \POWHEGBOXV{}, following the instructions at the web site 
\\[2ex]
{\tt http://powhegbox.mib.infn.it/}
\\[2ex] 
In addition, retrieve the {\tt VBF\_Wp\_Wp} directory as explained, 
and go to the new directory via
\\[2ex]
{\tt \$ cd POWHEG-BOX-V2/User-Processes-V2/VBF\_Wp\_Wp}
\\[2ex]
Running is most conveniently done in a separate directory, for instance do
\\[2ex]
{\tt \$ mkdir testrun}
\\[2ex]
The directory must contain the {\tt powheg.input} file and, for
parallel running, a {\tt pwgseeds.dat} file (see {\tt manual-BOX.pdf}
and {\tt Manyseeds.pdf}).
\\[2ex]
Before compiling make sure that:
\begin{itemize}
\item 
{\tt fastjet} is installed and {\tt fastjet-config} is in the path,
\item 
{\tt lhapdf} is installed and {\tt lhapdf-config} is in the path,
\item
{\tt gfortran}, {\tt ifort} or {\tt g77} is in the path, and the
appropriate libraries are in the environment variable {\tt
  LD\_LIBRARY\_PATH}. 
\end{itemize}
%
If {\tt LHAPDF} or {\tt fastjet} are not installed, the code can still
be run using a dummy analysis routine and built-in PDFs, see the {\tt
  Makefile} in {\tt VBF\_Wp\_Wp}.
%
\\[2ex]
After compiling, enter the testrun directory:
\\[2ex]
{\tt \$ cd testrun}
\\[2ex]
When executing
\\[2ex]
{\tt \$../pwhg\_main}
\\[2ex]
the program will ask you to
\\[2ex]
{\tt enter which seed}
\\[2ex]
The program requires you to enter an index that specifies the line
number in the {\tt pwgseeds.dat} file where the seed of the random
number generator to be used for the run is stored. All results
generated by the run will be stored in files named {\tt
  *-[index].*}. When running on parallel CPUs, make sure that each
parallel run has a different index.
\\[2ex]
The program can be run in several steps. Each new step requires the
completion of the previous step.
% 
%%%%%%%%%%%%%%%%%%%
\subsection*{Step 1}
%%%%%%%%%%%%%%%%%%%
%
At this point the user
has to decide whether the weak bosons are to be generated on-shell
({\tt zerowidth = 1}) or off-shell, distributed according to a
Breit-Wigner distribution ({\tt zerowidth = 0}).  By default on-shell
weak bosons are generated. The user can also select the signs and decay modes of
the weak bosons by assigning appropriate PDG code to the parameters
{\tt vdecaymodew1} and {\tt vdecaymodew2} in {\tt powheg.input}. Note
that only leptonic decays are supported. 
%
In the $W^+W^+ jj$ mode, 
the parameters {\tt vdecaymodew1} and {\tt vdecaymodew2} are to be set to -11, -13 or -15  for decays to
positrons, anti-muons, or anti-taus, respectively. Is it also
possible to generate events with combinations of positron and anti-muon
decays, in which case {\tt vdecaymodew1} and {\tt vdecaymodew2} at the same time should
be set to -113. A combination of all leptons (positron, anti-muons, and 
anti-taus) can be obtained by setting {\tt vdecaymodew1} and {\tt vdecaymodew2}
to -135.
%
Predictions for $pp\to W^-W^-jj$ can be obtained by obvious sign changes in the aforementioned input parameters. 
%
The template analysis file
{\tt pwhg\_analysis.f} needed in subsequent steps of the analysis is
designed for the $e^+\nu_e \,\mu^+\nu_{\mu}$ and  $e^-\bar\nu_e \,\mu^-\bar\nu_{\mu}$ modes.
%
%In order to obtain predictions for $pp\to W^-W^-jj$, one can do a calculation for $\bar p\bar p\to W^+W^+jj$, treat the decay leptons of the $W^+$ bosons as if they were negatively charged and reverse the momentum directions for parity-odd distributions. In the \POWHEGBOX{}, predictions for anti-protons can be obtained by setting the input parameters {\tt ih1} and {\tt ih2} to {\tt -1} in the {\tt powheg.input} file.  
%
\\[2ex]
%
We recommend to generate the grid with the option {\tt fakevirt 1} in
{\tt powheg.input}. When using this option, the virtual contribution
is replaced by a fake one proportional to the Born term. This speeds
up the generation of the grid.
\\[2ex]
In the \POWHEGBOXV{} grids can be generated in the parallel mode. To that end, set
\\[2ex]
{\tt manyseeds   1}
\\
{\tt xgriditeration   1}
\\
{\tt parallelstage     1}
\\[2ex]
For a default setup one needs about 50--100 jobs with the number of calls set by
\\[2ex] 
{\tt ncall1 10000}
\\[2ex] 
for each. 
%
Run the program via
\\[2ex]
{\tt \$../pwhg\_main}
\\[2ex]
When prompted
\\[2ex]
{\tt enter which seed}
\\[2ex]
enter 1 or any other valid seed number.
\\[2ex]
The program stops automatically after the compilation of
this step.
%
We recommend performing at least  3 iterations. After completion of the first iteration, change the value of {\tt xgriditeration} according to the respective iteration, and re-do step~1. 
%
%%%%%%%%%%%%%%%%%%%
\subsection*{Step 2}
%%%%%%%%%%%%%%%%%%%
%
To produce true NLO results comment out the {\tt fakevirt} token from {\tt powheg.input}.
The runs must be performed in the directory where the previously generated grids are stored.
\\[2ex]
The integration and upper bound for the generation of btilde can be
performed with 50-100 runs with 5000-10000 calls each. Set for
instance
\\[2ex]
{\tt ncall2 5000}
\\
{\tt itmx2 1}
\\[2ex]
in {\tt powheg.input}.
\\[2ex]
Folding numbers that are appropriate for runs at LHC energy are:
\\[2ex]
{\tt foldcsi 5 ! number of folds on csi integration}
\\
{\tt foldy 5 ! number of folds on y integration}
\\
{\tt foldphi 10 ! number of folds on phi integration}
\\[2ex]
%
In addition, adapt the values for steering the parallel mode of the program to 
\\[2ex]
{\tt xgriditeration   1}
\\
{\tt parallelstage  2}
\\[2ex]
%
Run jobs in parallel, in the same way as explained for step~1 above. 
\\[2ex]
Upon the completion of this step, for each parallel run a file {\tt
  pwg-*-NLO.top} is generated (where the * denotes the integer
identifier of the run).  These files contain the histograms defined in
{\tt pwhg\_analysis.f} at NLO-QCD accuracy, if the variable {\tt
  bornonly} is set to zero in {\tt powheg.input}.  Setting {\tt
  bornonly} to 1 yields the respective LO results. In either case, the
individual results of the parallel runs can be combined with the help
of the {\tt mergedata.f} file contained in the {\tt plot-aux}
directory.  To this end, just compile the file by typing, e.g.,
\\[2ex]
{\tt \$ gfortran -o mergedata mergedata.f}
\\[2ex]
and run the resulting executable. 
 in your run directory
{\tt mergedata 1 *NLO.top}. 
The program expects a number between 1 and 5 and a list of files to
merge. If no number or no list is specified, the user will be prompted
by the program to enter them. Running mergedata without any arguments
will also explain what the 5 different options are. {\tt mergedata}
will combine the histograms into a file called {\tt fort.12}. We
provide a file {\tt plot-aux/genplots.sh} and {\tt
  plot-aux/gnuplotsplit.gp} which can be used to plot the resulting
histograms. First run {\tt genplots.sh}
\\[2ex]
{\tt ./genplots.sh file nameoutput}
\\[2ex]
or if a comparison between two different runs is wanted
\\[2ex]
{\tt ./genplots.sh file1 file2 nameoutput}
\\[2ex]
This will result in a file called {\tt genplots.gp}. In the later case
the file {\tt plot-aux/pastegnudata.f} has to be compiled and either put
in the working directory or in the users
path. After {\tt genplots.sh} has been run, the user can produce a set
of .eps files with gnuplot running
\\[2ex]
{\tt ./gnuplotsplit.gp genplots.gp}
%
%%%%%%%%%%
\subsection*{Step 3}
%%%%%%%%%%
%
Also this step can be run in parallel. 
We recommend to set
\\[2ex]
{\tt ncall1 500000}
\\
{\tt itmx1 4}
\\
{\tt ncall2 500000}
\\
{\tt itmx2 4}
\\
{\tt nubound 100000}
\\
{\tt parallelstage  3}
\\[2ex]
%
The program will stop after
completion of this step.  The parallel execution of the program is
performed as in the previous steps.
%
%%%%%%%%%%%%
\subsection*{Step 4}
%%%%%%%%%%%%
%
Set {\tt numevts} to the number of events you want to generate per
process, for example
\\[2ex]
{\tt numevts 5000}
\\[2ex]
and run in parallel.  
\\[2ex]
At this point, files of the form {\tt pwgevents-[index].lhe} are
present in the run directory.
\\[2ex]
Count the events:
\\[2ex]
{\tt \$ grep '/event' pwgevents-*.lhe | wc}
\\[2ex]
The events can be merged into a single event file by
\\[2ex]
{\tt cat pwgevents-*.lhe | grep -v '/LesHouchesEvents' >
  pwgevents.lhe}
%
%%%%%%%%%%%%%%%%%%
\section*{Analyzing the events}
%%%%%%%%%%%%%%%%%%
%
It is straightforward to feed the {\tt *.lhe} events into a generic
shower Monte Carlo program, within the analysis framework of each
experiment. We also provide a sample analysis that computes several
histograms and stores them in topdrawer output files.
\\[2ex]
Doing (from the {\tt VBF\_Wp\_Wp} directory)
\\[2ex]
{\tt \$ make lhef\_analysis}
\\[2ex]
{\tt \$ cd testrun}
\\[2ex]
{\tt ../lhef\_analysis}
\\[2ex]
analyses the bare {\tt POWHEG BOX} output, creating the topdrawer file
{\tt LHEF\_analysis.top}. The targets {\tt main-HERWIG-lhef} and {\tt
  main-PYTHIA-lhef} are instead used to perform the analysis on events
fully showered using {\tt HERWIG} or {\tt PYTHIA}. Various setting of
the Monte Carlo can be modified by editing the files {\tt
  setup-PYTHIA-lhef.f} and {\tt setup-HERWIG-lhef.f} respectively.
%
%%%%%%%%%%%%%%%%%%%%%%%%%%
%
\begin{thebibliography}{99}
\bibitem{Jager:2009xx} B.~J\"ager, C.~Oleari, D.~Zeppenfeld, {\em
    Next-to-leading order QCD corrections to $W^+W^+jj$ and $W^-W^-jj$
    production via weak-boson fusion}, Phys.~Rev.~{\bf D80} (2009)
  034022.  [arXiv:0907.0580 [hep-ph]].

\bibitem{JZ} B.~J\"ager, G.~Zanderighi, {\em NLO corrections to
    electroweak and QCD production of $W^+W^+$ plus two jets in the
    POWHEG BOX}, JHEP {\bf 1111} (2011) 055.
  [arXiv:1108.0864 [hep-ph]].
  
\bibitem{Alioli:2010xd} S.~Alioli, P.~Nason, C.~Oleari and E. Re, {\em
    A general framework for implementing NLO calculations in shower
    Monte Carlo programs: the POWHEG BOX}, JHEP {\bf 1006} (2010)
  043.  [arXiv:1002.2581 [hep-ph]].

\end{thebibliography}
%%%%%%%%%%%%%%%%%%
\end{document}
