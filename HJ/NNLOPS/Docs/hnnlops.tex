\documentclass[11pt,a4paper]{article}
\pdfoutput=1 % if your are submitting a pdflatex (i.e. if you have
             % images in pdf, png or jpg format)

\usepackage{jheppub}

\usepackage[T1]{fontenc}

\usepackage[latin9]{inputenc}
\usepackage{esint}


\newcommand{\noun}[1]{{\tt #1}}
\newcommand{\LHAPDF}{\noun{LHAPDF}}
\newcommand{\POWHEG}{\noun{POWHEG}}
\newcommand{\POWHEGBOX}{\noun{POWHEG BOX}}
\newcommand{\MINLO}{\noun{MiNLO}}
\newcommand{\HJMINLO}{\noun{HJ-MiNLO}}
\newcommand{\HNNLO}{\noun{hnnlo}}
\newcommand{\NNLOPS}{\noun{NNLOPS}}
\newcommand{\NLOPS}{\noun{NLOPS}}
\newcommand{\PYTHIA}{\noun{Pythia}}
\newcommand{\HERWIG}{\noun{Herwig}}
\newcommand{\MCatNLO}{\noun{MC@NLO}}
\newcommand{\ggH}{\noun{H}}
\newcommand{\HQT}{\noun{HqT}}
\newcommand{\HJ}{\noun{HJ}}
\newcommand{\MEPS}{\noun{Meps}}
\newcommand{\JETVHETO}{\noun{JetVHeto}}
\newcommand{\FASTJET}{\noun{FastJet}}
\newcommand{\as}{\alpha_{\scriptscriptstyle \mathrm{S}}}
\newcommand{\Kr}{K_{\scriptscriptstyle \mathrm{R}}}
\newcommand{\Kf}{K_{\scriptscriptstyle \mathrm{F}}}
\newcommand{\mur}{\mu_{\scriptscriptstyle \mathrm{R}}}
\newcommand{\muf}{\mu_{\scriptscriptstyle \mathrm{F}}}
\newcommand{\pt}{p_{\scriptscriptstyle \mathrm{T}}}
\newcommand{\pth}{p_{\scriptscriptstyle \mathrm{T}}^{\scriptscriptstyle \mathrm{H}}}
\newcommand{\ptjone}{p_{\scriptscriptstyle \mathrm{T}}^{\scriptscriptstyle \mathrm{j_{1}}}}
\newcommand{\kt}{k_{\scriptscriptstyle \mathrm{T}}}
\newcommand{\mh}{m_{\scriptscriptstyle \mathrm{H}}}
\newcommand{\mH}{mh}

\newcommand{\ptjo}{\pt^{{\scriptscriptstyle \mathrm{j}_{1}}}}
\newcommand{\hc}{\beta}
\newcommand{\hgam}{\gamma}
\newcommand{\smallk}{\kappa}
\newcommand{\comment}[1]{{\bf [#1] }}

\usepackage[mathscr]{euscript}

% continuation line arrows
\newcommand{\continueend}[0]{\raisebox{2.5mm}%
{\rotatebox{-90}{$\curvearrowright$}}}
\newcommand{\continuebeg}[0]{\raisebox{-0mm}%
{\rotatebox{90}{$\curvearrowleft$}}}



\title{{NNLOPS simulation of Higgs boson production}}

% more complex case: 4 authors, 3 institutions, 2 footnotes
\author[a]{Keith Hamilton}
\author[b]{Paolo Nason}
\author[c]{Emanuele Re}
\author[c]{Giulia Zanderighi}

% The "\note" macro will give a warning: "Ignoring empty anchor..."
% you can safely ignore it.

\affiliation[a]{Department of Physics and Astronomy, University College London,\\London, WC1E 6BT, UK}
\affiliation[b]{INFN, Sezione di Milano Bicocca,\\Piazza della Scienza 3, 20126 Milan, Italy}
\affiliation[c]{Rudolf Peierls Centre for Theoretical Physics, University of Oxford\\1 Keble Road, UK}

% e-mail addresses: one for each author, in the same order as the authors
\emailAdd{keith.hamilton@ucl.ac.uk}
\emailAdd{paolo.nason@mib.infn.it}
\emailAdd{e.re1@physics.ox.ac.uk}
\emailAdd{g.zanderighi1@physics.ox.ac.uk}

\abstract{We describe the usage
  of the generator, developed in ref.~\cite{Hamilton:2013fea},
  for gluon fusion Higgs production interfaced to a shower Monte Carlo
  that achieves next-to-next to leading order accuracy in QCD.}

%\makeatother

\begin{document}
\maketitle
\flushbottom

\section{Introduction}
Here we give the instructions to use the code developed in
ref.~\cite{Hamilton:2013fea} for the construction of a Monte
Carlo showered sample for gluon fusion Higgs production that
is accurate at the next-to-next to leading order (NNLO)
in QCD. The method makes use of the \HNNLO{} code
of refs.~\cite{Catani:2007vq,Grazzini:2008tf} and of the
\POWHEG{} \HJ{} generator~\cite{Campbell:2012am} improved with the \MINLO{}
procedure according to refs.~\cite{Hamilton:2012np,Hamilton:2012rf}.

The \MINLO{} improved \HJ{} generator (\HJMINLO{}
from now on) as it is set out in \cite{Hamilton:2012rf},
is crucial for our NNLOPS implementation. In fact, it reaches
NLO accuracy not only for $H+{\rm 1jet}$ observables, but
also for inclusive ones. Furthermore, it does this in a seamless
way, without introducing unphysical cuts in order to separate
the 0 and 1-jet final states. This feature allows to reach
NNLO accuracy by a simple reweighting procedure, making
use of a one parameter distribution computed with the
\HNNLO{} code.

These instructions explain how to use the relevant code\footnote{The
code can be found in the {\tt HNNLOPS} subdirectory of the {\tt HJ}
process in the {\tt User-Processes-V2} directory. Instructions for
downloading the code can be found in \url{http://powhegbox.mib.infn.it}.}
in conjunction with the \HNNLO{} and \HJ{} programs to
generate \NNLOPS{} accurate samples.

\section{Input: NNLO rapidity spectrum\label{sec:NNLO-ingredients}}
The fundamental NNLO ingredient required to obtain NNLO accurate
event samples is, for gluon fusion Higgs production, the Higgs
boson's rapidity spectrum.\footnote{In general, for \NNLOPS{} simulations
one requires an NNLO accurate distribution for some Born kinematics.
In this case the Born kinematics is fully specified in terms of the
rapidity of the boson.} We here give instructions on how to obtain
such distributions, suitable for combination with the \HJMINLO{}
events via the {\tt{HJ/NNLOPS}} reweighting code.

\begin{enumerate}
\item Make sure that the \LHAPDF{} package is installed:\\
\hspace*{5mm}
{\url{https://lhapdf.hepforge.org}}\\
The current version of this code has only been tested with up to
version 5.8.9. In particular, for the installation of \HNNLO{},
the command {\tt{lhapdf-config ---libdir}} should return the 
location of the installed {\noun{Lhapdf}} libraries.

\item Download \HNNLO{} from the following {\tt url}:\\
\hspace*{5mm}
{\url{http://theory.fi.infn.it/grazzini/codes/hnnlo-v1.3.tgz}}

\item Unpack the tarball in a convenient location\\
\hspace*{5mm}
{\tt{work\$ cp ~/Downloads/hnnlo-v1.3.tgz ./}}\\
\hspace*{5mm}
{\tt{work\$ tar -xzvf hnnlo-v1.3.tgz}}\\
\hspace*{5mm}
{\tt{work\$ ls hnnlo-v1.3}}\\
Under the parent directory {\tt{hnnlo-v1.3}} one should find subdirectories
{\tt{bin}}, {\tt{doc}}, {\tt{obj}}, {\tt{src}}, and a {\tt{makefile}}.

\item Enter the \HNNLO{} parent directory\\
\hspace*{5mm}
{\tt{work\$ cd hnnlo-v1.3}}

\item  Replace the \HNNLO{} default {\tt{makefile}} with the one
from the {\tt{HJ/NNLOPS}} directory\\
\hspace*{5mm}
{\tt{work\$ cp /path/to/HJ/NNLOPS/HNNLO-makefile ./makefile}}

\item  Copy the \HNNLO{} patches directory into the parent directory\\
\hspace*{5mm}
{\tt{work\$ cp -r /path/to/HJ/NNLOPS/HNNLO-patches ./}}

\item  Build the code\\
\hspace*{5mm}{\tt{work\$ make}}\\
A message {\tt{----> HNNLO compiled with LHAPDF routines <----}}
indicates success.

\item  Copy over the template input file and, if parallel runs are required,
the related script to set up multiple runs\\
\hspace*{5mm}{\tt{work\$ cd bin}}\\
\hspace*{5mm}{\tt{work\$ cp /path/to/HJ/NNLOPS/HNNLO.input ./}}\\
\hspace*{5mm}{\tt{work\$ cp /path/to/HJ/NNLOPS/HNNLO-mur-muf-scan.sh ./}}\\

\item  Either edit the input file or the script for parallel runs as desired.
The former is nothing but a standard \HNNLO{} input file and may be
run simply by typing\\
\hspace*{5mm}{\tt{work\$ ./hnnlo < HNNLO.input >\,> my.log}}\\ 
yielding a {\tt{hnnlo.top}} histogram file.
\vspace{3mm}\\
Alternatively one can run the {\tt{HNNLO-mur-muf-scan.sh}} script
in the same directory. The script generates \HNNLO{} input files
and shell scripts for the generation of runs for a number of different
rescaling factors for $\mur$ and
$\muf$, and, moreover, a number of sequentially ordered seeds (specified
by the {\tt{njobs}} variable) for each $\mur$, $\muf$ value.
Runs different only by the value of the random seed may be combined to increase
the statistical precision. Statistically equivalent histogram files may be
combined with the mergedata program:\\
\hspace*{5mm}{\tt{work\$ cp /path/to/HJ/NNLOPS/mergedata.f ./}}\\
\hspace*{5mm}{\tt{work\$ gfortran -o mergedata -c mergedata.f ; chmod +x mergedata}}\\
\hspace*{5mm}{\tt{work\$ ./mergedata 1 file\_1.top ... file\_N.top}}\\
\hspace*{5mm}{\tt{work\$ mv fort.12 hnnlo.top}}\vspace{3mm}\\
%
Lastly, we draw attention to the fact that the default
{\tt{HNNLO.input}} file supplied sets
$\mur=\muf={\frac{1}{2}}m{_{\scriptscriptstyle{H}}}$ as the central scale
choices about which the scale variations are carried out by the factors
mentioned earlier.

In order to produce the \HNNLO{} output for ref.~\cite{Hamilton:2013fea}, for each scale choice, 
we used  120M integration points for the grid preparation, 
and about 150-200 runs with 
240M points each for the integration stage. 

\item  Finally, for the purposes of setting the \HJMINLO{} input parameters
  consistently with those used for \HNNLO{}, make a note of the Higgs boson
  width calculated by \HNNLO{} in its initialisation phase. This will appear
  at the beginning of the \HNNLO{} on-screen output, as follows
\vspace{3mm}\\
{\tt{CCCCCCCCCCCCCCC SM Higgs parameters CCCCCCCCCCCCCCCC}}\\
{\tt{C \ \ \ \ \ \ \ \ \ \ \ \ \ \ \ \ \ \ \ \ \ \ \ \ \ \ \ \ \ \ \ \ \ \ \ \ \ \ \ \ \ \ \ \ \ \ \ \ \ C}}\\
{\tt{C   Mh      =  125.50 GeV\ \ \ \ \ \ \ \ \ \ \ \ \ \ \ \ \ \ \ \ \ \ \ \ \ \ \ \ \ \ \ \ \ \ C}}\\
{\tt{C   Gamma(H)= 0.004221 GeV\ \ \ \ \ \ \ \ \ \ \ \ \ \ \ \ \ \ \ \ \ \ \ \ \ \ \ C}}\\
{\tt{C \ \ \ \ \ \ \ \ \ \ \ \ \ \ \ \ \ \ \ \ \ \ \ \ \ \ \ \ \ \ \ \ \ \ \ \ \ \ \ \ \ \ \ \ \ \ \ \ \ C}}\\

\end{enumerate}

\noindent We emphasise that \HNNLO{} and \HJMINLO{} must be run with
the same values of the Higgs mass and decay width, and with the same
PDFs (and $\alpha_s$).

\section{Input: {\HJMINLO{}} events\label{sub:HJ-MINLO-ingredients}}
With the NNLO input in hand the other fundamental ingredient needed for the
NNLO reweighting procedure are the \HJMINLO{} Les Houches event files.
Strictly speaking, the \HJMINLO{} code does not require any
modification for this purpose. Moreover the running of the program proceeds
in the usual way. The important (albeit obvious) point is to maintain
consistency of the physical parameters in \HNNLO{} and \HJMINLO{}. To this
end we describe, below, all relevant details involved in this.

\begin{enumerate}
\item  The denominator in the reweighting factor is computed from the
       generated Les Houches sample. Thus, for the reweighting to make
       sense, there should be enough events in the Les Houches file.
       As a guideline, we suggest to use no less than 500K events.
\item  A large number of physical parameters used by the \HJMINLO{} code,
  such as the Fermi constant, are assigned by the subroutine
  {\tt{init\_couplings}}
  defined in file\linebreak {\tt{HJ/init\_couplings.f}}. As in the case of
  the \HNNLO{} {\tt{mdata.f}} file, some of the parameters in this file may
  be irrelevant for gluon fusion Higgs production. 
  Nevertheless, in {\tt{HJ/NNLOPS/init\_couplings.f}} we provide
  a version of this routine edited for maximum consistency with the physical
  parameters in the {\tt{mdata.f}} file we supply for \HNNLO{}. We therefore
  recommend using these inputs in place of the \POWHEGBOX{} defaults. This
  can be done, for instance, by overwriting the latter and (re-)compiling the \HJMINLO{}
  code:
  \vspace{3mm}\\
  \hspace*{5mm}{\tt{work\$ cd /path/to/HJ/}}\\
  \hspace*{5mm}{\tt{work\$ cp NNLOPS/init\_couplings.f ./}}\\
  \hspace*{5mm}{\tt{work\$ make}}

\item  The main physical parameters to ensure consistency with respect to
  the \HNNLO{} runs are set in the {\tt{powheg.input}} file under
  {\tt{HJ/testrun-lhc/}},
  or, if running parallel \HJMINLO{} jobs, in {\tt{powheg.input-save}}
  under {\tt{HJ/testparallel-lhc/}}. In particular, the Higgs boson mass
  should be set according to the value input to \HNNLO{}, and the
  Higgs boson width should be set to the
  on-screen value computed by \HNNLO{}.
  For a Higgs mass of 125.5 GeV for example we set in
  {\tt{powheg.input}}  or {\tt{powheg.input-save}}:\vspace{3mm}\\
  {\tt{hmass 125.5}}\\
  {\tt{hwidth 0.004221D0}}\\
  {\tt{bwcutoff 1000}}\vspace{3mm}\\
  where {\tt{bwcutoff 1000}} determines that the code generate off-shell
  Higgs masses, ${m_{\scriptscriptstyle{\mathrm{H}}}^{*}}$, in the range
  ${m_{\scriptscriptstyle{\mathrm{H}}}}-1000\,{\Gamma_{\scriptscriptstyle{\mathrm{H}}}}<{m_{\scriptscriptstyle{\mathrm{H}}}^{*}}<{m_{\scriptscriptstyle{\mathrm{H}}}}+1000\,{\Gamma_{\scriptscriptstyle{\mathrm{H}}}}$. In addition we require that the Higgs
  particle is left undecayed at the Les Houches event level\vspace{3mm}\\
  {\tt{hdecaymode -1}}\vspace{3mm}\\
  A further useful (although not mandatory) technical input, involves
  suppressing the generation of low transverse momentum events in order
  to better  populate high ${p_{\scriptscriptstyle{\mathrm{T}}}}$ tails. This can
  be achieved by setting e.g.\vspace{3mm}\\
  {\tt{bornsuppfact 10}}\vspace{3mm}\\
  Further explanation of the {\tt{bornsuppfact}} parameter and other workings
  of the \HJMINLO{} program can be found in
  {\tt{/path/to/HJ/Docs/manual-BOX-HJ.pdf}}. Also, for detailed instructions
  on setting up the \HJMINLO{} program to perform numerous runs in parallel
  see section 4.1 in
  {\tt{/path/to/W2jet/Docs/manual-BOX-WZ2jet.pdf}}\vspace{3mm}\\
  Lastly, we add that although
  it is not strictly necessary to generate the \HJMINLO{} events using a
  NNLO PDF, in the limited studies that we have carried out to date with
  the \NNLOPS{} code, we have found good agreement with higher order 
  resummation programs using the MSTW 2008 NNLO set \cite{Martin:2009iq}.
  This can be selected by setting \LHAPDF{} indices\vspace{3mm}
  {\tt{lhans1   21200}} and
  {\tt{lhans2   21200}} 
  in {\tt{powheg.input}} or {\tt{powheg.input-save}}.


\end{enumerate}

\noindent Again, we emphasise that ultimately the responsibility for ensuring
the consistent running of \HNNLO{} and \HJMINLO{}, to obtain inputs
for reweighting procedure, rests with the user.

\section{Reweighting\label{sub:Reweighting}}

The NLO-to-NNLO weight factor assigned to the \HJMINLO{} events is
\begin{equation}
\mathcal{W}\left(y,\, \pt\right)=h\left(\pt\right)\,\frac{\smallint d\sigma^{{\scriptscriptstyle \mathrm{NNLO\phantom{i}}}}\,\delta\left(y-y\left(\Phi\right)\right)-\smallint d\sigma_{B}^{{\scriptscriptstyle \mathrm{MINLO}}}\,\delta\left(y-y\left(\Phi\right)\right)}{\smallint d\sigma_{A}^{{\scriptscriptstyle \mathrm{MINLO}}}\,\delta\left(y-y\left(\Phi\right)\right)}+\left(1-h\left(\pt\right)\right)\,,\label{eq:NNLOPS-overall-rwgt-factor-1}
\end{equation}
where $y$ is the Higgs boson rapidity and $\pt$ is the transverse momentum
of the leading jet in the LHE events, and
\begin{eqnarray}
d\sigma_{A} & = & d\sigma\, h\left(\pt\right)\,,\label{eq:NNLOPS-dsig0}\\
d\sigma_{B} & = & d\sigma\,\left(1-h\left(\pt\right)\right)\,,\label{eq:NNLOPS-dsig1}
\end{eqnarray}
with $h$ a monotonic profile function 
\begin{equation}
h(\pt)=\frac{(\hc\, m_{{\scriptscriptstyle \mathrm{H}}})^{2}}{(\hc\, m_{{\scriptscriptstyle \mathrm{H}}})^{2}+\pt^{2}},\label{eq:NNLOPS-hpT-fn-defn}
\end{equation}
and $\hc$ a constant parameter. 
On convoluting $\mathcal{W}\left(y,\, \pt\right)$ with the \HJMINLO{} differential cross section
and integrating over $\pt$ one finds, exactly,
\begin{eqnarray}
\left(\frac{d\sigma}{dy}\right)^{{\scriptscriptstyle \mathrm{NNLOPS}}} & = & \left(\frac{d\sigma}{dy}\right)^{{\scriptscriptstyle \mathrm{NNLO}}}\,.\label{eq:NNLOPS-NNLOPS-eq-NNLO_0+MINLO_1-1}
\end{eqnarray}
For a proof of why such a reweighting procedure leads to NNLO accuracy in
general, and not simply the inclusive rapidity spectrum, we refer the
reader to Sect.~2.1 of ref.~\cite{Hamilton:2013fea}. 

The role of the profile function $h$ and, in particular, the $\hc$
parameter within it, is, roughly speaking, to determine how to spread out
the NLO-to-NNLO corrections along the $\pt$ axis. For
$\hc=\infty$ the corrections are spread uniformly in $\pt$ (see e.g.
fig.~3 of ref.~\cite{Hamilton:2013fea}), while for $\hc={\frac{1}{2}}$
they are concentrated in the region $\pt<\hc\,\mh$. In the latter
respect the $\hc$ parameter plays a similar role to the resummation
scale in dedicated resummation calculations, and as such we favour 
that $\hc$ be set consistently with the preferred values in those.
Thus we recommend $\hc={\frac{1}{2}}$ in carrying out the reweighting.
Indeed for $\hc={\frac{1}{2}}$ we find good agreement with dedicated
NNLL+NNLO calculations of the Higgs boson transverse momentum and the
0-jet veto efficiency wherein the resummation scale was set to
${\frac{1}{2}}\mh$ (see figs.~5 and 7 of ref.~\cite{Hamilton:2013fea}).

We should emphasise that, while our NNLOPS simulation is formally NNLO
accurate for inclusive quantities, the accuracy of its resummation of
all-orders large logarithms is not at the
NNLL+NNLO level. We recommend that the predictions of such calculations
be used to `tune' the \NNLOPS{} simulation (in particular the $\hc$
parameter) to approximate the yet higher order, large logarithmic terms
which are not properly accounted for. Setting $\hc={\frac{1}{2}}$ appears
to do a very good job in this respect, thus it is our default, recommended
value.

\subsection{A simple run\label{sec:simple-reweighting}} 

The code to be used in order to produce an NNLO-reweighted event file
can be found in the {\tt{NNLOPS/Reweighter/}} directory.
This program takes as input the \HNNLO{} output files (i.e. the
files with the {\tt .top} suffix) and the \HJMINLO{} LHE file, and
produces as output a new LHE file, where after each event new weights
are added, corresponding to the NNLO reweighted events. The new weights
appear, before the {\tt <\textbackslash{}event>} line,
in lines of the form
\begin{verbatim}
#new weight,renfact,facfact,pdf1,pdf2 <weight> <KR> <KF> <npdf1> <npdf2> \ 
       <pdftag>  <HNNLO-identifier>
\end{verbatim}
where  {\tt <weight>} is the weight of the event, {\tt <KR>} and {\tt <KF>}
are the renormalization and factorization scale factors, {\tt <npdf1>} and
{\tt  <npdf2>} are the pdf set number for the negative and positive rapidity
 incoming hadrons, {\tt <pdftag>} is either {\tt mlm}
if the internal PDF package of the \POWHEGBOX{} is used, or {\tt lha} if
\LHAPDF{} is used instead.
Finally  {\tt <HNNLO-identifier>} is an identifier of no more than six characters,
provided by the user when the run is performed.

The run is carried out in the following way. In the {\tt{NNLOPS/Reweighter/}}
directory one enters
\begin{verbatim}
$ make minnlo
$ ./minnlo <HJMiNLO-eventfile> <HNNLO-identifier1> <HNNLO-outputfile1> \
          [<HNNLO-identifier2> <HNNLO-outputfile2> ... ]}}\vspace{3mm}
\end{verbatim}
where {\tt <HJMiNLO-eventfile[1...]>} is a LH file (suffix {\tt .lhe})
containing events produced with \HJMINLO{}, obtained as described in
sec.~\ref{sub:HJ-MINLO-ingredients}, whereas {\tt <HNNLO-outputfile[1 ...]>}
are the final histogram file (suffix {\tt .top}) produced from the \HNNLO{}
run(s), as described in sec~\ref{sec:NNLO-ingredients}.
It is good practice to use for the {\tt HNNLO-identifier}
a name related to the particular \HNNLO{} settings being used when
generating the corresponding {\tt <HNNLO-outputfile[1 ...]>} file (i.e. scales,
parton distributions, etc.), as
it will be more clear in sec.~\ref{sec:Estimating-uncertainties}.
% The default profile function used to perform the
%NLO-to-NNLO reweighting is computed internally from the inputs
%given to the {\tt minnlo} executable, and it corresponds to the
%recommended setting described at the end of
%sec.~\ref{sub:Reweighting}.\footnote{To select a different profile
%  function to perform the NLO-to-NNLO reweighting, {\tt minnlo} should
%  be called with an optional (integer) argument ({\tt [<index
%    (1:13)>]}), after the last \HNNLO{} histogram file. This option
%  should be used only by developers.}  
At the end of the program execution, a new LH file will be present in
the run directory, named as the {\tt HJMiNLO-eventfile} event file,
but with a suffix {\tt .nnlo}. 
%Notice in particular that the jet
%radius $R$ has to match the value used with \HNNLO{} (which we
%recommend to be 0.7). 
This LH file is the final output of the {\tt
  minnlo} NLO-to-NNLO reweighter program, and can now be read and
showered by \PYTHIA{} or \HERWIG{}, as it is usually done with LH
event files generated by \POWHEGBOX{}.


The program takes few minutes to reweight a LH file produced by
\HJMINLO{} containing 500000 events. The output printed on the
terminal at run time is self explanatory.

%We have also included a
%template script ({\tt runminnlo\_template.sh}) to help the user.


\subsection{Estimating uncertainties\label{sec:Estimating-uncertainties}}

The conservative Ansatz we recommend in estimating errors (the one
employed in ref.~\cite{Hamilton:2013fea}) is that the $\mur$ and
$\muf$ scales in the NNLO and NLO inputs should be varied in a fully
independent way. In doing so we regard the uncertainties in the
normalizations of distributions, e.g. the transverse momentum spectrum
of the Higgs boson, as being independent of the respective
uncertainties in the shapes --- at least in the low $\pt$ region
covered by the profile function, $h(\pt)$.  The former are determined
by the \HNNLO{} program, while the latter are due to
\HJMINLO{}. Outside of the low $\pt$ region, in the part corresponding
to the $1-h(\pt)$ term in eq.~(\ref{eq:NNLOPS-overall-rwgt-factor-1}),
the uncertainty is given by the standard \HJMINLO{} computation (which
there corresponds to that of conventional NLO with $\mur=\muf=\pt$ for
the central scale choice).

In order to compute an uncertainty band, one first needs to have
obtained multiple outputs from \HNNLO{} and \HJMINLO{} varying $\mur$
and $\muf$. For the sake of simplicity, we will now describe a case
where $\mur$ and $\muf$ are kept equal both when running \HNNLO{} and
\HJMINLO{}. We call this situation a ``3x3 pts'' scale variation
study: for each event we will obtain 9 NNLO weights, associated to
each of the 9 possible combinations among 3 results from \HNNLO{} and
3 from \HJMINLO{}.  This procedure is general enough to be
straightforwardly adapted to the more general case of a ``7x7 pts''
scale variation, or variation thereof.

In the ``3x3 pts'' case, the needed inputs are 3 histogram files from
\HNNLO{}, which we will call {\tt HNNLO-outputfile-QQ.top}, {\tt
  HNNLO-outputfile-HH.top} and {\tt HNNLO-outputfile-11.top}, for
values of $\mur=\muf=\{\mh/4,\mh/2,\mh\}$ respectively.  Similarly,
the user needs to have obtained a LH file from \HJMINLO{} where 3
weights are associated to each event. This file has to be obtained
with the \POWHEGBOX{} reweighting machinery: the 3 lines in between
the last line of each event record and the {\tt </event>} tag should
have the format
\begin{verbatim}
#new weight,renfact,facfact,pdf1,pdf2 <weight> <KR> \ 
    <KF> <npdf1> <npdf2> <pdftag>
\end{verbatim}
%
The NLO-to-NNLO reweighter program should now be invoked as follows:
\begin{verbatim}
$ ./minnlo HJMiNLO-events.lhe HH HNNLO-outputfile-HH.top \
        QQ HNNLO-outputfile-QQ.top 11  HNNLO-outputfile-11.top
\end{verbatim}
and in this case it is reasonable to use {\tt HH, QQ, 11} as
``identifier'' strings for the 3 \HNNLO{} histogram files.  At the end
of the run, 9 lines will be present after each event record, each one
containing the NNLO weight associated to the \HJMINLO{} result
labelled by the values of the pair ({\tt <KR>}, {\tt
  <KF>}) and the \HNNLO{} result identified by the {\tt
  HNNLO-identifier} appended at the end of each line.

\section{Example analysis\label{sec:Analysis}}

The analysis used for the study in
ref.~\cite{Hamilton:2013fea} can be found in
{\tt{pwhg\_analysis-pheno\_2.f}}. Either this analysis, or the {\tt pwhg\_analysis-release.f} should be used to 
produce the NNLO-reweighted LH file (the former is extension of the latter to include physical distributions of interest).
%
More interestingly, it can also be used to analyse events after the
showering stage. In this case, in the {\tt Makefile} under the {\tt
  HJ} directory, in the section where the analysis files are defined,
the following line should be used:
\vspace{3mm}\\
\hspace*{5mm}{\tt{PWHGANAL=pwhg\_bookhist-multi.o
    pwhg\_analysis-pheno\_2.o
    fastjetfortran.o \\
    \hspace*{15mm} genclust\_kt.o miscclust.o ptyrap.o r.o swapjet.o
    jet\_finder.o  \\
    \hspace*{15mm} auxiliary.o get\_hdamp.o}}\vspace{3mm}\\
%
As usual, \FASTJET{} should be linked too. 

In order to perform showering and analysis of the \NNLOPS{} LHE files,
one needs to be able to use the information in the {\tt \#new weight}
lines. This information can easily be converted to the standard Les
Houches format specified in
\url{http://phystev.in2p3.fr/wiki/2013:groups:tools_lheextension}. At
the moment we do not provide script executable in order to perform
this conversion. These can be provided upon request.

\section{Suggested citations}
If you use this code please cite the inclusive Higgs boson production
\NNLOPS{} paper \cite{Hamilton:2013fea} and the \HJMINLO{} paper which
precedes and lays much of the foundations for it \cite{Hamilton:2012rf}.
The \NNLOPS{} simulation fundamentally relies on the NNLO Higgs boson
production calculation of refs.~\cite{Catani:2007vq,Grazzini:2008tf},
and so these works should also be cited.


\appendix


\section{Code description}
In this section we briefly record the various additions and modifications
to existing \HNNLO{} and \POWHEGBOX{} code used to produce \NNLOPS{} events.

\subsection{\HNNLO\label{sec:HNNLO-code}}
\begin{itemize}

\item {\tt{makefile}} ({\tt{HNNLO-makefile}})
  \begin{itemize}
  \item The \HNNLO{} {\tt{makefile}} is modified by
    prepending {\tt{\$(HNNLOHOME)/HNNLO-patches}} to
    the {\tt{\$DIRS}} variable, introducing a variable {\tt{\$PATCHES}}
    equal to the concatenation of the following object files in this list,
    plus the removal of those elements from the other {\tt{Makefile}}
    variables (avoiding duplication). In this way the modified \HNNLO{}
    files below are compiled and linked from {\tt{HNNLO-patches}} instead
    of the default versions in the default locations.
  \end{itemize}

\item {\tt{mbook.f}}
  \begin{itemize}
  \item The {\tt{mtop}} subroutine, which outputs the \HNNLO{} histograms
    to a text file, has undergone a minor modification so as to have the
    same format as the \HJMINLO{} histogram output, to ease comparisons
    and for use in the reweighting program.
  \end{itemize}

\item {\tt{mdata.f}}
  \begin{itemize}
  \item This file contains the values of numerous physical constants in
    \HNNLO{} e.g. the Fermi's constant $G_{F}$. The great majority of these
    constants should be irrelevant in the context of gluon fusion Higgs
    boson production. Moreover, any inconsistencies between these and
    the corresponding values used in the \HJMINLO{} program should result
    in negligible differences. The main parameters to 
    seek consistency between \HJMINLO{} and \HNNLO{} codes are the Fermi's
    constant $G_{F}$ and the top mass. Nevertheless, we have edited all
    parameters in this file to maximize agreement with the corresponding
    \HJMINLO{} settings.
  \end{itemize}

\item {\tt{plotter.f}}
  \begin{itemize}
  \item This file contains an example analysis for \HNNLO{} by
    default.  We have modified this analysis substantially to produce
    histograms of the Higgs boson's rapidity distribution inclusively
    and also in the presence of a number of different profile
    functions. All but the first of these histograms, the inclusive
    Higgs boson rapidity, are not used in our default, recommended,
    reweighting procedure. The inclusive rapidity spectrum is the
    first histogram appearing in the output {\tt{.top}} file and is
    titled simply {\tt{`yh'}}. The histogram range has been set to
    $-5<y<5$ and the bin width is 0.1. These values may be altered by
    the user as desired, by editing the relevant {\tt{bookplot}}
    subroutine call. However, in this case one must take care to
    modify the relevant \POWHEGBOX{} analysis file used by the
    reweighting code, under the {\tt{HJ/NNLOPS}} directory, in the
    same way (either {\tt{pwhg\_analysis-release.f}} or
    {\tt{pwhg\_analysis-pheno\_2.f}}).
  \end{itemize}

\item {\tt{setup.f}}
  \begin{itemize}
  \item This file was modified to set \HNNLO{} to fill histograms with the
    relevant Higgs decay branching fractions divided out:
    {\tt{removebr=.true.}}. Thus the integral of the Higgs boson rapidity
    spectrum histogram output is the total cross section. This is required
    for consistency with the \HJMINLO{} program and its analysis, in order
    to carry out the NNLO reweighting.
  \end{itemize}

\item {\tt{writeinfo.f}}
  \begin{itemize}
  \item Originally the {\tt{writeinfo}} subroutine in this file copied
    the contents of the input file used in running the program, plus the
    cross section, as a series of comments to the top of the histogram
    output file. In order to have a simple uniform format for \HJMINLO{}
    and \HNNLO{} we removed these comments (the bulk of which was simply
    a copy of the input file used to run the program).
  \end{itemize}

\end{itemize}

\subsection{\HJMINLO{}\label{sec:HJMINLO}}

\begin{itemize}

\item {\tt{init\_couplings.f}}
  \begin{itemize}
  \item This file contains the values of numerous physical constants in
    \HJMINLO{}.
  \end{itemize}

\item {\tt{powheg.input}}/{\tt{powheg.input-save}}
  \begin{itemize}
  \item Here the Higgs boson mass and
    width should be set consistently with the values used by \HNNLO{} in
    the \HJMINLO{} input file ({\tt{hmass}} and {\tt{hwidth}} parameters).
    Also, the mass window on the off-shell mass range for the Higgs boson
    (the Breit-Wigner generation cut-off) should be set to a large value
    (in ref.~\cite{Hamilton:2013fea} we have set {\tt{bwcutoff 1000}}).
    Lastly, again for consistency with the running of \HNNLO{} we require
    that the Higgs boson be left undecayed at the Les Houches event level,
    this is achieved by setting {\tt{hdecaymode -1}} in the relevant
    \HJMINLO{} input file.
  \end{itemize}

\end{itemize}

\subsection{NNLO reweighting\label{sec:HNNLO code}}

\begin{itemize}
  
\item {\tt{Makefile}}
  \begin{itemize}
  \item The default value for the {\tt ANALYSIS} variable should be
    {\tt release}. Other options are described below, but they are
    intended mainly for debugging purposes or developers. \FASTJET{}
    must be linked properly too, and as usual it is recommended to let
    the {\tt Makefile} call the {\tt fastjet-config} command.
  \end{itemize}
  
\item {\tt{minnlo.f}}
  \begin{itemize}
  \item This file contains the main program to perform the NLO-to-NNLO
    reweighting. Some parameters useful for debugging purposes can be
    found here, as described in the commented section at the beginning
    of the file. However, the user is recommended not to change them.
  \end{itemize}
  
\item {{\tt opencount.f}, {\tt auxiliary.f}, {\tt lhef\_readwrite.f},
    {\tt get\_hdamp.f}, {\tt genclust\_kt.f}, \\ 
   {\tt swapjet.f},
    {\tt miscclust.f}, {\tt ptyrap.f}, {\tt r.f}}
  \begin{itemize}
  \item These files contain several functions and routines needed by
    {\tt minnlo.f} and/or by the analysis subroutine used to process
    the \HJMINLO{} LH file and compute
    $d\sigma_{A/B}^{{\scriptscriptstyle
        \mathrm{MINLO}}}\,\delta\left(y-y\left(\Phi\right)\right)$.
  
  ({\tt jetlabel.f}, {\tt jetcuts.f}, {\tt mxpart.f} and {\tt npart.f}
  contain common blocks used in the source files.)
\end{itemize}

\item {{\tt pwhg\_analysis-release.f}, {\tt jet\_finder-release.f}}
  \begin{itemize}
  \item These files contain the minimal analysis needed to extract
    $d\sigma_{A/B}^{{\scriptscriptstyle
        \mathrm{MINLO}}}\,\delta\left(y-y\left(\Phi\right)\right)$
  from the LH input file. They are compiled if {\tt ANALYSIS=release}
  is set, which is the default option.
\end{itemize}

\item {{\tt pwhg\_analysis-pheno\_2.f}, {\tt jet\_finder.f}}
  \begin{itemize}
  \item These files are compiled when {\tt ANALYSIS=pheno\_2}.
    Although they extend the '{\tt release}' analysis, allowing to
    plot more distributions, the LH file obtained by running the {\tt
      minnlo} reweighter is identical to that obtained with the
    previously-described files. The '{\tt pheno\_2}' analysis was used
    to produce the plots in ref.~\cite{Hamilton:2013fea}.
  \end{itemize}
  
\item {{\tt pwhg\_analysis-pheno.f}, {\tt pwhg\_analysis-HNNLO.f}}
  \begin{itemize}
  \item These files are kept for backward-compatibility reasons.
  \end{itemize}
  
\end{itemize}

\bibliographystyle{JHEP}
\bibliography{hnnlops}


\end{document}
