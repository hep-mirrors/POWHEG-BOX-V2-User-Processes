\documentclass[paper]{JHEP3}
\usepackage{amsmath,amssymb,enumerate,url}
\bibliographystyle{JHEP}

%%%%%%%%%% Start TeXmacs macros
\newcommand{\tmtextit}[1]{{\itshape{#1}}}
\newcommand{\tmtexttt}[1]{{\ttfamily{#1}}}
\newenvironment{enumeratenumeric}{\begin{enumerate}[1.] }{\end{enumerate}}
\newcommand\sss{\mathchoice%
{\displaystyle}%
{\scriptstyle}%
{\scriptscriptstyle}%
{\scriptscriptstyle}%
}

\newcommand\as{\alpha_{\sss\rm S}}
\newcommand\POWHEG{{\tt POWHEG}}
\newcommand\POWHEGBOX{{\tt POWHEG BOX}}
\newcommand\POWHEGBOXV{{\tt POWHEG-BOX-V2}}
\newcommand\HERWIG{{\tt HERWIG}}
\newcommand\PYTHIA{{\tt PYTHIA}}
\newcommand\MCatNLO{{\tt MC@NLO}}
\newcommand\Wbbj{{\tt Wbbj}}

\newcommand\kt{k_{\sss\rm T}}

\newcommand\pt{p_{\sss\rm T}}
\newcommand\LambdaQCD{\Lambda_{\scriptscriptstyle QCD}}
%%%%%%%%%% End TeXmacs macros


\title{The POWHEG BOX V2 user manual:\\
  $\boldsymbol{Wb\bar{b}j}$ production at NLO with POWHEG+MiNLO} \vfill
\author{Gionata Luisoni\\ 
  Max-Planck Institut f{\"u}r Physik, F\"ohringer 6, D-80805 Munich, Germany\\
  E-mail: \email{luisonig@mpp.mpg.de}
}

\author{Carlo Oleari\\
  Universit\`a di Milano-Bicocca and INFN, Sezione di Milano-Bicocca\\
  Piazza della Scienza 3, 20126 Milano, Italy\\
  E-mail: \email{carlo.oleari@mib.infn.it}}

\author{Francesco Tramontano\\
  Universit\`a di Napoli ``Federico II'' and INFN, Sezione di Napoli,\\
  Complesso di Monte Sant'Angelo, via Cintia, 80126 Napoli, Italy\\
  E-mail: \email{francesco.tramontano@na.infn.it}
}
\vskip -0.5truecm

\keywords{POWHEG, Shower Monte Carlo, NLO}

\abstract{This note documents the use of the package \POWHEGBOXV{} for the
  associated production of a $W$ boson plus a $b\bar{b}$ pair and a jet. All
  bottom-quark mass effects and spin correlations of the leptonic decay
  products of the $W$ boson have been fully taken into account.  Results can
  be easily interfaced to shower Monte Carlo programs, in such a way that
  both NLO and shower accuracy are maintained.}  \preprint{\today\\ {\tt
    POWHEG BOX V2} 1.0}

\begin{document}


\section{Introduction}

The \POWHEGBOX{} program is a framework for implementing NLO calculations in
shower Monte Carlo programs according to the \POWHEG{} method. An explanation
of the method and a discussion on how the code is organized can be found in
refs.~\cite{Nason:2004rx,Frixione:2007vw,Alioli:2010xd}.
The code is distributed according to the ``MCNET GUIDELINES for Event Generator Authors
and Users'' and can be found at the web page
%
\begin{center}
 \url{http://powhegbox.mib.infn.it}
\end{center}
%
In this manual, we describe the \POWHEG{} implementation of the associated
production of a $W$ boson plus a $b\bar{b}$ quark pair and a jet, as
described in ref.~\cite{Luisoni:2015mpa}. The relevant matrix elements have been
computed with {\tt MadGraph4}~\cite{Stelzer:1994ta,Alwall:2007st} and {\tt
  GoSam}~\cite{Cullen:2011ac,Cullen:2014yla} and embedded in the \POWHEG{}
framework in a fully automated way.

The code can be found in the \Wbbj{} directory under {\tt User-Processes-V2},
and can be used to generate NLO+PS QCD corrections to $Wb\bar{b}j$ production
at hadronic colliders.  The $W$ boson is allowed to decay only
leptonically.

This document describes the input parameters that are specific to this
implementation. The parameters that are common to all \POWHEGBOX{}
implementation can be found in the manual in the \POWHEGBOXV{}/{\tt Docs}
directory.


\section{Generation of events}

According to the \POWHEGBOXV{} manual, it is possible to split a \POWHEG{}
run into several parallel ones.  See \POWHEGBOXV{}/{\tt Docs/V2-paper.pdf}
for a detailed discussion. Here we briefly recall how it works.

A template input file ({\tt powheg.input-save}) can be found under {\tt
  Wbbj/testrun-lhc} directory. The following flags have to be activated:

\noindent
{\tt manyseeds 1 \qquad \quad \,\,\,! get the
  seeds for the random number generator from 
\\ \# \qquad \qquad \qquad
  \qquad \qquad pwgseeds.dat
\\ parallelstage <x> \,\,\,! <x>=1...4, which
  level of parallel stage
\\ xgriditeration <y> ! <y> is the iteration level
  for the calculation of the \# \qquad \qquad \qquad \qquad \qquad importance
  sampling grid improvement (relevant only 
\\ \# \qquad \qquad \qquad \qquad
  \qquad for parallelstage=1)}

\noindent
The file {\tt pwgseeds.dat} in the directory {\tt testrun-lhc}
contains a list of integer seeds, one per line. When the program starts, it
asks for an integer number $m$. This number is used to select a line in the
{\tt pwgseeds.dat} file, and the integer found there is used to
initialize the random number generator. In this way, several runs can be
performed simultaneously in the same directory, all using different random
number seeds. Furthermore, the integer $m$ appears as a four digit integer
(with leading zeros) in the name of all files that are generated by the
corresponding run.

The generation of events is done by the \POWHEGBOXV{} in four stages.  An
example of script file to drive the runs can be found in {\tt
  testrun-lhc/run.sh}.  The user has to set the number of cores she/he wants
to use ({\tt ncores}) and the number {\tt xgriditeration} for the first
stage. All the other parameters are set in the {\tt powheg.input-save} file,
that is copied into the {{\tt powheg.input} at each stage by the {\tt run.sh}
  script.

Other useful scripts can be found in the {\tt Wbbj} directory:
\begin{itemize}
\item {\tt runNLO.sh}: to combine the NLO partial results of the parallel
  runs into a unique file

\item {\tt runLHEF.sh}: to run the analysis in the {\tt
  pwhg\_analysis-Wbnj.f} on the LHE events

\item {\tt runpy.sh}: to run {\tt Pythia} on the event files

\item {\tt runrwgt.sh}: to run the reweighting procedure on the event files

\end{itemize}




\section{Process-specific input parameters}


\noindent The main mandatory process-specific parameters that have to be set are {\tt idvecbos} and {\tt vdecaymode}:\\
{\tt idvecbos -24 \,\,\, \quad ! 24: W+ boson, \   -24: W- boson}\\
{\tt vdecaymode  1 \quad \quad ! 1, 2, 3: W decay to electron, muon, tau}\\


\noindent The mass of the bottom quark, in GeV,  can be set throught the
parameter:\\ 
{\tt bmass \quad \,\,\, 4.75}\\
\noindent If this parameter is not present in the {\tt powheg.input} file,
the default value of 4.2~GeV is assumed.

\noindent The last step performed by the \POWHEGBOXV{} before writing a
generated event on a Les Houches file, is to perform the reshuffling of the
momenta of the final-state particles, in order to give masses to them.  The
user can set these masses in the {\tt init\_couplings.f} file, by changing
{\tt physpar\_mq}, for the quarks, and {\tt physpar\_ml}, for the
charged leptons.\newpage

\noindent Three useful optional process-specific  parameters are the following:

\noindent {\tt Msbarscheme 0 \, ! (default value 1) 1: MSbar renormalisation scheme}\\
\noindent {\tt \phantom{Msbarscheme 0 \, }! \phantom{(default value 1)} 0:  decoupling scheme}\\
\noindent {\tt \phantom{Msbarscheme 0 \, }! \phantom{(default value 1)} PDFs should be changed accordingly}\\

\noindent {\tt minloscfac ~1 \,! (default 1d0) MiNLO scaling factor}\\


\noindent {\tt atlas\_scale \,0 \,! (default 0), 1: use ATLAS scale (see arXiv:1302.2929)}\\

\noindent The first parameter, {\tt Msbarscheme}, allows to switch between
two possible renormalization schemes: the $\overline{\mathrm{MS}}$ and the
decoupling scheme. The default scheme is the $\overline{\mathrm{MS}}$ one
that allows to use standard 5-flavours pdf sets and $\overline{\mathrm{MS}}$
$\as$ running with 5 light flavours. If the decoupling scheme is chosen,
i.e.~{\tt Msbarscheme 0}, the user must choose only 4-flavours pdf sets.
%
The second parameter, {\tt minloscfac}, allows to change the scale of the
{\tt MiNLO} primary process by a constant factor. This parameter is similar
to {\tt renscfac} and {\tt facscfac}, for the renormalization and
factorization scales, respectively.  Finally, the flag {\tt atlas\_scale} can
be used to set the {\tt MiNLO} scale of the primary process to the scale
chosen by the ATLAS Collaboration in ref.~\cite{Aad:2013vka}.

\begin{thebibliography}{10}

\bibitem{Nason:2004rx}
  P.~Nason,
  ``A new method for combining NLO QCD with shower Monte Carlo algorithms,''
  JHEP {\bf 0411} (2004) 040
  [arXiv:hep-ph/0409146].
  %%CITATION = JHEPA,0411,040;%%

%\cite{Frixione:2007vw}
\bibitem{Frixione:2007vw}
  S.~Frixione, P.~Nason and C.~Oleari,
``Matching NLO QCD computations with Parton Shower simulations: the POWHEG
method,''
  JHEP {\bf 0711} (2007) 070
  [arXiv:0709.2092 [hep-ph]].
  %%CITATION = JHEPA,0711,070;%%

%\cite{Alioli:2010xd}
\bibitem{Alioli:2010xd}
  S.~Alioli, P.~Nason, C.~Oleari and E.~Re,
``A general framework for implementing NLO calculations in shower Monte Carlo
  programs: the POWHEG BOX,''
  [arXiv:1002.2581 [hep-ph]].
  %%CITATION = ARXIV:1002.2581;%%


\bibitem{Luisoni:2015mpa}
  G.~Luisoni, C.~Oleari and F.~Tramontano,
  ``$Wb\bar{b}j$ production at NLO with POWHEG+MiNLO'', arXiv:1502.01213.

%\cite{Stelzer:1994ta}
\bibitem{Stelzer:1994ta}
  T.~Stelzer and W.~F.~Long,
  ``Automatic generation of tree level helicity amplitudes,''
  Comput.\ Phys.\ Commun.\  {\bf 81} (1994) 357
  [hep-ph/9401258].
  %%CITATION = HEP-PH/9401258;%%
  
%\cite{Alwall:2007st}
\bibitem{Alwall:2007st}
  J.~Alwall, P.~Demin, S.~de Visscher, R.~Frederix, M.~Herquet, F.~Maltoni, T.~Plehn and D.~L.~Rainwater {\it et al.},
  ``MadGraph/MadEvent v4: The New Web Generation,''
  JHEP {\bf 0709} (2007) 028
  [arXiv:0706.2334 [hep-ph]].
  %%CITATION = ARXIV:0706.2334;%%

%\cite{Cullen:2011ac}
\bibitem{Cullen:2011ac}
  G.~Cullen, N.~Greiner, G.~Heinrich, G.~Luisoni, P.~Mastrolia, G.~Ossola, T.~Reiter and F.~Tramontano,
  ``Automated One-Loop Calculations with GoSam,''
  Eur.\ Phys.\ J.\ C {\bf 72} (2012) 1889
  [arXiv:1111.2034 [hep-ph]].
  %%CITATION = ARXIV:1111.2034;%%

%\cite{Cullen:2014yla}
\bibitem{Cullen:2014yla}
  G.~Cullen, H.~van Deurzen, N.~Greiner, G.~Heinrich, G.~Luisoni, P.~Mastrolia, E.~Mirabella and G.~Ossola {\it et al.},
  ``G$\scriptsize{O}$S$\scriptsize{AM}$-2.0: a tool for automated one-loop calculations within the Standard Model and beyond,''
  Eur.\ Phys.\ J.\ C {\bf 74} (2014) 8,  3001
  [arXiv:1404.7096 [hep-ph]].
  %%CITATION = ARXIV:1404.7096;%%

%\cite{Aad:2013vka}
\bibitem{Aad:2013vka}
  G.~Aad {\it et al.}  [ATLAS Collaboration],
  ``Measurement of the cross-section for W boson production in association with b-jets in pp collisions at $\sqrt{s}$ = 7 TeV with the ATLAS detector,''
  JHEP {\bf 1306} (2013) 084
  [arXiv:1302.2929 [hep-ex]].
  %%CITATION = ARXIV:1302.2929;%%


\end{thebibliography}

\end{document}





